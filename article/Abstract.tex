%------------------------------------------------------%
%- 概要
%--- Abstractは文字数超過するとはみ出るので,
%--- 良い感じに調節して下さい.
%
%--- 生成された枠の一番したの行でおよそ800字ちょいになるように
%--- 文字サイズ・レイアウトを調整しています.
%------------------------------------------------------
現在機械学習の分野では,人工知能や人工無脳に加えDeepleaningを始めとした
革新的な技術が注目を浴びている.

現在人工知能を,開発するだけではなくその開発した人工知能を用いて特定の分野で
応用することにも注目が集まっている.
その例として人工知能を搭載したソフトバンクのロボットPepperなど,各企業が人工知能などを用いた製品を開発している.

そこで本研究では,人工知能の活用を目的とした「キャラクターと対話を行うことのできる
プログラム」の開発を支援するフレームワークを提案する.

既存の人工知能自体の開発を目的としている人工知能のフレームワークでは,人工知能自体を作成する過程
をサポートしている.

しかし本研究では,近年注目を浴びている人工知能を有効的に利用することをサポートするフレームワークを作成した.

このフレームワークを用いることで「キャラクターとの対話を行うプログラム」を簡単に作成することが可能となり,
利用者が発案した対話アルゴリズムを素早く開発し,キャラクターとの対話という形で試すことが可能となる.

キャラクターとの対話を行う事の利点として,人工知能を開発している際に気付きにくい点である「このキャラクター
に実際にこのセリフを言われたときにどの様な気持ちになるだろう」というキャラクターと実際に対話をすることで初めて
わかる情報やフィードバックを得られる点がある.
