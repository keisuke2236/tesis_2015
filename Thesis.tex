%======================================================%
%----- T-lab LaTeX Template File for "Thesis"
%
%--- 必ず README を読んで下さい (by sakura)
%======================================================%

%------------------------------------------------------%
%- Documentclass & Basic setting
%------------------------------------------------------%

\documentclass[a4paper,10pt,onecolumn,oneside,openany]{jsbook}

% Import original package
\usepackage{./conf/cs}
\usepackage{./conf/style}
%======================================================%
%----- T-lab LaTeX Configuration template
%======================================================%

%------------------------------------------------------%
%- Import package
%------------------------------------------------------%

\usepackage{url}						% URL
\usepackage{ifthen}						% 条件分岐


%------------------------------------------------------%
%- if
%------------------------------------------------------%

\newif\ifTaA					% 表紙と概要
\newif\ifToC					% 目次
\newif\ifPDF 					% 画像出力
\newif\ifSRC					% ソースコード


%------------------------------------------------------%
%- Conditional branch
%--- コンパイルしたい条件によって変えて
%------------------------------------------------------%

\TaAtrue					% 表紙と概要を出力するか
%\TaAfalse					% ┗ しない場合
\ToCtrue					% 目次を出力するか
%\ToCfalse					% ┗ しない場合
\PDFtrue					% 画像出力をPDFにするか
%\PDFfalse					% ┗ しない場合

%- ソースコードを使うか (※ README を読む事)
%\SRCtrue					% 使う場合は true に
\SRCfalse					% ┗ 使わない場合


%------------------------------------------------------%
%- Image output setting
%------------------------------------------------------%

\ifPDF
	% PDF image output
	\usepackage[dvipdfmx]{graphicx}
	\usepackage[dvipdfmx]{color}
	\usepackage[dvipdfmx]{colortbl}
\else
	% dviout image output
	\usepackage[dviout]{graphicx}
	\usepackage[dviout]{color}
	\usepackage[dviout]{colortbl}
\fi


%------------------------------------------------------%
%- listings
%------------------------------------------------------%

\ifSRC

	\usepackage{listings}
	\usepackage{jlisting}

	\lstset{
		classoffset=0,
		numbers={left},
		stepnumber={1},
		sensitive={true},
		frame={tRBl},
		framesep={5pt},
		frameround={fttt},
		rulesep = 2pt,
		showstringspaces={false},
		tabsize={2},
		breaklines=true,
		xleftmargin=5mm,
		xrightmargin=3mm,
		%framexleftmargin=6mm,							% 行番号をフレームに入れる
		basicstyle={\ttfamily \footnotesize},
		numberstyle={\scriptsize},
		stringstyle={\ttfamily \color[cmyk]{0,0.8,0,0}},
		commentstyle={\color[rgb]{0,0.5,0}},
		keywordstyle={\ttfamily \color[rgb]{0,0,1}}
	}

	%- Java
	\newenvironment{Java}[0]{
	\lstset{
		language={Java},
		classoffset=1,
		keywordstyle={\ttfamily \color[rgb]{1, 0, 0}},
		morekeywords={
			Louise,
		}
	}}{}

	%- Code paste
	\newcommand{\srcPst}[4]{
		\vspace{3mm}
			\begin{#1} \lstinputlisting[caption=#4 ( #2 ), label=src:#3]{./src/#2} \end{#1}
		\vspace{5mm}
	}

	\newcommand{\srcref}[1]{{\bf \lstlistingname~\ref{src:#1}}}		% 参照
	\renewcommand{\lstlistingname}{{\bf リスト}}					% キャプション
	\renewcommand{\thelstnumber}{\arabic{lstnumber}:}				% 行番号の表示

\fi


%------------------------------------------------------%
%- Renewcommand
%------------------------------------------------------%


%------------------------------------------------------%
%- Define original command
%------------------------------------------------------%

%- Reference
\newcommand{\figref}[1]{{\bf \figurename~\ref{fig:#1}}}
\newcommand{\tabref}[1]{{\bf \tablename~\ref{tab:#1}}}
\newcommand{\equref}[1]{{\bf 式~(\ref{equ:#1})}}

%- Quotation
\newcommand{\dq}[1]{`` #1 ''}
\newcommand{\bdq}[1]{{\bf \dq{#1}}}

%- Bibliography
\newcommand{\bib}[4]{\bibitem{#1} #2 : ``#3''{ }(#4).}
\newcommand{\bibURL}[5]{\bibitem{#1} #2 : ``#3''{ }\url{#4}{ }(#5).}

%- Image paste
\newcommand{\figPst}[3]{
	\vspace{3mm} \begin{figure}[tbh]
		\begin{center}
			\includegraphics[width=#1mm]{./figure/#2.png}
			\caption{#3}\label{fig:#2}
		\end{center}
	\end{figure} \vspace{-2mm}
}


% Define basic information
% - ここら辺を空気読んで編集して
\author{田中 遼}										% 著者
\id{C0109343}											% 学籍番号
\juryoshoid C0109343									% 学籍番号 ({}を付けないで)
\title{学内クラウドの運用・改良および評価}				% タイトル短い人はこっち (\longTitle[AB] は空白に)
\longTitleA{}											% タイトル長い人はこっちに A,B に分けて書く
\longTitleB{}											% (\title は空白に)
\juryoshotitle{学内クラウドの\\運用・改良および評価}		% タイトル (こっちは長い場合は \\ で改行を入れる)
\courseofcs												% ここからはみんな同じ
\clab{田胡}												% 研究室名
\teacher{田胡 和哉}										% 指導教員
\date{2013年01月23日}									% 提出日
\cnendo{2012}											% 提出年度
\nendo{2012年度}										% 提出年度


%------------------------------------------------------%
%- Document
%------------------------------------------------------%

\begin{document}
%------------------------------------------------------%
%- 目次とか色々
%------------------------------------------------------%

% 表紙など
\ifTaA
	\makejuryosho
	\makecover
	\maketitle
	\jabst{
		%------------------------------------------------------%
%- 概要
%--- Abstractは文字数超過するとはみ出るので,
%--- 良い感じに調節して下さい.
%
%--- 生成された枠の一番したの行でおよそ800字ちょいになるように
%--- 文字サイズ・レイアウトを調整しています.
%------------------------------------------------------%

ルいず

	}
	\makejabstract
\fi

% 目次 ・ 図目次 ・ 表目次
\ifToC
	\pagenumbering{roman}
	\setcounter{tocdepth}{3}
	\tableofcontents
	\listoffigures
	\listoftables
	\clearpage
	\pagenumbering{arabic}
\fi


%------------------------------------------------------%

\chapter{序論}
%------------------------------------------------------%
%- 序論
%------------------------------------------------------%

\section{背景}
メイジと呼ばれる魔法使いの貴族と平民が暮らす異世界ハルケギニア,トリステイン王国.
全寮制のトリステイン魔法学院では,いつものように\dq{ルイズ・ド・ラ・ヴァリエール}が授業中に魔法を失敗し,
教室を滅茶苦茶にしている.
{\bf 「魔法属性を持たないゼロのルイズ」}と他の生徒にからかわれる中,プライドが高いルイズはあくまでも強気.

そんなルイズは学校の使い魔召喚の儀式で,他の生徒が次々と召喚獣の呼び出しに成功する中,
こともあろうに平民らしき男の子を呼び出してしまう・・・?!\cite{louise}

\subsection{本テンプレについて}
レイアウトを調整して,表紙などを自動生成出来るようにしたテンプレなんだよ.
\LaTeXe の基本的な使い方は文献\cite{latex}とか {\it Google} で調べよう.
\footnote{やろうと思えば大体は出来るはずだ}

よく使うコマンドはこのテンプレ内でなるべく使っているつもりだけど,
「これがやりたいのに!」 という事があったら調べるか,誰かに聞こう.\\

基本ルールは

\begin{itemize}
	\item{レイアウト関係 (余白など) は統一しよう}
	\item{句読点は{\bf カンマ・ピリオド}で統一しよう}
	\item{chapter > section > subsection > subsubsection の順に章立てしよう}
\end{itemize}

\begin{enumerate}
	\item{慣れないうちはちょくちょくコンパイルしよう(どこでエラーが出るかわからない)}
	\item{基本的に数学記号関係などの記号はエスケープが必要な場合が多い}
	\item{コンパイルできない場合は,\dq{\$ make clean}をしてからコンパイルしてみよう}
	\item{作成者がレポート作成などで使用していたものを弄ったテンプレなので意味不明な定義もある}
	\item{気に食わない点があったら,作成者 \footnote{sakura : dev@saku-lab.net} に文句を言おう}
\end{enumerate}

って感じ☆ミ ウフフ☆おっけー♪

%------------------------------------------------------%

\section{目的}
こなぁぁぁあああああああああああああああああああああああああああああああああああああああああああああ
ぁぁああああああああああああああああああああああああああああああああああああああああああああああああ
ぁぁああああああああああああああああああああああああああああああああああああああああああああああああ
ぁぁああああああああああああああああああああああああああああああああああああああああああああああああ
ぁぁああああああああああああああああああああああああああああああああああああああああああああああああ
ぁぁああああああああああああああああああああああああああああああああああああああああああああああああ
ぁぁああああああああああああああああああああああああああああああああああああああああああああああああ
ぁぁああああああああああああああああああああああああああああああああああああああああああああああああ
ぁぁああああああああああああああああああああああああああああああああああああああああああああああああ
ぁぁああああああああああああああああああああああああああああああああああああああああああああああああ
ぁぁああああああああああああああああああああああああああああああああああああああああああああああああ
ぁぁああああああああああああああああああああああああああああああああああああああああああああああああ
ぁぁああああああああああああああああああああああああああああああああああああああああああああああああ
ぁぁああああああああああああああああああああああああああああああああああああああああああああああああ
ぁぁああああああああああああああああああああああああああああああああああああああああああああああああ
ぁぁああああああああああああああああああああああああああああああああああああああああああああああああ
ぁぁああああああああああああああああああああああああああああああああああああああああああああああああ
ぁぁああああああああああああああああああああああああああああああああああああああああああああああああ
ぁぁああああああああああああああああああああああああああああああああああああああああああああああああ
ぁぁああああああああああああああああああああああああああああああああああああああああああああああああ
ぁぁああああああああああああああああああああああああああああああああああああああああああああああああ
ぁぁああああああああああああああああああああああああああああああああああああああああああああああああ
ぁぁああああああああああああああああああああああああああああああああああああああああああああああああ
ぁぁああああああああああああああああああああああああああああああああああああああああああああああああ
ぁぁああああああああああああああああああああああああああああああああああああああああああああああああ
ぁぁああああああああああああああああああああああああああああああああああああああああああああああああ
ぁぁああああああああああああああああああああああああああああああああああああああああああああああああ
ぁぁああああああああああああああああああああああああああああああああああああああああああああああああ
ぁぁああああああああああああああああああああああああああああああああああああああああああああああああ
ぁぁああああああああああああああああああああああああああああああああああああああああああああああああ
ぁぁああああああああああああああああああああああああああああああああああああああああああああああああ
ぁぁああああああああああああああああああああああああああああああああああああああああああああああああ
ぁぁああああああああああああああああああああああああああああああああああああああああああああああああ
ぁぁああああああああああああああああああああああああああああああああああああああああああああああああ
ぁぁああああああああああああああああああああああああああああああああああああああああああああああああ
ゆきぃいいいいいいいいいいいいいいいいいいいいいいいいいいいいいいいいいいいいいいいいいいいいいいい
いいいいいいいいいいいいいいいいいいいいいいいいいいいいいいいいいいいいいいいいいいいいいいいいいい
いいいいいいいいいいいいいいいいいいいいいいいいいいいいいいいいいいいいいいいいいいいいいいいいいい
いいいいいいいいいいいいいいいいいいいいいいいいいいいいいいいいいいいいいいいいいいいいいいいいいい
いいいいいいいいいいいいいいいいいいいいいいいいいいいいいいいいいいいいいいいいいいいいいいいいいい
いいいいいいいいいいいいいいいいいいいいいいいいいいいいいいいいいいいいいいいいいいいいいいいいいい
いいいいいいいいいいいいいいいいいいいいいいいいいいいいいいいいいいいいいいいいいいいいいいいいいい
いいいいいいいいいいいいいいいいいいいいいいいいいいいいいいいいいいいいいいいいいいいいいいいいいい
いいいいいいいいいいいいいいいいいいいいいいいいいいいいいいいいいいいいいいいいいいいいいいいいいい
いいいいいいいいいいいいいいいいいいいいいいいいいいいいいいいいいいいいいいいいいいいいいいいいいい
いいいいいいいいいいいいいいいいいいいいいいいいいいいいいいいいいいいいいいいいいいいいいいいいいい
いいいいいいいいいいいいいいいいいいいいいいいいいいいいいいいいいいいいいいいいいいいいいいいいいい
いいいいいいいいいいいいいいいいいいいいいいいいいいいいいいいいいいいいいいいいいいいいいいいいいい
いいいいいいいいいいいいいいいいいいいいいいいいいいいいいいいいいいいいいいいいいいいいいいいいいい
いいいいいいいいいいいいいいいいいいいいいいいいいいいいいいいいいいいいいいいいいいいいいいいいいい
いいいいいいいいいいいいいいいいいいいいいいいいいいいいいいいいいいいいいいいいいいいいいいいいいい
いいいいいいいいいいいいいいいいいいいいいいいいいいいいいいいいいいいいいいいいいいいいいいいいいい
いいいいいいいいいいいいいいいいいいいいいいいいいいいいいいいいいいいいいいいいいいいいいいいいいい
いいいいいいいいいいいいいいいいいいいいいいいいいいいいいいいいいいいいいいいいいいいいいいいいいい
いいいいいいいいいいいいいいいいいいいいいいいいいいいいいいいいいいいいいいいいいいいいいいいいいい
いいいいいいいいいいいいいいいいいいいいいいいいいいいいいいいいいいいいいいいいいいいいいいいいいい
いいいいいいいいいいいいいいいいいいいいいいいいいいいいいいいいいいいいいいいいいいいいいいいいいい
いいいいいいいいいいいいいいいいいいいいいいいいいいいいいいいいいいいいいいいいいいいいいいいいいい
いいいいいいいいいいいいいいいいいいいいいいいいいいいいいいいいいいいいいいいいいいいいいいいいいい
いいいいいいいいいいいいいいいいいいいいいいいいいいいいいいいいいいいいいいいいいいいいいいいいいい
いいいいいいいいいいいいいいいいいいいいいいいいいいいいいいいいいいいいいいいいいいいいいいいいいい
いいいいいいいいいいいいいいいいいいいいいいいいいいいいいいいいいいいいいいいいいいいいいいいいいい
いいいいいいいいいいいいいいいいいいいいいいいいいいいいいいいいいいいいいいいいいいいいいいいいいい
いいいいいいいいいいいいいいいいいいいいいいいいいいいいいいいいいいいいいいいいいいいいいいいいいい
いいいいいいいいいいいいいいいいいいいいいいいいいいいいいいいいいいいいいいいいいいいいいいいいいい
いいいいいいいいいいいいいいいいいいいいいいいいいいいいいいいいいいいいいいいいいいいいいいいいいい
いいいいいいいいいいいいいいいいいいいいいいいいいいいいいいいいいいいいいいいいいいいいいいいいいい
いいいいいいいいいいいいいいいいいいいいいいいいいいいいいいいいいいいいいいいいいいいいいいいいいい
ねぇ.


%------------------------------------------------------%

\chapter{クラウド・コンピューティング}
%------------------------------------------------------%
%- クラウド・コンピューティング
%------------------------------------------------------%

\section{クラウド・コンピューティングとは}
クラウドって何なんだろうね.俺にもよくわかんないや.

\subsection*{てーぶるとか図とか}
次の\tabref{example}に色々と示す.

\begin{table}[tbh]
	\caption{可愛いのか} \label{tab:example}
	\begin{center}
		\begin{tabular}[htb]{c|c}
		\hline
		名前 & 印象 \\
		\hline
		ルイズ・フランソワーズ・ル・ブラン・ド・ラ・ヴァリエール & かわいい \\ \hline
		タバサ & かわいい \\ \hline
		キュルケ・アウグスタ・フレデリカ・フォン・アンハルツ・ツェルプストー & しらん \\ \hline
		シエスタ & 神 \\ \hline
		アンリエッタ & かわいい \\ \hline
		\end{tabular}
	\end{center}
\end{table}

次の\figref{t-lab}に何か示す.

\figPst{70}{t-lab}{田胡研のロゴ的な何か}

\dq{make}でコンパイルしてる人はルールに書いてあるからそのままでいい.
\dq{make}を使ってない人は図のあるディレクトリにいって \dq{\$ ebb *.png} や \dq{\$ xbb *.png} を実行しないと
サイズがおかしくなるよ.


\ifSRC
	\subsection*{ソースコードとか}
	ソースコードはこんな感じで出せるよ(\srcref{louise}).

	\srcPst{Java}{Louise.java}{louise}{ルイズコピペ}
\fi

おなか減ったなぁ.


%------------------------------------------------------%

% ... 章ごとに書くと楽だよ

%------------------------------------------------------%

\chapter*{謝辞}
\addcontentsline{toc}{chapter}{謝辞}
%------------------------------------------------------%
%- 謝辞
%------------------------------------------------------%

本研究を進めるにあたり,数々のご意見とご指導をくださった田胡和哉教授と柴田千尋助教に心より感謝いたします.
また,本研究への助言をくださった田胡研究室の皆様と開発をする際に参考にさせていただいた文献を書いた方々に感謝いたします.
加えてインタビューに答えていただいたコメダ珈琲八王子店のスタッフ方々に感謝いたします.


%------------------------------------------------------%
%- References
%------------------------------------------------------%

\begin{thebibliography}{99}
	\bibURL{louise}{ゼロの使い魔制作委員会}{ゼロの使い魔公式ウェブサイト}{http://www.zero-tsukaima.com/zero/index.html}{2012/12/28}
	\bib{latex}{奥村晴彦 著}{\LaTeXe 美文書作成入門 改訂第3版}{技術評論社 2004, 403pp}
\end{thebibliography}

\end{document}
