%------------------------------------------------------%
%- 結論
%------------------------------------------------------%
\section{結論}
今回作成した人工知能利用フレームワークを用いて,考えたアルゴリズムを素早く実装し
キャラクターとの会話で試すことができた.
このように既存のフレームワークにはない人工知能を利用する事に注目したフレームワークを作った事により,
対話システムを作る際の様々な手間や準備があるといった問題を解決する事ができた.

しかし,話題が固定された状況で返答アルゴリズムを書く必要があり,
どの話題でどのアルゴリズムが回答するかを選択する部分のアルゴリズムも含めて作成したい場合は,
他のアルゴリズムを登録する事が出来ないのが現状であり,
その場合既存の他のアルゴリズムが利用できなくなるという弊害があるのが現状である.

\subsection{アルゴリズムの追加による出力の変化}
\ref{sec:addAl}章の実行結果を見て分かる通りアルゴリズムを追加することでその分野の話題に
なった時に,追加したアルゴリズムが解析や応答をしていることが分かる.
また解析を行うアルゴリズムの追加に関しても,その分野のさらに詳しい話題の分析が\ref{sec:addAl}
の返答を見て分かる通り可能となった.

以上のようにアルゴリズムを追加することにより,より高精度な解析や出力を行うことができた.

\subsection{簡単にアルゴリズムを追加できたか}
考案したアルゴリズムを素早く試すために,アルゴリズムを追加する際の構造を簡単化したため,
解析や出力のアルゴリズムの作成と追加を非常に簡単に実現できる機構を作ることができた.

会話を行うことができるアルゴリズムを3行のプログラムソースコードの記述で実現する事が可能であり,
非常に開発を開始するまでの時間を短くすることが出来たと考えられる.

またそれに加えてキャラクターと対話をする事が可能なため,この会話内容を本物のキャラクターに言われたらどの様に
感じるかもシミュレーションを行うことが出来た.
これにより文字だけでの出力しか行えない場合よりも,
よりリアルなコミュニケーションを行う人工知能や人工無脳の作成を目指すことができることがわかった.
