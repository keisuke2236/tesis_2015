%------------------------------------------------------%
%- 概要
%--- Abstractは文字数超過するとはみ出るので,
%--- 良い感じに調節して下さい.
%
%--- 生成された枠の一番したの行でおよそ800字ちょいになるように
%--- 文字サイズ・レイアウトを調整しています.
%------------------------------------------------------
現在機械学習などの分野では,人工知能や人工無脳,また,Deepleaningを始めとした
革新的な技術が注目を浴びている.

現在、その人工知能を開発するだけではなく,その開発した人工知能を用いて特定の分野で
応用することにも注目が集まっており,人工知能を搭載したソフトバンクのロボットPepperなど,
各企業が人工知能やを用いた製品を開発している.
そこで本研究では人工知能の活用を目的としてキャラクターと対話を行うことのできる
プログラムの開発を支援する人工知能を利用するためのフレームワークを開発した.

既存の人工知能自体の開発を目的としている,人工知能フレームワークでは人工知能自体を作成する過程
をサポートしているが,
本研究では,近年注目を浴びている人工知能を有効的に利用することをサポートするフレームワークを作成した.
このフレームワークを用いることで,人工知能の利用例の1つであるキャラクターとの対話プログラムを簡単に
作成することが可能となり,利用者が発案した対話アルゴリズムをすぐに開発し,キャラクターとの対話
という形でためすことができる.

また,キャラクターとの対話を行うことで,人工知能を開発している際に気付きにくい点である,このキャラクター
に実際にこのセリフを言われたときに,どの様な気持ちになるだろうというキャラクターを目の前に対話をすることで初めて
わかる情報,フィードバックをシミュレートし,開発者に伝えることができる点も,
この人工知能利用フレームワークの特徴の1つである.