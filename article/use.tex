%------------------------------------------------------%
%- 提案
%------------------------------------------------------%
\section{開発した人工知能の活用}
今回提案するのは先ほど説明した一般的な人工知能フレームワークを用いて開発を行った
人工知能やその返答アルゴリズムを活用するためのフレームワークである.
%------------------------------------------------------%
\subsection{知能の開発をサポートする既存フレームワーク}
既存の人工知能フレームワークは,人工知能自体を作成することをサポートしており,
その作成した人工知能を用いて会話を行う.\\
%------------------------------------------------------%
\section{開発した知能を試す環境}
今回提案するのは,フレームワークを用いて開発したアルゴリズムや,
独自のアルゴリズムを考え,作成したプログラムを実際に動かし試す環境である.
\\
既存のフレームワークは人工知能を作ることに着目して,作る工程をサポートするものが多い,
今回提案する人工知能利用フレームワークでは考案して,作成したアルゴリズムや人工知能を
複数登録することでUnity上で動作するキャラクターと会話を行うことができるシステムである.
\\
通常,人工知能のアルゴリズムを試したいと考えた場合,
そのプログラムに対して入力を与える入力の部分と
その処理結果を出力する出力の部分を作成する必要がある.
\\
作成した知能の出力結果がただ単に文字で入力して,文字で出力されれば良い場合は,
出力画面を準備をするのは手間が不要であるが,キャラクターとの会話などで試したい場合,
非常に入出力の部分を作成するのに手間と時間がかかるという問題点がある.
\\
その部分をあらかじめ人工知能利用フレームワークで提供することで,
準備の手間が不要になる.
\\
また,キャラクターとの会話などで出力することで対話をする際に,
実際にそのキャラクターに言われたらどの様に感じるかをシミュレーションすることが
できるため,よりリアルなコミュニケーションを行う人工知能や人工無脳を目指して,
アルゴリズムを考え,開発することが可能になるという利点もある.
\\
今回はその様な人工知能を利用することに着目したフレームワークを提案する.
%------------------------------------------------------%
\section{人工知能利用フレームワークの提案}
人工知能利用フレームワークは会話や動作などの返答アルゴリズムを作成した際に,
それらの作成したプログラムをフレームワーク上に適当に配置することで,
状況や話題に応じて適切な作成した返答アルゴリズムが選択され
Unity上のキャラクターと会話を楽しむことができる,
人工知能を利用することに焦点を当てたフレームワークである.
%------------------------------------------------------%
\subsection{提案する全体構成}\label{sec:allAr}
この人工知能利用フレームワークの全体の構成を次の\figref{all_kose}に示す.

\figPst{90}{all_kose}{全体の構成図}

提案する人工知能ハブは大きく分けて3つの要素で構成され,大きな流れで説明をすると
Unityでユーザーが入力した内容をもとに人工知能活用ハブがその入力内容を受け取る.
そして人工知能活用ハブの中で返答する内容が作り出され,モーションデータベースから適切な動作を選択
しUnityへ動作と返答内容を出力する.
この流れによってユーザーはキャラクターとの会話を行うことが可能となっている.
%------------------------------------------------------%
\subsection{アルゴリズムのみを簡単に追加可能な知能ハブ}
それではまずはじめに私が開発する,アルゴリズムのみを簡単に追加することができる
人工知能活用ハブを提案する.
\\
このハブでは作成した会話の返答,もしくはキャラクターの動作を選択するアルゴリズムを簡単に追加し
話題によって追加したアルゴリズムの中から適切なアルゴリズムを用いて返答を行えるようにしている.
\\
例えばゲーム関連の返答アルゴリズムを作り,試したいと考えた場合は,
そのアルゴリズムを実装したプログラムをあらかじめ準備されている抽象クラスを用いて作成し,
\figref{all_kose}の返答アルゴリズム軍のプログラムに作成したプログラムを登録するだけで,
ゲームの話題が来た時にそのアルゴリズムでキャラクターが返答するシステムを作ることができる
というものである.
\\
同様にゲーム関連のキャラクターの動作を選択するアルゴリズムを作る場合は,そのアルゴリズムを
抽象クラスを用いて実装することが可能であり,
\figref{all_kose}の動作選択アルゴリズム軍の中に作成したプログラムを登録するだけで,ゲーム関連
の会話をしている最中は,そのアルゴリズムを用いて動作を決定する仕組みを作ることができるというものである.
\\
これらのアルゴリズムや人工知能が一つだけ実装されており,何も追加知能がない場合はその
デフォルトアルゴリズムが選択されるように設計し,複数の料理の話題に特化した話題解析アルゴリズム
やゲームの話題に特化した感情解析のアルゴリズムが実装されることでより正確な解析が可能になるだけで
はなく,返答する際もゲーム専用の返答アルゴリズムなどがあることでより円滑なコミュニケーションが
可能になるような構成を提案する.

また,様々なアルゴリズムが必要になることを考え,複数人で開発を行った際にも解析情報のデータベースによる
共有などにより,よりスムーズに連携を行うことができるほか,
人工知能ハブでは,すでに解析した感情情報などの情報は全てデータベースによって共有され,
ユーザーの入力した情報の解析を行うプログラムの開発は行わずに,すでにある感情解析プログラムの
解析結果を使ってユーザーの感情状態を考慮した「会話ボット」などの開発を行うことも可能にする.\\
%------------------------------------------------------%
\subsection{作成したアルゴリズムをUnityですぐに試せる機構}
この人工知能利用フレームワークの人工知能活用ハブに登録された知能はUnity上でのキャラクターとの
対話ですぐに試すことができる.\\

共同研究者の藤井克成の論文\cite{fuji}によると,MMDモデルを利用しているため好きなキャラクターで動作させることが
出来,また,この人工知能利用フレームワークのために開発した,リアルタイムに動作を保管しながら
動かす技術により,よりリアルな円滑なコミュニケーションが可能になっている.\\

このように作成した人工知能をすぐにキャラクターとの対話という形で実行することができるため,
入出力をどのような設計にするかや,開発はどうするかに迷うことなく,独自の対話アルゴリズムや
人工知能の開発に専念することが可能になり,より高精度な対話を実現できると提案する.\\

%------------------------------------------------------%
\subsection{Unityが利用可能なモーションを追加する機構}
この人工知能利用フレームワークでは現在会話と動作の2つの出力を実装している.\\

ここで返答パターンは文字列で生成され,Unityで実行されるので無限のバリエーションで返答すること
ができるが,動作(モーション)はその場で動作を生成することが難しいのが現状である.\\

そのため,あらかじめモーションデータを作る必要となるのですが,
そのモーション(動作)を定義するファイルを生成することは一般的には難しいと考えられる,
そこで共同開発の鈴木智博がKinectで動作を定義し,データベースに保存,人工知能活用ハブと通信可能な
プログラムを開発した\cite{suzuki}.\\

この機構があることによって,人工知能活用ハブの中で新しい動きのパターンを追加したいとなった時にも
すぐにKinectを用いて動作ファイルを生成し,データベースに登録することで使えるようなる.

