%------------------------------------------------------%
%- 概要
%--- Abstractは文字数超過するとはみ出るので,
%--- 良い感じに調節して下さい.
%
%--- 生成された枠の一番したの行でおよそ800字ちょいになるように
%--- 文字サイズ・レイアウトを調整しています.
%------------------------------------------------------

現在機械学習の分野では人工知能や人工無脳,Deepleaningを始めとした
革新的な技術が注目を浴びている.
加えて人工知能を開発するだけではなく,その開発した人工知能を様々な分野で応用することにも注目が集まっている.
その例としてMicrosoftが開発を行ったりんなや人工知能を搭載したソフトバンクのロボットPepperなどがあり,
各企業が人工知能などを用いた製品に力を入れて開発している.

現在人工知能自体の開発を目的としている人工知能のフレームワークでは,人工知能自体を作成する過程
をサポートしており,作成した人工知能を応用する部分を支援しているフレームワークは非常に少ない.
そこで本研究では近年注目を浴びている「人工知能を有効的に利用する」ということを支援する事を目的とした,
「キャラクターと対話を行うことのできるプログラム」の開発を支援するフレームワークを提案する.

このフレームワークを用いることで「キャラクターとの対話を行うプログラム」を簡単に作成することが可能となり,
利用者が発案した対話アルゴリズムを素早く開発し,キャラクターとの対話という形で試すことが可能となる.

フレームワークの開発は必要な機能や全体の設計については教授の指導のもとで開発を行い,
開発環境の作成や実際の実装は同じ研究室のメンバーにアドバイスを頂きながら行った.

フレームワークを開発した結果キャラクターとの対話を行う事の利点として,人工知能を開発している際に気付きにくい点である「このキャラクター
に実際にこのセリフを言われたときにどの様な気持ちになるだろう」というキャラクターと実際に対話をすることで初めて
わかる情報やフィードバックを得られる点があることがわかった.
