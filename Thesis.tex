\documentclass[a4paper,10pt,onecolumn,oneside,openany]{jsbook}

% Import original package
\usepackage{./conf/cs}
\usepackage{./conf/style}
\input{./conf/conf}

% Define basic information
% - ここら辺を空気読んで編集して
\author{寺田 佳輔}										% 著者
\id{C0112336}											% 学籍番号
\juryoshoid C0112336									% 学籍番号 ({}を付けないで)
\title{人工知能利用フレームワークの開発}				% タイトル短い人はこっち (\longTitle[AB] は空白に)
\longTitleA{}											% タイトル長い人はこっちに A,B に分けて書く
\longTitleB{}											% (\title は空白に)
\juryoshotitle{人工知能利用フレームワークの開発}		% タイトル (こっちは長い場合は \\ で改行を入れる)
\courseofcs												% ここからはみんな同じ
\clabA{田胡}
\clabB{柴田}												% 研究室名
\teacher{田胡 和哉}										% 指導教員
\date{2016年01月19日}									% 提出日
\cnendo{2015}											% 提出年度
\nendo{2015年度}										% 提出年度


%------------------------------------------------------%
%- Document
%------------------------------------------------------%

\begin{document}
%------------------------------------------------------%
%- 目次とか色々
%------------------------------------------------------%

% 表紙など
\ifTaA
	\makejuryosho
	\makecover
	\maketitle
	\jabst{
		%------------------------------------------------------%
%- 概要
%--- Abstractは文字数超過するとはみ出るので,
%--- 良い感じに調節して下さい.
%
%--- 生成された枠の一番したの行でおよそ800字ちょいになるように
%--- 文字サイズ・レイアウトを調整しています.
%------------------------------------------------------

現在機械学習の分野では人工知能や人工無脳,Deepleaningを始めとした
革新的な技術が注目を浴びている.
加えて人工知能を開発するだけではなく,その開発した人工知能を様々な分野で応用することにも注目が集まっている.
その例としてMicrosoftが開発を行ったりんなや人工知能を搭載したソフトバンクのロボットPepperなどがあり,
各企業が人工知能などを用いた製品に力を入れて開発している.

現在人工知能自体の開発を目的としている人工知能のフレームワークでは,人工知能自体を作成する過程
をサポートしており,作成した人工知能を応用する部分を支援しているフレームワークは非常に少ない.
そこで本研究では近年注目を浴びている「人工知能を有効的に利用する」ということを支援する事を目的とした,
「キャラクターと対話を行うことのできるプログラム」の開発を支援するフレームワークを提案する.

このフレームワークを用いることで「キャラクターとの対話を行うプログラム」を簡単に作成することが可能となり,
利用者が発案した対話アルゴリズムを素早く開発し,キャラクターとの対話という形で試すことが可能となる.

フレームワークの開発は必要な機能や全体の設計については教授の指導のもとで開発を行い,
開発環境の作成や実際の実装は同じ研究室のメンバーにアドバイスを頂きながら行った.

フレームワークを開発した結果キャラクターとの対話を行う事の利点として,人工知能を開発している際に気付きにくい点である「このキャラクター
に実際にこのセリフを言われたときにどの様な気持ちになるだろう」というキャラクターと実際に対話をすることで初めて
わかる情報やフィードバックを得られる点があることがわかった.

	}
	\makejabstract
\fi

% 目次 ・ 図目次 ・ 表目次
\ifToC
	\pagenumbering{roman}
	\setcounter{tocdepth}{3}
	\tableofcontents
	\listoffigures
	\listoftables
	\clearpage
	\pagenumbering{arabic}
\fi


%------------------------------------------------------%

\chapter{現状}
%------------------------------------------------------%
%- 現状
%------------------------------------------------------%
近年人工知能などの分野が注目を浴びており,ここではそれらの現状について説明する.

\section{人工知能}
人工知能とは人工的にコンピュータ上などで人間と同様の知能を実現させようという試み,
或いはそのための一連の基礎技術を指すものであり,1956年にダートマス会議でジョン・マッカーシーにより命名された.
人工知能の定義は未だ不確定な部分が多く,完全に正確な定義は存在していないのが現状である.
この人工知能は人工無脳とは異なりただ単にキーワードを広うだけではなく,
機械学習によって取得した情報を用いて解析や情報の取得を行い,状況に応じた返答などが可能なプログラムのことを指すことが多い.
また学習を行わず特定のキーワードを拾い返答するものを,人工無脳または対話ボットという.

\section{機械学習}
機械学習とは入力されたデータから反復的に特徴を学習することでそこに潜むパターンを見つけ出す技術のことであり,
人工知能には必要不可欠な要素である.

\section{Neural Network}
Neural Networkとは,脳細胞を構成する「Neuron(ニューロン)」の活動を単純化したモデルであり,これを利用する
ことによって人間の思考をシミュレーションすることができる.
現在Neural NetWorkは主にDeepLearningで利用されている.

\section{DeepLearning}
DeepLearning\cite{deep}とは多層構造のニューラルネットワークの機械学習の事であり,
ニューラルネットワークを多層積み重ねたモデルを機械学習させればディープラーニングとなる.
また,一般的には3層以上のニューラルネットワークがあるものとされている.

第一層から入った情報は,より深い3層へと行く過程で学習が行われ,その学習を行っていくことで
概念を認識する特徴量と呼ばれる重要な変数を自動で発見することができる.

このDeepLearningにより,東京大学の松尾豊准教授\cite{boom}を始めとする機械学習や人工知能の研究者は
「AI研究に関する大きなブレイクスルーであり、学習方法に関する技術的な革新である」と指摘している.

\section{一般的な人工知能開発フレームワーク}
現在一般的な人工知能を開発するフレームワークとしてchainerやGoogleのTensorFlowなどがある.

Chainerは,Preferred Networksが開発したニューラルネットワークを実装するためのライブラリであり,
人工知能自体の開発を行う際に高速な計算が可能なことや,様々なタイプのニューラルネットを実装可能であり,
またネットワーク構造を直感的に記述できる利点がある.

GoogleのBrain Teamの研究者たちが作った機械学習ライブラリであるTensor Flowは,
Python APIとC++インターフェイス一式が用意されているため開発を行う際に非常に有効だと考えられる.

これらの人工知能開発フレームワークは実際に開発を行うときに非常に有効であり,googleの検索アルゴリズムや
データ分析などの様々な分野での応用を試みる動きがある.

\section{知能の開発をサポートする既存フレームワークの現状}
上記で説明したようなフレームワークが登場する前までは,人工知能の開発は非常に手間と時間のかかるものであった.
それに加えてchainerはGoogleが開発を行った非常に高度な技術を用いるフレームワークであり,
個人でこのような機構を持った人工知能の開発を行いたいと考えたとしても事実上不可能であるのが現状である.

しかし様々なフレームワークの登場により人工知能自体を開発する事が非常に簡単になった事に加え,
より高度な技術を用いた人工知能の開発を行う事が可能になった.
このように個人でも高度な人工知能の開発が可能になった事により,現在非常に人工知能の分野が注目を浴びている.

しかし既存のフレームワークには,開発した人工知能を用いてキャラクターと会話を行うというところまでをサポートするフレームワークがない.
キャラクターとの対話で人工知能を試したい場合は対話を行うアルゴリズムを作成するのに加えて,
キャラクターとの対話を行うインターフェイスの準備やキャラクターを動作させるための動作ファイルなど様々な準備が必要となるのが現状である.

このように既存のフレームワークを用いて簡単に高性能なアルゴリズムを作ったとしても,それを気軽に楽しむ為の環境がないのが現状である.

\newpage

\section{人工知能のアプリケーションへの応用}
現在Microsoftのりんな\cite{rinna}やソフトバンクのpepper\cite{pepper}などの登場により,人工知能を応用した対話に注目が集まっている.

Micsoroftのりんなは,Microsoftが開発を行った人工知能であり,話しかけると女子高生のような返答を返してくれるものである.
現在LINE\footnote{LINEとは韓国最大のIT企業「NHN」の日本法人「LINE株式会社」が提供しているスマートフォン(iPhoneやAndroid)、ガラケー(フィーチャーフォン)、パソコンに対応したコミュニケーションアプリケーションです.\cite{line}}
アプリ上で公式アカウントを持っており,りんなを友達登録しているユーザー数は2016年1月13日現在では2,167,730人である.
これは日本の人口が現在1億2688万人\cite{humen}なので総人口の1.7\%にあたり,およそ100人に1人または2人がこの人工知能りんなとの対話を
行っているというのが現状である.

ソフトバンクのpepperに関しては2016年現在,テレビCMなども頻繁に行われており,
ソフトバンクショップを初めとして一般家庭や喫茶店などで幅広く活用されている.

八王子市の喫茶店コメダ珈琲店\cite{coffe}のスタッフに「pepper君を導入してから何か変わりましたか?」とインタビューを行ったところ,
スタッフからは「pepperを導入してからお子様が来店された時に喜び,pepper君と話している姿をみます」といった声や
「店内が混雑している時の待ち時間にお客様がお話ししており,待ち時間の退屈さを紛らわせてくれている」といった意見を聞く事が出来た.

\figPst{90}{pepper}{東京八王子市のコメダ珈琲店で活躍するpepper君}

\figref{pepper}のように珈琲店では帽子をかぶり来店されたお客様とコミュニケーションを行っている.
また私がpepper君の見える席に座っていたところ,来店される方とpepper君の目が会うたびに人々が笑顔になったり,
驚いていたりしていたのに加えてお年寄りの方が親身にpepper君に話しかけている光景を目にしており,
人々の人工知能との対話への抵抗がなくなってきていることがわかった.
特に今後生まれてくる人々は人工知能との会話が普通のこととなっていくことが予想できる.

このようにテキストチャットの形式で対話を行う事ができるりんなの登場や人工知能を活用したロボットpepperの登場によって人工知能
に非常に注目が集まっており,このような対話システムを企業だけではなく個人でも開発したいという需要は高まっていると考えられる.

これに加えて日本にはオタク
\footnote{おたく(オタク、ヲタク)とは、1970年代に日本で誕生した呼称であり大衆文化の愛好者を指す。元来はアニメ・SF・パソコンなどの、
なかでも嗜好性の強い趣味や玩具、の愛好者の一部に使われていた術語であったが、バブル景気期に一般的に知られはじめた。\cite{ota}}
という文化がある.
ディップ株式会社の調査によると日本人口の約40%はオタクであるという調査結果となっており,非常にアニメやゲームやパソコンなどに
興味を持っている人口が多い事がわかる.

中でもアニメに関しては毎年市場が伸びており,2015年度に株式会社メディア開発綜研が調査した結果によるとその市場は2595
億円(前年比106.9\%)と過去最大の規模になったという.\cite{anime}

このように現在日本では,人工知能を搭載したロボットや女子高生人工知能との会話,アニメが注目されている事がわかった.

そしてなぜアニメがここまで注目されているのか,それはアニメキャラクターのかっこよさや可愛さといった現実にはない二次元キャラクター特有の
”人々の理想”がそこにあったからだと現状私は分析している.

\figPst{90}{hime}{理想の見た目のキャラクターとの理想の会話例\cite{hime}}

そこで\figref{hime}のように理想の見た目をしているキャラクターに,
理想の会話をしてくれる人工知能を加えたら間違いなくこのオタクの人々を感動させる事ができるという仮説を立てた.

今回はこの仮説をもとに二次元のアニメキャラクターの人工知能を気軽に開発し,二次元のキャラクターと気軽に対話を行うところまでをサポートする
フレームワークを提案する.




%------------------------------------------------------%

\chapter{人工知能利用フレームワークの提案}
%------------------------------------------------------%
%- 提案
%------------------------------------------------------%
\section{開発した人工知能の活用}
今回提案するのは先ほど説明した一般的な人工知能フレームワークを用いて開発を行った
人工知能やその返答アルゴリズムを活用するためのフレームワークである.
%------------------------------------------------------%
\subsection{知能の開発をサポートする既存フレームワーク}
既存の人工知能フレームワークは,人工知能自体を作成することをサポートしている.
その作成した人工知能自体を試す場合,ユーザーはキーボードを使い発言内容を記述する
必要がある.

%------------------------------------------------------%
\section{開発した知能を試す環境}
今回提案するのは,開発を行うフレームワークを用いて開発したアルゴリズムや,
独自のアルゴリズムを実際に動かし試す環境を簡単に構築するフレームワークである.

既存の開発を行うことに特化しているフレームワークでは,人工知能を作ること自体に着目し
人工知能の作成を行うプロセスをサポートしている.
しかし今回提案する人工知能利用フレームワークでは,考案したアルゴリズムを
1つの人工知能ハブに対して複数登録することでUnity上で動作するキャラクターと
会話を行うことができる人工知能を利用することをサポートするフレームワークである.

通常,人工知能のアルゴリズムを試す場合,そのプログラムに対してユーザーが入力を行う入力の部分と
その処理結果を出力する出力画面や出力を行うキャラクターの部分を作成する必要がある.

作成したアルゴリズムの出力結果が文字で出力されれば良い場合は,
出力画面を準備をする手間は不要である.
しかしキャラクターとの会話で最終的には試したいと考えた場合,
入出力の部分を作成する際に非常に手間と時間がかかるため,
その実行環境とユーザーが入力を行う部分を予め人工知能利用フレームワークで提供することで,
その準備の手間が不要になるという利点がある.

キャラクターと会話を実際にすることで,実際にキャラクターに言われたらどの様に感じるかを
シミュレーション,フィードバックすることが可能であり,
よりリアルなコミュニケーションを行うアルゴリズムを,開発することを
サポートすることが可能になる.

今回はその様な人工知能を利用することに着目したフレームワークを提案する.
%------------------------------------------------------%
\section{人工知能利用フレームワークの提案}
人工知能利用フレームワークは会話や動作などの返答アルゴリズムを作成した際に,
それらの作成したアルゴリズムをフレームワーク上に適当に記述することで,
状況や話題に応じて適切な作成した返答アルゴリズムが選択され
Unity上のキャラクターと会話を楽しむことができる,
人工知能を利用することに焦点を当てたフレームワークである.
%------------------------------------------------------%
\subsection{提案する全体構成}\label{sec:allAr}
この人工知能利用フレームワークの全体の構成を次の\figref{all_kose}に示す.

\figPst{90}{all_kose}{全体の構成図}

提案する人工知能利用フレームワークにはUnityで作成したキャラクターを出力する部分,
キャラクターに行わせる動作を考える人工知能ハブ,及びUnity上で利用するためのモーションを保存する
モーションデータベースの3つから構成される.

今回私が担当し,作成した人工知能ハブについて解説を行う.
人工知能ハブは大きく分けて3つの要素で構成され,
Unityでユーザーが入力した内容をもとに人工知能活用ハブがその入力内容を受け取り解析を行う部分,
解析した情報を保存するためのデータベース及び返答する内容が作り出される部分である.
動作を選択する部分は,共同研究の鈴木が作成したモーションデータベースから適切な動作を選択し,
Unityへ動作と返答内容を出力する.
%------------------------------------------------------%

\subsection{アルゴリズムのみを簡単に追加可能な知能ハブ}
アルゴリズムのみを簡単に追加することができる人工知能活用ハブを提案する.

人工知能活用ハブでは,作成した会話の返答キャラクターの動作を選択するアルゴリズムを簡単に追加する
ことができる.
追加したアルゴリズムにそれぞれ話題を設定することで
ユーザーが話しかけた話題に応じて適切なアルゴリズムを用いて返答を行うことができるようにすることを提案する.
例えばゲーム関連の返答アルゴリズムを作る場合は,
そのアルゴリズムをあらかじめ準備されている抽象クラスを用いて実装し,
\figref{all_kose}のプログラムに抽象クラスを実装したプログラムを登録することで,
ゲームの話題が来た時にそのアルゴリズムでキャラクターが返答するシステムを作ることが可能である.

同様にゲーム関連のキャラクターの動作を選択するアルゴリズムを作る場合は,そのアルゴリズムを
抽象クラスを用いて実装し,
\figref{all_kose}の動作選択アルゴリズム軍の中に作成したプログラムを登録するだけで,ゲーム関連
の会話をしている最中は,そのアルゴリズムを用いて動作を決定する仕組みを作ることができるというものである.

これらのアルゴリズムや人工知能が1つだけ実装されている場合は,
デフォルトアルゴリズムが選択されるように設計ている.
複数の料理の話題に特化した話題解析アルゴリズムやゲームの話題に特化した感情解析のアルゴリズムが
実装されることで,様々な話題に特化したより正確な解析が可能になる.
返答を作成する際も,ゲーム専用の返答アルゴリズムや料理に特化した返答アルゴリズムがあることでより
円滑なコミュニケーションがを行うことが可能になると提案する.

今後様々なアルゴリズムが必要になるため,複数人で開発を行うことが想定される.
その際にも解析情報をデータベースを用いて共有することで効率的に開発ができるようになっている.
人工知能ハブでは,すでに解析した感情情報などの情報は全てデータベースによって共有されている.
そのため,入力情報の解析を行うプログラムの開発は行わずに,すでにある感情解析プログラムの
解析結果などの情報を使いユーザーの感情状態を考慮した「会話ボット」などの開発を行うことも可能となる.\\
%------------------------------------------------------%
\subsection{作成したアルゴリズムをUnityですぐに試せる機構}
この人工知能利用フレームワークの人工知能活用ハブに登録されたアルゴリズムはキャラクターとの
対話ですぐに試すことができる.

共同研究者の藤井克成の論文\cite{fuji}によると,MMDモデルを利用しているため,モデルを入れ替えることで
好きなキャラクターで動作させることが可能である.
また,人工知能利用フレームワークのために開発したリアルタイムに動作を補完しながらキャラクターを
動かす技術により,よりリアルなコミュニケーションが可能となっている.

作成した人工知能を,すぐにキャラクターとの対話という形で実行することができるため,
入出力の設計や,開発はどうするかに迷うことなく,独自の対話アルゴリズムや
人工知能の開発に専念することが可能になる.
%------------------------------------------------------%
\subsection{Unityが利用可能なモーションを追加する機構}
この人工知能利用フレームワークでは現在会話と動作の2つの出力を実装している.
返答パターンは文字列で生成され,Unityで実行されるので様々なバリエーションで返答すること
ができる.
しかしキャラクターの動作は,動的にプログラムを用いて動作を生成することが難しいため,
予めモーションデータを作成しておく必要性がある.

そのモーションデータを定義するファイルを生成することも手間と時間がかかる.
そこで共同開発の鈴木智博がKinectで動作を定義し,データベースに保存,人工知能活用ハブと通信可能な
プログラムを開発した\cite{suzuki}.

この機構があることによって,人工知能活用フレームワークの中で新しい動きのパターンを追加したい場合,
すぐにKinectを用いて自ら追加したい動作をキャプチャすることで動作ファイルが生成され,
データベースに登録することで使えるようなる.

%------------------------------------------------------%

\chapter{設計}
%------------------------------------------------------%
%- 構成
%------------------------------------------------------%
それでは人工知能活用ハブの構成を3段階に分けてと,Unityのキャラクターとの連携,
解析するアルゴリズムを選定する時に利用しているGoogkeAPIの3つに分けて,
人工知能利用フレームワークについて解説を行いたいと思います.
\\
\section{解析アルゴリズムの追加を可能にする機構}
まず初めにUnityのキャラクターから受け取った情報を解析する際の機構の構成について解説したいと
思います.\\
こちらの解析アルゴリズムの追加ができることで,
\subsection{解析する内容別にプログラムを保持する機能}
\subsection{会話の話題別に解析するアルゴリズムを選ぶ機能}
\subsection{解析アルゴリズムを簡単に追加する機能}

\section{解析した情報を共有する機能}
\subsection{解析情報を保存する機能}
\subsection{解析情報を取得する機能}

\section{返答アルゴリズムの追加を可能にする機構}
\subsection{返答を行うタイミング}
\subsection{返答する内容別にアルゴリズムのを保持する機能}
\subsection{会話の話題別に返答アルゴリズムを選ぶ機能}

\section{作成した知能をUnityで試す機構}
\subsection{Unityでの出力について}
\subsection{Unityとの連携に利用するWebSocket}
\subsection{Unityへの送信フォーマットと作成}
\subsection{Unityからの受信フォーマット}

\section{アルゴリズムを選定する際に用いるGoogleAPI}
\subsection{GoogleAPIについて}
\subsection{GoogleAPIの有効性}


%------------------------------------------------------%
\chapter{実装}
%------------------------------------------------------%
%- 実装
%------------------------------------------------------%
\section{開発環境}
\subsection{Javaの利用}
今回の開発では,実行が高速かつオブジェクト指向が今回開発する人工知能フレームワークに適していると
判断したためJavaを用いて開発を行った.
また,当研究室に所属する学生はJavaの開発に慣れており,学習コストが低いため採用した.

\subsection{Mavenフレームワーク}
Unityとの通信を行うためMavenフレームワークを用いて開発を行った.
ライブラリのバージョン管理をソースコードで行える利点がある.


%------------------------------------------------------%
%インプットコントローラー
%------------------------------------------------------%
\section{解析部分の実装}
\subsection{解析コントローラー}
解析コントローラーでは,「話題解析」や「感情解析」と言った解析を行う分野自体を管理するクラスである.
今回はその解析分野\footnote{解析分野:話題や感情といったものを指し示す分野}
別に,解析アルゴリズムを保持する各解析知能ハブ
\footnote{話題解析知能ハブや感情解析知能ハブなどの解析分野別に作成されるアルゴリズムを複数所持するクラス}
を作るためにInputControllerを作成した,
実装した主要なメソッドなどを以下の\tabref{InputController}に示す.

\begin{table}[tbh]
	\caption{実装した主要メソッド} \label{tab:InputController}
	\begin{center}
		\begin{tabular}[htb]{c|c}
		\hline
		コンストラクタ & 各解析知能ハブを登録する \\
		\hline
		InputData & 入力があった時に各解析知能ハブへデータを渡す \\
		\hline
		\end{tabular}
	\end{center}
\end{table}


\tabref{InputController}のコンストラクタでは「感情」や「話題」などの各解析分野を登録する処理を行う.
その部分のソースコードを一部抜粋したものを以下の\srcref{inCnt}に示す.

\srcPst{Java}{inCnt.java}{inCnt}{新しい解析分野を登録する際のソースコードの一部}

\srcref{inCnt}の6行目8行目10行目で解析分野が,この解析分野に登録されていることがわかる.

\tabref{InputController}のInputDataメソッドではユーザーから入力があった時に入力された情報
を各解析分野の実装済みクラスに値を渡す.
その情報を用いて各解析知能ハブはそれぞれの解析分野にあった内容を解析する.

%------------------------------------------------------%
%親の抽象クラス
%------------------------------------------------------%
\subsection{解析知能ハブ}
初めに解析知能ハブとは,感情解析知能ハブや話題解析知能ハブに当たるものであり,
各ハブはそれぞれ様々な解析アルゴリズムを所持しているものである.

解析する話題別にアルゴリズムを保持するために,Abstract Mode(以下親抽象クラス)という抽象クラスを
実装した.
親抽象クラスには解析する情報ごとに,プログラムを保持するための機構が記述されており
この親抽象クラスに実装した主要なメソッドなどを次の\tabref{Abstract Mode}に示す.

\begin{table}[tbh]
	\caption{Abstract Modeに実装した主要メソッド} \label{tab:Abstract Mode}
	\begin{center}
		\begin{tabular}[htb]{c|c}
		\hline
		init & 初期化を行うメソッド \\ \hline
		getAnalyzeParts & 解析アルゴリズムを選択するメソッド \\ \hline
		analyzeChat & 入力があった際に解析を行わせるメソッド \\ \hline
		\end{tabular}
	\end{center}
\end{table}


\tabref{Abstract Mode}のinitメソッドでは,各解析アルゴリズムが保持している話題
を独自実装を行ったGoogleAPIに渡すことで頻出単語表を作成し,その検索結果の頻出単語表を取得している.

\tabref{Abstract Mode}のgetAnalyzePartsメソッドでは,実際に解析を行うアルゴリズムを生成された
頻出単語をもとに決めるメソッドである.
具体的には各解析アルゴリズムごとに生成された頻出単語表\footnote{検索結果の文字列から固有名詞だけをのこし,出現した
単語とその数をカウントしたもの}とユーザーが入力した文章をGoogleAPIを用いて作成した頻出単語表を
比較して,もっとも似ている頻出単語表を持つ解析アルゴリズムが解析を行うという実装になっている.
%------------------------------------------------------%
%子の抽象クラス
%------------------------------------------------------%
\subsection{解析アルゴリズムの各解析知能ハブへの追加}
人工知能利用フレームワークの,解析アルゴリズムを簡単に実装する構成について説明する.
各解析知能ハブに対して,アルゴリズムを追加実装するために,
解析アルゴリズムを実装する際に用いる,抽象クラス「Abstract Mode Parts」
(以下,子抽象クラス)を実装した.

この子抽象クラスには,親抽象クラスであるAbstract Modeを実装したクラスから解析する際に呼び出されるメソッドや
データベースなどとの連携をが記述されている.
アルゴリズムを新規に追加し試したい場合,これらの部分については追記する必要がない点がメリットである.

以下の\tabref{Abstract Mode Parts}に実際に解析知能を作るために,必要な抽象クラスである
Abstract Mode Partsに実装した主要なメソッドを示す.
\begin{table}[tbh]
	\caption{Abstract Modeの実装} \label{tab:Abstract Mode Parts}
	\begin{center}
		\begin{tabular}[htb]{c|c}
		\hline
		クラス変数 & 保存を行うための変数が定義されている \\ \hline
		コンストラクタ & アルゴリズムの分野を記述する場所 \\ \hline
		ChatAnalyze & アルゴリズムを記述する部分 \\ \hline
		saveData & 解析結果を保存 \\ \hline
		\end{tabular}
	\end{center}
\end{table}

\tabref{Abstract Mode Parts}のクラス変数は,解析した情報を保存するための変数である.
解析結果をクラス変数に入れることで処理が終わった後に適切な形式でデータベースに自動で保存される.

\tabref{Abstract Mode Parts}のコンストラクタは,その解析アルゴリズムの分野について記述する
必要性がある.
具体的に言うとaboutという変数に話題の名前を入れる必要があり,
入れることで,その話題をユーザーが話した時にそのアルゴリズムが選択されて返答されるという構造が実装される.
その部分のソースコードから重要な部分のみを抜粋したソースコードを以下の\srcref{about}に示す.

\srcPst{Java}{about.java}{about}{コンストラクタで話題を設定するソースコードの一部}

\srcref{about}は料理に関する話題を解析するプログラムの,コンストラクタを抜粋したものだが,
ここで1行目のプログラムを記述する.

\tabref{Abstract Mode Parts}のChatAnalyzeは,実際に解析アルゴリズムを書く部分である.
ここで入力された内容が,String型で引数として渡されてくるのでその内容を用いて解析を行う.
解析した情報はあらかじめ定義してあるクラス変数に保存することで,自動でデータベースに格納されるようになっている.

\srcPst{Java}{anaAlgo.java}{anaAlgo}{解析アルゴリズムを記述するメソッドのソースコードの一部}

\srcref{anaAlgo}は解析を行うアルゴリズムを記述するメソッドのみを抜粋したもので,
ここでデータベースに保存する情報を解析し,解析した情報を変数に格納するプログラムを記述する.

\tabref{Abstract Mode Parts}のsaveDataは,全ての解析が終わった後に呼び出され,変数の中に値
が入っていた場合のみその内容をデータベースにそのクラス名+データ型の名前をつけて保存する.
最後にこの抽象クラスを拡張して作成したクラスは,親クラスであるAbstract Modeを実装した
感情解析知能ハブや話題解析知能ハブなどの親抽象クラスに登録することでアルゴリズムの追加が完了する.

%------------------------------------------------------%
%現在実装しているアルゴリズム
%------------------------------------------------------%
\subsection{現在実装している解析アルゴリズム}
現在実装している解析プログラムは2種類あり,感情の解析と話題の解析である.
話題の解析に関しては例えばゲームの話題解析知能を作った場合,さらに細かい何のゲームかということや
を解析することを目的に作成した.

%------------------------------------------------------%
1つ目に感情の解析を行うアルゴリズムの実装について説明する.
感情の解析を行うプログラムは,親抽象クラスである感情解析知能ハブに所属する解析知能の1つである.
現在感情の解析を行うプログラムは1つなため,必ずこのアルゴリズムがデフォルト解析アルゴリズムとして,
選ばれるようになっている.

具体的な解析を行うアルゴリズムに関しては「哀れ」「恥」「怒り」「嫌」「怖い」「驚き」「好き」「高ぶり」「安らか」「喜び」
の10種類の感情に分類して感情の解析を行なっている.
このそれぞれの感情にはその感情に対応する単語が付いており,入力した文章の中にその単語があった時にその感情値
に1を加えて数字でその感情に関する文字列が出てきた回数を表現する仕組みになっている.

%------------------------------------------------------%
2つ目の話題を解析する知能では料理とゲームに関する話題を解析するプログラムが実装してあり,
料理の分野では「作る」「食べる」「片付ける」の3つの話題にさらに細かく解析する仕組みがある.
また,ゲームの分野では「戦闘」「負け」「勝利」の3つの話題にさらに細かく解析する仕組みを実装している.

%------------------------------------------------------%
%データベース実装
%------------------------------------------------------%

\section{データベースの実装}
\subsection{全ての解析情報を保存する機構}
まず初めにデータベースは独自実装したデータベースクラスを用いて実現した.
データベースの実装では様々な解析情報を保存する必要があるため,複数の変数型に対応するために,HashMap
のキーをStringにし,保存する値であるvalueをObject型に指定した.

このデータベースクラスにはデータを保存するためのメソッドが用意されており,以下にソースコードから
一部抜粋したものを示す.

\srcPst{Java}{dbIn.java}{dbIn}{データベースの解析結果を保存するメソッドの一部}

\srcref{dbIn}のメソッドsetDataに対して,値を保存する際は解析アルゴリズムの子抽象クラスを実装する
際には記述する必要がないという実装になっている.
それは子の抽象クラスの中であらかじめ定義されている変数に値を保存することで,親クラスがsaveDataメソッド
を呼び出し,saveDataメソッドにはこのデータベースクラスのsetDataに対してクラス名と値を送るように
記述してあるためである.

\subsection{解析した情報を取得する機構}
解析した情報を取得する際はこのデータベースクラスのgetDataメソッドを呼び出す.
以下の\srcref{dbOut}にgetDataメソッドのソースコードを示す.

\srcPst{Java}{dbOut.java}{dbOut}{データベースの解析結果をするメソッド}

\srcref{dbOut}の2行目を見ると,データを取得する際にString型の鍵が必要である.
このString型の鍵はデータベースの中にあるデータを保存しているHashMapの鍵を示しており,取得したい
情報の鍵をshowDataというメソッドを用いて情報を取得するための鍵を調べ,その鍵を用いて情報を取得する.

%------------------------------------------------------%
%OutputController
%------------------------------------------------------%

\section{出力を行う知能ハブの実装}
Unityへ返答命令の送信を行うための,情報を作成する部分の実装について説明する.

\subsection{出力コントローラー}
出力する情報をまとめ,JSON形式などを形成する出力コントローラーについて解説する.
まず初めに出力コントローラーに実装した主要なメソッド一覧を以下の\tabref{OutputController}に示す.

\begin{table}[tbh]
	\caption{実装した主要なメソッド} \label{tab:OutputController}
	\begin{center}
		\begin{tabular}[htb]{c|c}
		\hline
		コンストラクタ & 各出力知能ハブを登録する \\
		\hline
		getJson & Unityへ出力情報を送るときに呼ばれるメソッド \\
		\hline
		getTimeAction & キャラクターが自発的に発言する際に用いられるメソッド \\
		\hline
		\end{tabular}
	\end{center}
\end{table}

\tabref{OutputController}のコンストラクタの部分では,各出力知能ハブを登録し,
登録された出力知能ハブはgetJsonメソッドが呼ばれたときに出力内容を作成するようになっている.

\tabref{OutputController}のgetJsonメソッドでは,Unityから入力があった呼ばれるメソッドであり,
あらかじめ登録されている出力知能ハブの出力を作成するメソッドを呼び出す実装になっている.

\tabref{OutputController}のgetTimeActionメソッドでは,時間経過に応じてキャラクターが発言する
設定を有効にしているときに呼び出されるメソッドであり,このメソッドが呼ばれると各出力知能ハブの
自発的に発言する際に用いるメソッドを呼び出すことで応答を行う.

%------------------------------------------------------%
%出力知能はぶ Abstract_Mode
%------------------------------------------------------%


\subsection{出力知能ハブ}
出力する動作や返答内容といった出力情報別に,アルゴリズムを保持する機構について解説する.
この機構も情報解析の時と同じ仕組みで構成されており,この出力の情報別に保持する機構についても
Abstact Modeという親抽象クラスが作成されているので,その抽象クラスを実装することで新たな,
返答知能ハブや動作選択知能ハブと言ったアルゴリズムを複数所持するクラスを新規に作成することが可能である.

以下の\tabref{abstractmode}に出力知能ハブを作成するために必要な抽象クラスである
Abstract Modeの主要メソッドを示す.\\

\begin{table}[tbh]
	\caption{実装した主要なメソッド} \label{tab:abstractmode}
	\begin{center}
		\begin{tabular}[htb]{c|c}
		\hline
		init & 初期化 \\
		\hline
		getOutput & 返答内容の作成 \\
		\hline
		getTimerAction & キャラクターが自発的に発言する際に用いられるメソッド \\
		\hline
		\end{tabular}
	\end{center}
\end{table}

\tabref{abstractmode}のinitメソッドでは初期設定を行っており,独自実装をおこなったGoogleの検索結果データベースを
最新の情報に更新をするなどの処理を行っている.

\tabref{abstractmode}のgetOutputメソッドでは,キャラクターにユーザーが話しかけてきたときに返答内容を
作成するアルゴリズムから出力内容を取得して,その内容を返すメソッドになっている.
返答するアルゴリズムを選択する機構もこの部分にあり,
アルゴリズムの選択はユーザーが発言した内容から作成した頻出単語表と各解析を実際に行う
アルゴリズムが持っている話題をもとに作成したHashMapを比較して,もっとも頻出単語表が似ているものを選ぶという仕組みになっている.
そのソースコードから一部抜粋したものを\srcref{getOutput}に示す.

\srcPst{Java}{getOutput.java}{getOutput}{getOutput.javaのソースコードの一部}

まず初めに,\srcref{getOutput}の8行,9行目で最新の発言情報を取得し,10行目で発言内容をGoogleAPIに
渡すことで頻出単語表を作成する.
次にあらかじめ作成してある各解析アルゴリズムごとの,話題単語の頻出単語表とGoogleAPIを用いて取得した頻出単語表を
11行目から25行目にかけて比較し,一番最適な解析アルゴリズムを選択している.
最後に45行目にて選択された解析アルゴリズムに解析を行わせ,解析結果をそのまま返している.

\tabref{abstractmode}のgetTimerActionメソッドは時間経過に応じて反応するときに呼び出され,
実際に出力内容を作成するアルゴリズムに対して,自発的に発言する際の出力内容を作成させて取得する.

\subsection{出力アルゴリズムの出力知能ハブへの追加}
実際に出力情報を作成するためのアルゴリズムを記述するプログラムを,簡単に出力知能ハブへ追加するために
出力専用のAbstract Mode Partsという子抽象クラスを作成した.
その抽象クラスを用いることで3行プログラムを書くだけで新しいアルゴリズムを追加できるようになっている.

それではまず初めに,その抽象クラスに実装した以下の\tabref{parts}に示した主要なメソッドについて解説する.

\begin{table}[tbh]
	\caption{実装した主要なメソッド} \label{tab:parts}
	\begin{center}
		\begin{tabular}[htb]{c|c}
		\hline
		コンストラクタ & 担当分野の設定 \\
		\hline
		Action & 返答アルゴリズムの \\
		\hline
		TimeAction & キャラクターが自発的に発言する際に用いられるメソッド \\
		\hline
		dataRefresh & 常にデータベースを最新に保つためのメソッド \\
		\hline
		\end{tabular}
	\end{center}
\end{table}

\tabref{parts}のコンストラクタでは,出力を行う際に担当する分野や話題について記述する部分である.
解析を行う際と同じように変数aboutに対して適切な担当する話題名を入れることで,GoogleAPIを用いて
その話題名に関する頻出単語表が自動で生成される.
その生成された頻出単語表も解析の時と同じく先ほど説明したアルゴリズムの選定に利用される.
以下の\srcref{aisatu}に挨拶の分野を指定する場合のプログラムを一部抜粋したものを示す.

\srcPst{Java}{aisatu.java}{aisatu}{話題を指定する際のサンプルソースコードの一部}

\srcref{aisatu}では話題を挨拶に指定しており,ユーザーが挨拶と関係のある単語を発話した時にこのアルゴリズムが
選択され,実際に返答内容を作成するようになるように実装されている.

\subsection{現在実装している出力アルゴリズム}\label{sec:back}
現在実装している出力知能ハブは2つあり,会話を行う知能ハブと動作選択を行う知能ハブの2種類である.
会話を行う知能ハブには2つのアルゴリズムが搭載されており,料理とゲームの話題に関するアルゴリズムを実装した.
料理出力のアルゴリズムでは,解析した時に取得した作る,食べる,片付けるの状態を用いて,返答を行う.
また,ゲームのアルゴリズムでは戦闘,負け,勝利の状態を解析しているのでそれを用いて返答を行っている.

次に動作を選択するアルゴリズムでは共同開発の鈴木さんのデータベースから動作一覧,その1つ1つの動作に関係する単語,
その関係する単語をGoogleAPIをもちいて頻出単語表を作ったものを取得する.
動作を選択するアルゴリズムはその動作に関連付けられている頻出単語表と,ユーザーの発言内容から作成した
頻出単語表を比較して,最も関連性のあるモーションを選択するようになっており,
その具体的なアルゴリズムや通信に関しては\ref{sec:motion}に記述する.
%------------------------------------------------------%
%Unityとの通信に関して
%------------------------------------------------------%

\section{Unityとの通信の実装}
\subsection{通信方式}
Unityとの通信にはWebSocketを用いており,双方向任意のタイミングでの情報の送受信が可能となっている.
Unityとの情報の送受信を行うために人工知能利用フレームワークの中に送受信を行うためのクラスである,
NewWSEndpointを実装した.

以下の\srcref{endpoint}にソースコードを一部抜粋したものを示す.

\srcPst{Java}{endpoint.java}{endpoint}{WebSocketのJavaサーバー側実装の一部}

\subsection{Unityからの入力情報の受信}
\srcref{endpoint}のソースコードではUnity(クライアント)が,接続を行った時にそのセッションを保存するために
23行目でリストにセッションを保存している.
この実装により,セッションが確立され,受信を行う準備が完了する.

Unityからメッセージが来た場合は\srcref{endpoint}の44行目のprocessUploadが呼ばれる.
33行目には同じく受信するメソッドであるOnMessageというメソッドがあるが,
これはUnity上のキャラクター以外の端末からメッセージを受け取った際に呼ばれる.

processUploadメソッドではバイト形式で入力情報を受け取るため,47行目にてバイトをString型に変換している.
変換した後は48行目でその値を入力情報の解析を行う解析知能ハブへ渡し,その処理が終わった後の49行目で返答する値を生成,
作成したjson形式の出力情報を利用し,51行目で全ての接続クライアントに対して命令を送信している.
最後にこれ以上通信を行わない場合は\srcref{endpoint}の27行目で,セッションが切断された時に
リストから削除する処理を行っている.

\subsection{Unityへの命令の送信}
\srcref{endpoint}の51行目で全ての接続クライアントに対して出力知能ハブから得たjson形式の値を
送信している.

人工知能利用フレームワークが家庭で利用されることを目標にしており,
例えばテレビのブラウザでキャラクターと対話,PCの画面でキャラクターと対話,スマホの画面でキャラクターと対話
を行った際に全てのデバイスから同じキャラクターと対話することを実現するために51行目では
全てのクライアントへ対して情報を送信するように実装をおこなった.

%------------------------------------------------------%
%データベースとの通信とモーションデータベースについて
%------------------------------------------------------%

\section{人工知能利用フレームワークに追加したモーションの利用}\label{sec:motion}
\subsection{動作選択アルゴリズムの実装}
先ほど\ref{sec:back}で説明した,動作選択を行う際に用いているアルゴリズムについて解説する.
以下に動作を選択する際に用いている動作選択アルゴリズムを一部抜粋したものを示す.

\srcPst{Java}{motion.java}{motion}{動作選択アルゴリズムの一部抜粋}

まず初めに,\srcref{motion}の3行目に動作選択が行われる際に呼び出される,Actionメソッドが記述されている,
このメソッドでは8行目に記述されたメソッドmatchを呼び出し,その中で適切なモーションを選択している.
\srcref{motion}のmatchメソッドでは15行目から17行目にかけてデータベースを管理するコントローラーから
データベースの情報を取得している.
このコントローラーはサーバー上のMongoDBからモーションデータなどの情報を取得している.

また,このコントローラの実装に関しては共同開発の鈴木さんの論文\cite{suzuki}を参照すると,
このコントローラーからデータベースへの接続を行なっていることがわかる.

実際にどの動作を実行するかを判定しているアルゴリズムを解説する.
\srcref{motion}の22行目から39行目を見ると,
そこではユーザーが発言した内容から作成した頻出単語表と,データベースの中にあるモーションごとに関連付けられている頻出単語表を比較している.
その2つの頻出単語表の比較結果の中で一番同じ単語が多く出現した動作が選択され,40行目でその動作名が返される仕組みとなっている.
%------------------------------------------------------%
%GoogleAPI
%------------------------------------------------------%

\section{GoogleAPIによる頻出単語表の作成}
今回実装したGoogleAPIでは,Google検索を用いてウェブ上から情報を取得する機能から
kuromojiを用いて形態素解析を行わせる機能,頻出単語表を作成する機能と3つの機能から成り立っている.

\subsection{形態素解析による検索ワードの作成}
ユーザーが入力した情報をGoogle検索を用いて検索し,どのような分野の単語なのかということを調べるにあたり,
その検索する際のキーワードというものは検索結果やアルゴリズムを選択する際の精度に関わるため,非常に重要である.

そこでJavaの形態素解析器であるkuromoji\cite{gitkuromoji}を用いて形態素解析を行い,
適切な検索ワードを指定できるようにmwSoft blog\cite{kuromoji}を参照,参考にしつつ実装を行った.

具体的には,kuromojiには複数のモードがあり,Searchモードを利用することで「今日の夕飯何にしよう」を
「今日 夕飯 何 しよ」のように検索で利用しやすい形に分解してくれる機能があるのでその機能を用いて
入力を行った情報をそのまま検索するのではなく形態素解析を行ってから検索を行っている.

\subsection{GoogleAPIを利用して検索結果を取得}
検索をかける際にはHttpClientを用いて検索を行っている,HttpClientはPOST通信を用いてサーバーへ接続を
行い,XML形式で結果を受け取るもので,これを用いることでgoogleの検索結果をそのまま取得することが可能である.
また,この機構を実現するためにmwSoft blog\cite{google}を参考にした.

以下の\srcref{google}に実装を行ったGoogleAPIのソースを一部抜粋したものを記載する.
\srcPst{Java}{google.java}{google}{GoogleAPIの一部抜粋ソースコード}

\srcref{google}の20行目にてgoogleの検索結果ページのURLを作成し,
情報を取得することでGoogle検索の結果の内容を取得している.

\srcref{google}の23行目から37行目にかけて検索キーワードを作成しており,先ほど説明したkuromojiの
検索しやすい単語ごとに区切るSearchモードの機能だけではなく,今回は助詞と助動詞を検索ワードから排除して検索を
行なっている.

\srcref{google}の47行目では実際にウェブサイト上から情報を取得しており,取得した結果を
GoogleSerchResultAnalyzerクラスを用いて解析を行い,リストに格納している.



\subsection{検索結果のフィルタリング}
検索結果にはどの単語を検索しても必ず含まれる単語が複数ある.
例としては「キャッシュ」といった単語や「類似」という単語である.
これらの単語は頻出単語表を作成する上で不要なため,フィルタリングを行っており,実際にフィルタリングを行っているのは
\srcref{google}の84行目から108行目である.
また,84行目から行われているのは記号を取り除く作業であり,頻出単語とはなりえない記号を取り除いている.

\subsection{頻出単語表の作成}
頻出単語表を作成するにあたって独自にWordCounterというクラスを実装した.
以下にWordCounterから一部抜粋したソースコードを記載する.

\srcPst{Java}{word.java}{word}{WordCounterの一部抜粋ソースコード}

\srcref{word}の7行目に定義されているwordcountというメソッドに対して,単語ごとにカウントを行って欲しい
文字列を渡すことで解析を行うことが可能である,
また引数としてはborderを設定することができ,1回しか出てこない単語に関してはカウントを行わないという設定を行うことが可能である.

初めにカウントを行うにあたって入力された文章を形態素解析にかける必要があるため,実装では
\srcref{word}の11行目にて,kuromojiを用いた形態素解析を行なっている.
次に形態素解析を行った文章に出てくる単語をカウントする部分については\srcref{word}の14行目から24行目にて
行っている.
最後にカウントした結果をHashMap形式にし,41行目でそのHashMapを返していることがわかる.
以下の\srcref{words}に例として「今日の夕飯何にしよう」で検索を行った結果をもとに作成した頻出単語表を記述する.

\srcPst{Java}{words.json}{words}{頻出単語表}

\srcref{words}の頻出単語表を見ると国名が多く,その国々の料理の検索結果を取得していることがわかる.
例えば日という単語であれば日=5の様に表記されており,日という単語が5回出てきたことがわかる.
また,そのほかにも料理のレシピを検索することが可能なウェブサービスであるクックパッドのクックの文字を多く取得しているほか,
調味料の名前であるキッコーマンなどの単語も取得することが可能である.

以上の様に検索結果からその検索した単語と同じ様な意味を持つ単語を取得することが可能である.

%------------------------------------------------------%
\chapter{実行結果}
%------------------------------------------------------%
%- 実行結果
%------------------------------------------------------%

\section{Unityの出力画面の図}
実際にキャラクターとの対話を行う際の画面は以下の様な画面で対話を行うことが可能である.

\figPst{100}{doUnity}{キャラクターとの対話画面}

\figref{doUnity}の画面にて実際にキャラクターと対話を行う形式で作成した人工知能のアリゴリズムが
正しく動作しているのを確認することが可能である.

%------------------------------------------------------%

\section{実際の会話}
実際に人工知能利用フレームワークを用いて会話を行う例を以下の\tabref{Chat}に示す.

\begin{table}[tbh]
	\caption{キャラクターとの対話例} \label{tab:Chat}
	\begin{center}
		\begin{tabular}[htb]{c|c|c}
		\hline
		ユーザーの発言 & キャラクターからの返答 & 動作 \\
		\hline
		カルボナーラ作ってるんだ & 隠し味にチョコレートいれちゃう? & 悪巧みの動作 \\
		うわお!クッパ強い,負けたよ & ゲームオーバーだねー & がっかり \\
		\hline
		\end{tabular}
	\end{center}
\end{table}

以上の\tabref{Chat}の様にユーザーが話しかけた内容に対して,返答を行う仕組みが実装されていることがわかる.

またユーザーが話しかけた内容から,現在の会話の話題を解析することでその話題に対応している解析や出力を行うアルゴリズムが処理を行う仕組みにより,
詳しい解析や出力を行うことが可能となっていることが\tabref{Chat}を見ることでわかる.


\section{アルゴリズムを追加した後の会話}\label{sec:addAl}
実際にアルゴリズムを追加した後,どの様な変化があるかを検証する.

今回追加を行うアルゴリズムの話題はゲームの「スーパーマリオブラザーズ\footnote{『スーパーマリオブラザーズ』(Super Mario Bros.)は,任天堂が発売したファミリーコンピュータ用ゲームソフト.}」(以下,マリオ)
という話題をもつ解析と出力を行うアルゴリズムである.
話題を解析するアルゴリズムは,
マリオに特化したさらに細かい話題の解析を行うアルゴリズムを実装した.

ユーザーと会話を行う際に用いる,返答アルゴリズムに関しても「マリオ」の話題に対応を行い,
マリオの話題に対して特化しているアルゴリズムを追加した.
このマリオのアルゴリズムを追加した後の会話を以下の\tabref{afterChat}に示す.

\begin{table}[tbh]
	\caption{アルゴリズム追加後の会話} \label{tab:afterChat}
	\begin{center}
		\begin{tabular}[htb]{c|c|c}
		\hline
		ユーザーの発言 & キャラクターからの返答 & 動作 \\
		\hline
		うわお!クッパ強い,負けたよ & きのこを取っておこう! & がっかり \\
		\hline
		\end{tabular}
	\end{center}
\end{table}

この「マリオ」という話題は,ゲームという大枠の中の単一ゲームタイトル名である.
そのためユーザーが話しかけた内容が,単に幅広くゲームに関連のある単語であれば,ゲームの話題が解析を行う.
しかしここで「マリオ」というゲームに登場する敵キャラクター「クッパ\footnote{クッパとは、
マリオシリーズに登場するキャラクターであり,敵キャラクター正式名称は「大魔王クッパ」}」
という単語を含めることにより,\tabref{afterChat}の様に
マリオの解析と出力に特化したアルゴリズムが選択される.

返答内容に関しても「負けたよ」という発言から,このゲームのプレイヤーを強化することができるアイテム
である「きのこ」を取得して再度,敵に対して挑戦しようという返答を行うことができていることがわかる.

この様に料理やゲームなどの会話を行う際に対応する種類を横に広げていくアルゴリズムの追加方法と,
料理やゲームのさらに細かい話題に対して解析を行うことができるようにするアルゴリズムの追加方法の両方に対応している.


%------------------------------------------------------%
\chapter{結論}
%------------------------------------------------------%
%- 結論
%------------------------------------------------------%

\section{結論}
今回作成した人工知能利用フレームワークを使い,考案したアルゴリズムを作成し
キャラクターと会話することで試すことはできたと考えている.
しかし,話題が固定された状況で返答アルゴリズムを書く必要があり,
現在どの話題で,どのアルゴリズムが回答するかを選択する部分のプログラムを書きたい場合は
その部分のプログラムを直接書き換えなくてはならない欠点がある.

\subsection{アルゴリズムの追加による出力の変化}
\ref{sec:addAl}の実行結果を見て分かる通り,アルゴリズムを追加することでその分野の話題に
なった時に追加したアルゴリズムが応答していることが分かる.
また,解析を行うアルゴリズムを追加したことによって,その分野のさらに詳しい話題の分析が\ref{sec:addAl}
の返答を見て分かる通り,可能となった.

\subsection{簡単にアルゴリズムを追加できたか}
考案したアルゴリズムを素早く試すために,アルゴリズムの追加する際の構造を簡単化したため,
特定の話題の時に解析や出力情報のアルゴリズムの作成と追加を非常に簡単に実現できる機構は
達成できた.

会話を行うことができるアルゴリズムを,この人工知能利用フレームワークでは3行のプログラムソースコード
の記述で実現できるため,非常に開発を開始するまでの時間を短くすることが出来,
また,Unityによる出力先のサポートにより,この会話内容を本物のキャラクターに言われたらどの様に
感じるかもシミュレーションを行うことができるため,文字だけでの出力しか行えない場合よりも
よりリアルなコミュニケーションを行う人工知能や人工無脳の作成を目指すことができることがわかった.



%------------------------------------------------------%

\chapter*{謝辞}
\addcontentsline{toc}{chapter}{謝辞}
%------------------------------------------------------%
%- 謝辞
%------------------------------------------------------%

本論文を作成するにあたって,監督いただいた教授および参考文献を作成した方々に感謝いたします.


%------------------------------------------------------%
%- References
%------------------------------------------------------%

\begin{thebibliography}{99}

	\bibURL{tino}{wikipedia}{人工知能 Wikipedia}
	{https://ja.wikipedia.org/wiki/人工知能}{2016年1月16日}

	\bibURL{muno}{wikipedia}{人工無脳 Wikipedia}
	{https://ja.wikipedia.org/wiki/人工無脳}{2016年1月16日}

	\bibURL{deep}{wikipedia}{DeepLearning}
	{https://ja.wikipedia.org/wiki/ディープラーニング}{2015年12月27日}

	\bibURL{boom}{東京工科大学}{東京大学松尾 豊}
	{http://ymatsuo.com/japanese/}{2015年12月27日}

	\bibURL{ceo}{LeadingCom}{人工知能により仕事がなくなる}
	{http://lrandcom.com/automation}{2015年1月12日}

	\bibURL{lab}{O2O INNOVATION LAB}{人工知能の全貌に迫る!人工知能の活用事例10選}
	{http://o2o.abeja.asia/product/post-10262/}{2015年1月13日}

	\bibURL{rinna}{Microsoft}{りんな}
	{http://rinna.jp/rinna/}{2015年1月12日}

	\bibURL{Pepper}{Softbank}{Pepper}
	{http://www.softbank.jp/robot/consumer/products/}{2015年1月12日}

	\bibURL{Pepper2}{Softbank}{Pepper World 2016公式サイト}
	{http://www.softbank.jp/robot/special/Pepper-world/?waad=Uy8daWr3}{2015年1月12日}

	\bibURL{line}{line}{line}
	{http://xn--o9jo1npkwb.com/}{2015年1月12日}

	\bibURL{humen}{総務省統計局}{2015年確定日本人人口}
	{http://www.stat.go.jp/data/jinsui/new.htm}{2015年1月12日}

	\bibURL{tis}{TIS株式会社}{TISニュースリリース}
	{http://www.tis.co.jp/news/2016/20160112_1.html}{2015年1月13日}

	\bibURL{coffe}{コメダ珈琲}{コメダ珈琲八王子北口店}
	{http://www.komeda.co.jp/search/shopdetail.php?id=716}{2015年1月12日}

	\bibURL{ota}{Wikipedia}{オタクとは}
	{https://ja.wikipedia.org/wiki/おたく}{2016年1月16日}

	\bibURL{anime}{株式会社メディア開発綜研}{アニメ市場調査}
	{http://www.mdri.co.jp/review/}{2016年1月16日}

	\bibURL{dip}{ディップ株式会社}{日本のオタク人口}
	{http://www.dip-net.co.jp/news/press-release/2014/04/4075.html}{2016年1月16日}

	\bib{fuji}{藤井克成}{複数の動作の合成によるリアルタイムアニメーションの生成}{東京工科大学 2016年 卒業論文}

	\bib{suzuki}{鈴木智博}{日常会話を想定したモーションデータベースの構築と運用}{東京工科大学 2016年 卒業論文}

	\bibURL{gitkuromoji}{github}{形態素解析器kuromoji}
	{https://github.com/atilika/kuromoji}{2015年8月12日}

	\bibURL{google}{mwSoft blog}{HttpClientとHttpCleanerでGoogle検索結果を解析する例}{http://blog.mwsoft.jp/article/34841195.html}{2016年1月16日}

\end{thebibliography}

\end{document}
