%------------------------------------------------------%
%- 提案
%------------------------------------------------------%
\section{開発した人工知能の活用}
今回提案するのは先ほど説明した一般的な人工知能フレームワークを用いて開発を行った
人工知能やその返答アルゴリズムを活用するためのフレームワークである.
%------------------------------------------------------%
\subsection{知能の開発をサポートする既存フレームワーク}
既存の人工知能フレームワークは,人工知能自体を作成することをサポートしている.
その作成した人工知能自体を試す場合,ユーザーはキーボードを使い発言内容を記述する
必要がある.

%------------------------------------------------------%
\section{開発した知能を試す環境}
今回提案するのは,開発を行うフレームワークを用いて開発したアルゴリズムや,
独自のアルゴリズムを実際に動かし試す環境を簡単に構築するフレームワークである.

既存の開発を行うことに特化しているフレームワークでは,人工知能を作ること自体に着目し
人工知能の作成を行うプロセスをサポートしている.
しかし今回提案する人工知能利用フレームワークでは,考案したアルゴリズムを
1つの人工知能ハブに対して複数登録することでUnity上で動作するキャラクターと
会話を行うことができる人工知能を利用することをサポートするフレームワークである.

通常,人工知能のアルゴリズムを試す場合,そのプログラムに対してユーザーが入力を行う入力の部分と
その処理結果を出力する出力画面や出力を行うキャラクターの部分を作成する必要がある.

作成したアルゴリズムの出力結果が文字で出力されれば良い場合は,
出力画面を準備をする手間は不要である.
しかしキャラクターとの会話で最終的には試したいと考えた場合,
入出力の部分を作成する際に非常に手間と時間がかかるため,
その実行環境とユーザーが入力を行う部分を予め人工知能利用フレームワークで提供することで,
その準備の手間が不要になるという利点がある.

キャラクターと会話を実際にすることで,実際にキャラクターに言われたらどの様に感じるかを
シミュレーション,フィードバックすることが可能であり,
よりリアルなコミュニケーションを行うアルゴリズムを,開発することを
サポートすることが可能になる.

今回はその様な人工知能を利用することに着目したフレームワークを提案する.
%------------------------------------------------------%
\section{人工知能利用フレームワークの提案}
人工知能利用フレームワークは会話や動作などの返答アルゴリズムを作成した際に,
それらの作成したアルゴリズムをフレームワーク上に適当に記述することで,
状況や話題に応じて適切な作成した返答アルゴリズムが選択され
Unity上のキャラクターと会話を楽しむことができる,
人工知能を利用することに焦点を当てたフレームワークである.
%------------------------------------------------------%
\subsection{提案する全体構成}\label{sec:allAr}
この人工知能利用フレームワークの全体の構成を次の\figref{all_kose}に示す.

\figPst{90}{all_kose}{全体の構成図}

提案する人工知能利用フレームワークにはUnityで作成したキャラクターを出力する部分,
キャラクターに行わせる動作を考える人工知能ハブ,及びUnity上で利用するためのモーションを保存する
モーションデータベースの3つから構成される.

今回私が担当し,作成した人工知能ハブについて解説を行う.
人工知能ハブは大きく分けて3つの要素で構成され,
Unityでユーザーが入力した内容をもとに人工知能活用ハブがその入力内容を受け取り解析を行う部分,
解析した情報を保存するためのデータベース及び返答する内容が作り出される部分である.
動作を選択する部分は,共同研究の鈴木が作成したモーションデータベースから適切な動作を選択し,
Unityへ動作と返答内容を出力する.
%------------------------------------------------------%

\subsection{アルゴリズムのみを簡単に追加可能な知能ハブ}
アルゴリズムのみを簡単に追加することができる人工知能活用ハブを提案する.

人工知能活用ハブでは,作成した会話の返答キャラクターの動作を選択するアルゴリズムを簡単に追加する
ことができる.
追加したアルゴリズムにそれぞれ話題を設定することで
ユーザーが話しかけた話題に応じて適切なアルゴリズムを用いて返答を行うことができるようにすることを提案する.
例えばゲーム関連の返答アルゴリズムを作る場合は,
そのアルゴリズムをあらかじめ準備されている抽象クラスを用いて実装し,
\figref{all_kose}のプログラムに抽象クラスを実装したプログラムを登録することで,
ゲームの話題が来た時にそのアルゴリズムでキャラクターが返答するシステムを作ることが可能である.

同様にゲーム関連のキャラクターの動作を選択するアルゴリズムを作る場合は,そのアルゴリズムを
抽象クラスを用いて実装し,
\figref{all_kose}の動作選択アルゴリズム軍の中に作成したプログラムを登録するだけで,ゲーム関連
の会話をしている最中は,そのアルゴリズムを用いて動作を決定する仕組みを作ることができるというものである.

これらのアルゴリズムや人工知能が1つだけ実装されている場合は,
デフォルトアルゴリズムが選択されるように設計ている.
複数の料理の話題に特化した話題解析アルゴリズムやゲームの話題に特化した感情解析のアルゴリズムが
実装されることで,様々な話題に特化したより正確な解析が可能になる.
返答を作成する際も,ゲーム専用の返答アルゴリズムや料理に特化した返答アルゴリズムがあることでより
円滑なコミュニケーションがを行うことが可能になると提案する.

今後様々なアルゴリズムが必要になるため,複数人で開発を行うことが想定される.
その際にも解析情報をデータベースを用いて共有することで効率的に開発ができるようになっている.
人工知能ハブでは,すでに解析した感情情報などの情報は全てデータベースによって共有されている.
そのため,入力情報の解析を行うプログラムの開発は行わずに,すでにある感情解析プログラムの
解析結果などの情報を使いユーザーの感情状態を考慮した「会話ボット」などの開発を行うことも可能となる.\\
%------------------------------------------------------%
\subsection{作成したアルゴリズムをUnityですぐに試せる機構}
この人工知能利用フレームワークの人工知能活用ハブに登録されたアルゴリズムはキャラクターとの
対話ですぐに試すことができる.

共同研究者の藤井克成の論文\cite{fuji}によると,MMDモデルを利用しているため,モデルを入れ替えることで
好きなキャラクターで動作させることが可能である.
また,人工知能利用フレームワークのために開発したリアルタイムに動作を補完しながらキャラクターを
動かす技術により,よりリアルなコミュニケーションが可能となっている.

作成した人工知能を,すぐにキャラクターとの対話という形で実行することができるため,
入出力の設計や,開発はどうするかに迷うことなく,独自の対話アルゴリズムや
人工知能の開発に専念することが可能になる.
%------------------------------------------------------%
\subsection{Unityが利用可能なモーションを追加する機構}
この人工知能利用フレームワークでは現在会話と動作の2つの出力を実装している.
返答パターンは文字列で生成され,Unityで実行されるので様々なバリエーションで返答すること
ができる.
しかしキャラクターの動作は,動的にプログラムを用いて動作を生成することが難しいため,
予めモーションデータを作成しておく必要性がある.

そのモーションデータを定義するファイルを生成することも手間と時間がかかる.
そこで共同開発の鈴木智博がKinectで動作を定義し,データベースに保存,人工知能活用ハブと通信可能な
プログラムを開発した\cite{suzuki}.

この機構があることによって,人工知能活用フレームワークの中で新しい動きのパターンを追加したい場合,
すぐにKinectを用いて自ら追加したい動作をキャプチャすることで動作ファイルが生成され,
データベースに登録することで使えるようなる.