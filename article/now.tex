%------------------------------------------------------%
%- 現状
%------------------------------------------------------%

\section{近年の機械学習}
まず初めに,機械学習とはデータから反復的に学習し,そこに潜むパターンを見つけ出すことです.
近年その機械学習を会話に用いる動きがあります.
\\

\subsection{人工無能}
人工無脳\cite{muno}とは,人工知能に対応する用語で,英語圏ではchatterbot,もしくはchatbotと呼ばれ,
その訳語として会話ボットあるいはおしゃべりボットとも呼ばれることがあります.
\\
この人工無脳はテキストや音声などを用いて,会話をシミュレートすることができるプログラムであり,
一見知的に人間の様な応答をしている様に見えるが,多くの場合は会話の中の特定のキーワードを拾い
そのキーワードに対応する返答を行っている場合が多いのが現状です.
\\

\subsection{人工知能}
人工知能\cite{tino}とは人工的にコンピュータ上などで人間と同様の知能を実現させようという試み,
或いはそのための一連の基礎技術を指すものであり,1956年にダートマス会議でジョン・マッカーシーにより命名されました.
\\

しかし,人工知能の定義は未だ不確定な部分が多く,完全に正確な定義は存在していません.
\\

この人工知能は人工無脳とは異なり,ただ単にキーワードを広くだけではなく,
取得した情報を用いて解析や情報の取得を行い,状況に応じた返答などが可能なプログラムのことを指すことが多いです.
\\

\subsection{neural network}
Neural Networkというのは,脳細胞を構成する「Neuron(ニューロン)」の活動を単純化したモデルであり,これを利用する
ことによって人間の思考をシミュレーションすることができます.
\\

\subsection{DeepLearning}
DeepLearning\cite{deep}とは多層構造のニューラルネットワークの機械学習の事であり,
ニューラルネットワークを多層積み重ねたモデルを機械学習させればディープラーニングということになる,
また,一般的には3層以上のニューラルネットワークがあるものとされています.
\\

第一層から入った情報は,より深い3層へと行く過程で学習が行われ,その学習を行っていくことで
概念を認識する特徴量と呼ばれる重要な変数を自動で発見することができます.
\\

このDeepLearningにより,東京工科大学の柴田千尋さん\cite{boom}などの
様々な人工知能の研究者が第三の人工知能のブームが来たと発言しています.
\\

\section{一般的な人工知能開発フレームワーク}
現在一般的な人工知能を開発するフレームワークとしてchainerやGoogleのTensorFlowなどがあります.
\\
Chainerは,Preferred Networksが開発したニューラルネットワークを実装するためのライブラリであり,
人工知能自体の開発を行う際に高速な計算が可能なことや,様々なタイプのニューラルネットを実装可能,また
ネットワーク構造を直感的に記述できる利点があります.
\\
GoogleのBrain Teamの研究者たちが作った機械学習ライブラリであるTensor Flowは,
グラフを構築・実行するためのPython APIとC++インターフェイス一式が用意されて実際の開発を行う際に
非常に有効だと考えられます.

これらの人工知能開発フレームワークは実際に開発を行うときに非常に有効であり,様々な分野での応用を試みる
動きがあります.
\\
