%------------------------------------------------------%
%- 提案
%------------------------------------------------------%
\section{開発した人工知能の活用}
今回提案するのは,独自に考えた対話アルゴリズムをJavaを用いて実装するだけで,
その作成したアルゴリズムを元に対話やキャラクターの動作を行うことを実現するフレームワークの開発である.

%------------------------------------------------------%
\subsection{知能の開発をサポートする既存フレームワーク}
既存の人工知能フレームワークは,人工知能自体を作成することをサポートしている.
現状開発を行った人工知能を用いてキャラクターと会話を行うというところまでをサポート
するものはない.
キャラクターとの対話で人工知能を試したい場合は別途キャラクターが動作するプログラムを1から作る必要があり,
アルゴリズムを作ったとしてもそれを気軽に楽しむ為の環境がないのが現状である.

%------------------------------------------------------%
\section{提案する人工知能利用フレームワーク}
今回提案するのは,開発した対話アルゴリズムや動作を選択するアルゴリズムを動かし,
キャラクターとの対話でそのアルゴリズムを試すことができるフレームワークである.

開発の部分に関しては可能な限り考案したアルゴリズム以外の部分は書かずに実装できるものとし,
考えたアルゴリズムをすぐに実装することができるシステムを提案する.

キャラクターとの対話の分に関しては,現在人工知能のアルゴリズムを試す場合,
そのプログラムに対してユーザーが入力を行う入力の部分と,
その処理結果を出力する出力画面や出力を行うキャラクターの部分を作成する必要がある.

対話を行う為の入出力の部分を作成する際には,キャラクターのモデルの準備や入力を行う際に必要な
音声認識など,非常に手間と時間がかかることが予想される.
そのキャラクターを用いた出力環境とユーザーが入力を行う部分を予め人工知能利用フレームワークで提供することで,
その準備の手間が不要になるという利点がある.

キャラクターと会話を実際にすることで,実際にキャラクターに言われたらどの様に感じるかを
シミュレーション,フィードバックすることが可能であり,
よりリアルなコミュニケーションを行うアルゴリズムを,開発することを
サポートすることが可能になる.

以上の提案を実装する人工知能利用フレームワークとは,
まとめると人工知能ハブというものに対して考案したアルゴリズムを気軽に開発し複数登録することが可能であり,
開発したアルゴリズムを元にUnity上で動作するキャラクターとコミュニケーションを行うことができるものである.
%------------------------------------------------------%
\subsection{提案する全体構成}\label{sec:allAr}
この人工知能利用フレームワークの全体構成を次の\figref{all_kose}に示す.

\figPst{90}{all_kose}{全体の構成図}

提案する人工知能利用フレームワークにはUnityで作成したキャラクターを出力する部分,
キャラクターに行わせる動作を考える人工知能ハブ,及びUnity上で利用するためのモーションを保存する
モーションデータベースの3つから構成される.

今回私が担当し,作成した人工知能ハブについて解説を行う.
人工知能ハブも大きく分けて3つの要素で構成され,
Unityでユーザーが入力した内容をもとに人工知能活用ハブがその入力内容を受け取り解析を行う部分,
解析した情報を保存するためのデータベース,及び返答する内容が作り出される部分である.
動作を選択する部分は,共同研究の鈴木が作成したモーションデータベースから適切な動作を選択し,
Unityへ動作と返答内容を出力する.
%------------------------------------------------------%

\subsection{アルゴリズムのみを簡単に追加可能な知能ハブ}
アルゴリズムのみを簡単に追加することができる人工知能活用ハブを提案する.

人工知能活用ハブでは作成した会話の返答アルゴリズム,キャラクターの動作を選択するアルゴリズムを簡単に追加する
ことができる.
追加したアルゴリズムにそれぞれ話題\footnote{話題:そのアルゴリズムはどのような話題の時に解析を行うかを決める単語}
を設定することで,ユーザーが話しかけた内容に応じて適切なアルゴリズムを用いて返答を行うことができるようにすることを提案する.

例えばゲーム関連の返答アルゴリズムを作る場合は,
そのアルゴリズムをあらかじめ準備されている抽象クラスを用いて実装し,
\figref{all_kose}の返答アルゴリズム軍にその実装したクラスを登録することで,
ゲームの話題が来た時にそのアルゴリズムでキャラクターが返答するシステムを作ることが可能である.

同様にゲーム関連のキャラクターの動作を選択するアルゴリズムを作る場合は,そのアルゴリズムを
抽象クラスを用いて実装し,
\figref{all_kose}の動作選択アルゴリズム軍の中に作成したプログラムを登録するだけで,ゲーム関連
の会話をしている最中は,そのアルゴリズムを用いて動作を決定する仕組みを作ることができるというものである.

これらの実際に解析を行うアルゴリズムや人工知能が1つだけ実装されている場合は,
デフォルトアルゴリズムとしてその実装されているアルゴリズムが自動で選択されるように設計されている.

複数の料理の話題に特化した話題解析アルゴリズムやゲームの話題に特化した感情解析のアルゴリズムが
実装されることで様々な話題に対応するようになり,より正確な解析アルゴリズムが選択されるようにすることが可能である.
解析だけではなく返答内容の動作や発言内容を作成する際も,ゲーム専用の返答アルゴリズムや料理に特化した
返答アルゴリズムがあることでより円滑なコミュニケーションがを行うことが可能になると提案する.

今後様々なアルゴリズムが必要になるり,複数人で開発を行うことが想定される.
その際に解析情報をデータベースを用いて共有することで効率的に開発を行えるようにするために,
人工知能ハブではすでに解析した感情情報などの情報は,全てデータベースによって共有されている.

この構成にすることで入力情報の解析を行うプログラムの開発は行わずに,すでにある感情解析プログラムの
解析結果などの情報を使い,ユーザーの感情状態を考慮した「会話ボット」などの開発を行うことも可能となる.
%------------------------------------------------------%
\subsection{作成したアルゴリズムをUnityですぐに試せる機構}
この人工知能利用フレームワークの人工知能活用ハブに登録されたアルゴリズムはキャラクターとの
対話ですぐに試すことができる.

共同研究者の藤井克成の論文\cite{fuji}によると,MMDモデルを利用しているため,モデルを入れ替えることで
好きなキャラクターで動作させることが可能である.
また,人工知能利用フレームワークのために開発したリアルタイムに動作を補完しながらキャラクターを
動かす技術により,よりリアルなコミュニケーションが可能となっている.

作成した人工知能を,すぐにキャラクターとの対話という形で実行することができるため,
入出力の設計や,開発はどうするかに迷うことなく,独自の対話アルゴリズムや
人工知能の開発に専念することが可能になる.
%------------------------------------------------------%
\subsection{Unityが利用可能なモーションを追加する機構}
この人工知能利用フレームワークでは現在会話と動作の2つの出力を実装している.
返答パターンは文字列で生成され,Unityで実行されるので様々なバリエーションで返答すること
ができる.
しかしキャラクターの動作は,動的にプログラムを用いて動作を生成することが難しいため,
予めモーションデータを作成しておく必要性がある.

そのモーションデータを定義するファイルを生成することも手間と時間がかかる.
そこで共同開発の鈴木智博がKinectで動作を定義し,データベースに保存,人工知能活用ハブと通信可能な
プログラムを開発した\cite{suzuki}.

この機構があることによって,人工知能活用フレームワークの中で新しい動きのパターンを追加したい場合,
すぐにKinectを用いて自ら追加したい動作をキャプチャすることで動作ファイルが生成され,
データベースに登録することで使えるようなる.