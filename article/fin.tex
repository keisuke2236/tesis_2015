%------------------------------------------------------%
%- 結論
%------------------------------------------------------%
\section{結論}
今回作成した人工知能利用フレームワークを用いて,考案したアルゴリズムを作成し
キャラクターと会話することができたと考えている.

しかし,話題が固定された状況で返答アルゴリズムを書く必要があり,
どの話題でどのアルゴリズムが回答するかを選択する部分のプログラムを書きたい場合は
その部分のプログラムを直接書き換えなくてはならない欠点がある.

\subsection{アルゴリズムの追加による出力の変化}
\ref{sec:addAl}の実行結果を見て分かる通り,アルゴリズムを追加することでその分野の話題に
なった時に追加したアルゴリズムが応答していることが分かる.
解析を行うアルゴリズムを追加したことによって,その分野のさらに詳しい話題の分析が\ref{sec:addAl}
の返答を見て分かる通り,可能となった.

この通りアルゴリズムを追加することにより,より高精度な解析や出力を行うことができた.

\subsection{簡単にアルゴリズムを追加できたか}
考案したアルゴリズムを素早く試すために,アルゴリズムを追加する際の構造を簡単化したため,
特定の話題の時に解析や出力情報のアルゴリズムの作成と追加を非常に簡単に実現できる機構を
作ることができた.

会話を行うことができるアルゴリズムを,この人工知能利用フレームワークでは3行のプログラムソースコード
の記述で実現できるため,非常に開発を開始するまでの時間を短くすることが出来たと考えられる.

Unityによる出力先のサポートにより,この会話内容を本物のキャラクターに言われたらどの様に
感じるかもシミュレーションを行うことができた.
これにより文字だけでの出力しか行えない場合よりも,
よりリアルなコミュニケーションを行う人工知能や人工無脳の作成を目指すことができることがわかった.

\subsection{システム全体としての結論}
人工知能利用フレームワークを用いることで対話アルゴリズムや動作選択などの,
実際に作った対話アルゴリズムを用いてキャラクターと対話を行うことをサポートすることができた.

また,キャラクターと対話を行う際に得るフィードバックから人工知能の開発の際によりリアルなコミュニケーション
を行うことができると感じた.