%------------------------------------------------------%
%- 現状
%------------------------------------------------------%
近年人工知能などの分野が注目を浴びており,ここではそれらの現状について説明する.

\section{人工知能}
人工知能とは人工的にコンピュータ上などで人間と同様の知能を実現させようという試み,
或いはそのための一連の基礎技術を指すものであり,1956年にダートマス会議でジョン・マッカーシーにより命名された.
人工知能の定義は未だ不確定な部分が多く,完全に正確な定義は存在していないのが現状である.
この人工知能は人工無脳とは異なりただ単にキーワードを広うだけではなく,
機械学習によって取得した情報を用いて解析や情報の取得を行い,状況に応じた返答などが可能なプログラムのことを指すことが多い.
また学習を行わず特定のキーワードを拾い返答するものを,人工無脳または対話ボットという.

\section{機械学習}
機械学習とは入力されたデータから反復的に特徴を学習することでそこに潜むパターンを見つけ出す技術のことであり,
人工知能には必要不可欠な要素である.

\section{Neural Network}
Neural Networkとは,脳細胞を構成する「Neuron(ニューロン)」の活動を単純化したモデルであり,これを利用する
ことによって人間の思考をシミュレーションすることができる.
現在Neural NetWorkは主にDeepLearningで利用されている.

\section{DeepLearning}
DeepLearning\cite{deep}とは多層構造のニューラルネットワークの機械学習の事であり,
ニューラルネットワークを多層積み重ねたモデルを機械学習させればディープラーニングとなる.
また,一般的には3層以上のニューラルネットワークがあるものとされている.

第一層から入った情報は,より深い3層へと行く過程で学習が行われ,その学習を行っていくことで
概念を認識する特徴量と呼ばれる重要な変数を自動で発見することができる.

このDeepLearningにより,東京大学の松尾豊准教授\cite{boom}を始めとする機械学習や人工知能の研究者は
「AI研究に関する大きなブレイクスルーであり、学習方法に関する技術的な革新である」と指摘している.

\section{一般的な人工知能開発フレームワーク}
現在一般的な人工知能を開発するフレームワークとしてchainerやGoogleのTensorFlowなどがある.

Chainerは,Preferred Networksが開発したニューラルネットワークを実装するためのライブラリであり,
人工知能自体の開発を行う際に高速な計算が可能なことや,様々なタイプのニューラルネットを実装可能であり,
またネットワーク構造を直感的に記述できる利点がある.

GoogleのBrain Teamの研究者たちが作った機械学習ライブラリであるTensor Flowは,
Python APIとC++インターフェイス一式が用意されているため開発を行う際に非常に有効だと考えられる.

これらの人工知能開発フレームワークは実際に開発を行うときに非常に有効であり,googleの検索アルゴリズムや
データ分析などの様々な分野での応用を試みる動きがある.

\section{知能の開発をサポートする既存フレームワークの現状}
上記で説明したようなフレームワークが登場する前までは,人工知能の開発は非常に手間と時間のかかるものであった.
それに加えてchainerはGoogleが開発を行った非常に高度な技術を用いるフレームワークであり,
個人でこのような機構を持った人工知能の開発を行いたいと考えたとしても事実上不可能であるのが現状である.

しかし様々なフレームワークの登場により人工知能自体を開発する事が非常に簡単になった事に加え,
より高度な技術を用いた人工知能の開発を行う事が可能になった.
このように個人でも高度な人工知能の開発が可能になった事により,現在非常に人工知能の分野が注目を浴びている.

しかし既存のフレームワークには,開発した人工知能を用いてキャラクターと会話を行うというところまでをサポートするフレームワークがない.
キャラクターとの対話で人工知能を試したい場合は対話を行うアルゴリズムを作成するのに加えて,
キャラクターとの対話を行うインターフェイスの準備やキャラクターを動作させるための動作ファイルなど様々な準備が必要となるのが現状である.

このように既存のフレームワークを用いて簡単に高性能なアルゴリズムを作ったとしても,それを気軽に楽しむ為の環境がないのが現状である.

\newpage

\section{人工知能のアプリケーションへの応用}
現在Microsoftのりんな\cite{rinna}やソフトバンクのpepper\cite{pepper}などの登場により,人工知能を応用した対話に注目が集まっている.

Micsoroftのりんなは,Microsoftが開発を行った人工知能であり,話しかけると女子高生のような返答を返してくれるものである.
現在LINE\footnote{LINEとは韓国最大のIT企業「NHN」の日本法人「LINE株式会社」が提供しているスマートフォン(iPhoneやAndroid)、ガラケー(フィーチャーフォン)、パソコンに対応したコミュニケーションアプリケーションです.\cite{line}}
アプリ上で公式アカウントを持っており,りんなを友達登録しているユーザー数は2016年1月13日現在では2,167,730人である.
これは日本の人口が現在1億2688万人\cite{humen}なので総人口の1.7\%にあたり,およそ100人に1人または2人がこの人工知能りんなとの対話を
行っているというのが現状である.

ソフトバンクのpepperに関しては2016年現在,テレビCMなども頻繁に行われており,
ソフトバンクショップを初めとして一般家庭や喫茶店などで幅広く活用されている.

八王子市の喫茶店コメダ珈琲店\cite{coffe}のスタッフに「pepper君を導入してから何か変わりましたか?」とインタビューを行ったところ,
スタッフからは「pepperを導入してからお子様が来店された時に喜び,pepper君と話している姿をみます」といった声や
「店内が混雑している時の待ち時間にお客様がお話ししており,待ち時間の退屈さを紛らわせてくれている」といった意見を聞く事が出来た.

\figPst{90}{pepper}{東京都八王子市のコメダ珈琲店で活躍するpepper君}

\figref{pepper}のように珈琲店では帽子をかぶり来店されたお客様とコミュニケーションを行っている.
また私がpepper君の見える席に座っていたところ,来店される方とpepper君の目が会うたびに人々が笑顔になったり,
驚いていたりしていたのに加えてお年寄りの方が親身にpepper君に話しかけている光景を目にしており,
人々の人工知能との対話への抵抗がなくなってきていることがわかった.
特に今後生まれてくる人々は人工知能との会話が普通のこととなっていくことが予想できる.

このようにテキストチャットの形式で対話を行う事ができるりんなの登場や人工知能を活用したロボットpepperの登場によって人工知能
に非常に注目が集まっており,このような対話システムを企業だけではなく個人でも開発したいという需要は高まっていると考えられる.

これに加えて日本にはオタク
\footnote{おたく(オタク、ヲタク)とは、1970年代に日本で誕生した呼称であり大衆文化の愛好者を指す。元来はアニメ・SF・パソコンなどの、
なかでも嗜好性の強い趣味や玩具、の愛好者の一部に使われていた術語であったが、バブル景気期に一般的に知られはじめた。\cite{ota}}
という文化がある.
ディップ株式会社の調査によると日本人口の約40%はオタクであるという調査結果となっており,非常にアニメやゲームやパソコンなどに
興味を持っている人口が多い事がわかる.

中でもアニメに関しては毎年市場が伸びており,2015年度に株式会社メディア開発綜研が調査した結果によるとその市場は2595
億円(前年比106.9\%)と過去最大の規模になったという.\cite{anime}

このように現在日本では,人工知能を搭載したロボットや女子高生人工知能との会話,アニメが注目されている事がわかった.

そしてなぜアニメがここまで注目されているのか,それはアニメキャラクターのかっこよさや可愛さといった現実にはない二次元キャラクター特有の
”人々の理想”がそこにあったからだと現状私は分析している.

\figPst{90}{hime}{理想の見た目のキャラクターとの理想の会話例\cite{hime}}

そこで\figref{hime}のように理想の見た目をしているキャラクターに,
理想の会話をしてくれる人工知能を加えたら間違いなくこのオタクの人々を感動させる事ができるという仮説を立てた.

今回はこの仮説をもとに二次元のアニメキャラクターの人工知能を気軽に開発し,二次元のキャラクターと気軽に対話を行うところまでをサポートする
フレームワークを提案する.

