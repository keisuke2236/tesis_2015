%------------------------------------------------------%
%- 構成
%------------------------------------------------------%
それでは人工知能活用ハブの構成を3段階に分けてと,Unityのキャラクターとの連携,
解析するアルゴリズムを選定する時に利用しているGoogkeAPIの3つに分けて,
人工知能利用フレームワークについて解説を行いたいと思います.
\\
\section{解析アルゴリズムの追加を可能にする機構}
まず初めにUnityのキャラクターから受け取った情報を解析する際の機構の構成について解説したいと
思います.\\
こちらの解析アルゴリズムの追加ができることで,
\subsection{解析する内容別にプログラムを保持する機能}
\subsection{会話の話題別に解析するアルゴリズムを選ぶ機能}
\subsection{解析アルゴリズムを簡単に追加する機能}

\section{解析した情報を共有する機能}
\subsection{解析情報を保存する機能}
\subsection{解析情報を取得する機能}

\section{返答アルゴリズムの追加を可能にする機構}
\subsection{返答を行うタイミング}
\subsection{返答する内容別にアルゴリズムのを保持する機能}
\subsection{会話の話題別に返答アルゴリズムを選ぶ機能}

\section{作成した知能をUnityで試す機構}
\subsection{Unityでの出力について}
\subsection{Unityとの連携に利用するWebSocket}
\subsection{Unityへの送信フォーマットと作成}
\subsection{Unityからの受信フォーマット}

\section{アルゴリズムを選定する際に用いるGoogleAPI}
\subsection{GoogleAPIについて}
\subsection{GoogleAPIの有効性}
