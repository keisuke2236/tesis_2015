%------------------------------------------------------%
%- 結論
%------------------------------------------------------%
\section{結論}
今回作成した人工知能利用フレームワークを用いて,考案したアルゴリズムを作成し
キャラクターと会話することができたと考えている.

しかし,話題が固定された状況で返答アルゴリズムを書く必要があり,
どの話題でどのアルゴリズムが回答するかを選択する部分のアルゴリズムも含めて作成したい場合は,
他のアルゴリズムを登録する事が出来ないのが現状であり.
その場合既存の他のアルゴリズムが利用できなくなるという弊害があるのが現状である.

\subsection{アルゴリズムの追加による出力の変化}
\ref{sec:addAl}章の実行結果を見て分かる通り,アルゴリズムを追加することでその分野の話題に
なった時に追加したアルゴリズムが解析や応答をしていることが分かる.
また,解析を行うアルゴリズムの追加に関しても,その分野のさらに詳しい話題の分析が\ref{sec:addAl}
の返答を見て分かる通り可能となった事がわかる.

この通りアルゴリズムを追加することにより,より高精度な解析や出力を行うことができた.

\subsection{簡単にアルゴリズムを追加できたか}
考案したアルゴリズムを素早く試すために,アルゴリズムを追加する際の構造を簡単化したため,
特定の話題の時に解析や出力情報のアルゴリズムの作成と追加を非常に簡単に実現できる機構を
作ることができた.

会話を行うことができるアルゴリズムを,この人工知能利用フレームワークでは3行のプログラムソースコード
の記述で実現できるため,非常に開発を開始するまでの時間を短くすることが出来たと考えられる.

Unityによる出力先のサポートにより,この会話内容を本物のキャラクターに言われたらどの様に
感じるかもシミュレーションを行うことができた.
これにより文字だけでの出力しか行えない場合よりも,
よりリアルなコミュニケーションを行う人工知能や人工無脳の作成を目指すことができることがわかった.
