%------------------------------------------------------%
%- 結論
%------------------------------------------------------%

\section{結論}
今回作成した人工知能利用フレームワークを使い,考案したアルゴリズムを作成し
キャラクターと会話することで試すことはできたと考えている.
しかし,話題が固定された状況で返答アルゴリズムを書く必要があり,
現在どの話題で,どのアルゴリズムが回答するかを選択する部分のプログラムを書きたい場合は
その部分のプログラムを直接書き換えなくてはならない欠点がある.

\subsection{アルゴリズムの追加による出力の変化}
\ref{sec:addAl}の実行結果を見て分かる通り,アルゴリズムを追加することでその分野の話題に
なった時に追加したアルゴリズムが応答していることが分かる.
また,解析を行うアルゴリズムを追加したことによって,その分野のさらに詳しい話題の分析が\ref{sec:addAl}
の返答を見て分かる通り,可能となった.

\subsection{簡単にアルゴリズムを追加できたか}
考案したアルゴリズムを素早く試すために,アルゴリズムの追加する際の構造を簡単化したため,
特定の話題の時に解析や出力情報のアルゴリズムの作成と追加を非常に簡単に実現できる機構は
達成できた.

会話を行うことができるアルゴリズムを,この人工知能利用フレームワークでは3行のプログラムソースコード
の記述で実現できるため,非常に開発を開始するまでの時間を短くすることが出来,
また,Unityによる出力先のサポートにより,この会話内容を本物のキャラクターに言われたらどの様に
感じるかもシミュレーションを行うことができるため,文字だけでの出力しか行えない場合よりも
よりリアルなコミュニケーションを行う人工知能や人工無脳の作成を目指すことができることがわかった.


