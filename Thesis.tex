%======================================================%
%----- T-lab LaTeX Template File for "Thesis"
%
%--- 必ず README を読んで下さい (by sakura)
%======================================================%

%------------------------------------------------------%
%- Documentclass & Basic setting
%------------------------------------------------------%

\documentclass[a4paper,10pt,onecolumn,oneside,openany]{jsbook}

% Import original package
\usepackage{./conf/cs}
\usepackage{./conf/style}
\input{./conf/conf}

% Define basic information
% - ここら辺を空気読んで編集して
\author{田中 遼}										% 著者
\id{C0109343}											% 学籍番号
\juryoshoid C0109343									% 学籍番号 ({}を付けないで)
\title{学内クラウドの運用・改良および評価}				% タイトル短い人はこっち (\longTitle[AB] は空白に)
\longTitleA{}											% タイトル長い人はこっちに A,B に分けて書く
\longTitleB{}											% (\title は空白に)
\juryoshotitle{学内クラウドの\\運用・改良および評価}		% タイトル (こっちは長い場合は \\ で改行を入れる)
\courseofcs												% ここからはみんな同じ
\clab{田胡}												% 研究室名
\teacher{田胡 和哉}										% 指導教員
\date{2013年01月23日}									% 提出日
\cnendo{2012}											% 提出年度
\nendo{2012年度}										% 提出年度


%------------------------------------------------------%
%- Document
%------------------------------------------------------%

\begin{document}
%------------------------------------------------------%
%- 目次とか色々
%------------------------------------------------------%

% 表紙など
\ifTaA
	\makejuryosho
	\makecover
	\maketitle
	\jabst{
		%------------------------------------------------------%
%- 概要
%--- Abstractは文字数超過するとはみ出るので,
%--- 良い感じに調節して下さい.
%
%--- 生成された枠の一番したの行でおよそ800字ちょいになるように
%--- 文字サイズ・レイアウトを調整しています.
%------------------------------------------------------

現在機械学習の分野では人工知能や人工無脳,Deepleaningを始めとした
革新的な技術が注目を浴びている.
加えて人工知能を開発するだけではなく,その開発した人工知能を様々な分野で応用することにも注目が集まっている.
その例としてMicrosoftが開発を行ったりんなや人工知能を搭載したソフトバンクのロボットPepperなどがあり,
各企業が人工知能などを用いた製品に力を入れて開発している.

現在人工知能自体の開発を目的としている人工知能のフレームワークでは,人工知能自体を作成する過程
をサポートしており,作成した人工知能を応用する部分を支援しているフレームワークは非常に少ない.
そこで本研究では近年注目を浴びている「人工知能を有効的に利用する」ということを支援する事を目的とした,
「キャラクターと対話を行うことのできるプログラム」の開発を支援するフレームワークを提案する.

このフレームワークを用いることで「キャラクターとの対話を行うプログラム」を簡単に作成することが可能となり,
利用者が発案した対話アルゴリズムを素早く開発し,キャラクターとの対話という形で試すことが可能となる.

フレームワークの開発は必要な機能や全体の設計については教授の指導のもとで開発を行い,
開発環境の作成や実際の実装は同じ研究室のメンバーにアドバイスを頂きながら行った.

フレームワークを開発した結果キャラクターとの対話を行う事の利点として,人工知能を開発している際に気付きにくい点である「このキャラクター
に実際にこのセリフを言われたときにどの様な気持ちになるだろう」というキャラクターと実際に対話をすることで初めて
わかる情報やフィードバックを得られる点があることがわかった.

	}
	\makejabstract
\fi

% 目次 ・ 図目次 ・ 表目次
\ifToC
	\pagenumbering{roman}
	\setcounter{tocdepth}{3}
	\tableofcontents
	\listoffigures
	\listoftables
	\clearpage
	\pagenumbering{arabic}
\fi


%------------------------------------------------------%

\chapter{序論}
%------------------------------------------------------%
%- 序論
%------------------------------------------------------%

\section{背景}
メイジと呼ばれる魔法使いの貴族と平民が暮らす異世界ハルケギニア,トリステイン王国.
全寮制のトリステイン魔法学院では,いつものように\dq{ルイズ・ド・ラ・ヴァリエール}が授業中に魔法を失敗し,
教室を滅茶苦茶にしている.
{\bf 「魔法属性を持たないゼロのルイズ」}と他の生徒にからかわれる中,プライドが高いルイズはあくまでも強気.

そんなルイズは学校の使い魔召喚の儀式で,他の生徒が次々と召喚獣の呼び出しに成功する中,
こともあろうに平民らしき男の子を呼び出してしまう・・・?!\cite{louise}

\subsection{本テンプレについて}
レイアウトを調整して,表紙などを自動生成出来るようにしたテンプレなんだよ.
\LaTeXe の基本的な使い方は文献\cite{latex}とか {\it Google} で調べよう.
\footnote{やろうと思えば大体は出来るはずだ}

よく使うコマンドはこのテンプレ内でなるべく使っているつもりだけど,
「これがやりたいのに!」 という事があったら調べるか,誰かに聞こう.\\

基本ルールは

\begin{itemize}
	\item{レイアウト関係 (余白など) は統一しよう}
	\item{句読点は{\bf カンマ・ピリオド}で統一しよう}
	\item{chapter > section > subsection > subsubsection の順に章立てしよう}
\end{itemize}

\begin{enumerate}
	\item{慣れないうちはちょくちょくコンパイルしよう(どこでエラーが出るかわからない)}
	\item{基本的に数学記号関係などの記号はエスケープが必要な場合が多い}
	\item{コンパイルできない場合は,\dq{\$ make clean}をしてからコンパイルしてみよう}
	\item{作成者がレポート作成などで使用していたものを弄ったテンプレなので意味不明な定義もある}
	\item{気に食わない点があったら,作成者 \footnote{sakura : dev@saku-lab.net} に文句を言おう}
\end{enumerate}

って感じ☆ミ ウフフ☆おっけー♪

%------------------------------------------------------%

\section{目的}
こなぁぁぁあああああああああああああああああああああああああああああああああああああああああああああ
ぁぁああああああああああああああああああああああああああああああああああああああああああああああああ
ぁぁああああああああああああああああああああああああああああああああああああああああああああああああ
ぁぁああああああああああああああああああああああああああああああああああああああああああああああああ
ぁぁああああああああああああああああああああああああああああああああああああああああああああああああ
ぁぁああああああああああああああああああああああああああああああああああああああああああああああああ
ぁぁああああああああああああああああああああああああああああああああああああああああああああああああ
ぁぁああああああああああああああああああああああああああああああああああああああああああああああああ
ぁぁああああああああああああああああああああああああああああああああああああああああああああああああ
ぁぁああああああああああああああああああああああああああああああああああああああああああああああああ
ぁぁああああああああああああああああああああああああああああああああああああああああああああああああ
ぁぁああああああああああああああああああああああああああああああああああああああああああああああああ
ぁぁああああああああああああああああああああああああああああああああああああああああああああああああ
ぁぁああああああああああああああああああああああああああああああああああああああああああああああああ
ぁぁああああああああああああああああああああああああああああああああああああああああああああああああ
ぁぁああああああああああああああああああああああああああああああああああああああああああああああああ
ぁぁああああああああああああああああああああああああああああああああああああああああああああああああ
ぁぁああああああああああああああああああああああああああああああああああああああああああああああああ
ぁぁああああああああああああああああああああああああああああああああああああああああああああああああ
ぁぁああああああああああああああああああああああああああああああああああああああああああああああああ
ぁぁああああああああああああああああああああああああああああああああああああああああああああああああ
ぁぁああああああああああああああああああああああああああああああああああああああああああああああああ
ぁぁああああああああああああああああああああああああああああああああああああああああああああああああ
ぁぁああああああああああああああああああああああああああああああああああああああああああああああああ
ぁぁああああああああああああああああああああああああああああああああああああああああああああああああ
ぁぁああああああああああああああああああああああああああああああああああああああああああああああああ
ぁぁああああああああああああああああああああああああああああああああああああああああああああああああ
ぁぁああああああああああああああああああああああああああああああああああああああああああああああああ
ぁぁああああああああああああああああああああああああああああああああああああああああああああああああ
ぁぁああああああああああああああああああああああああああああああああああああああああああああああああ
ぁぁああああああああああああああああああああああああああああああああああああああああああああああああ
ぁぁああああああああああああああああああああああああああああああああああああああああああああああああ
ぁぁああああああああああああああああああああああああああああああああああああああああああああああああ
ぁぁああああああああああああああああああああああああああああああああああああああああああああああああ
ぁぁああああああああああああああああああああああああああああああああああああああああああああああああ
ゆきぃいいいいいいいいいいいいいいいいいいいいいいいいいいいいいいいいいいいいいいいいいいいいいいい
いいいいいいいいいいいいいいいいいいいいいいいいいいいいいいいいいいいいいいいいいいいいいいいいいい
いいいいいいいいいいいいいいいいいいいいいいいいいいいいいいいいいいいいいいいいいいいいいいいいいい
いいいいいいいいいいいいいいいいいいいいいいいいいいいいいいいいいいいいいいいいいいいいいいいいいい
いいいいいいいいいいいいいいいいいいいいいいいいいいいいいいいいいいいいいいいいいいいいいいいいいい
いいいいいいいいいいいいいいいいいいいいいいいいいいいいいいいいいいいいいいいいいいいいいいいいいい
いいいいいいいいいいいいいいいいいいいいいいいいいいいいいいいいいいいいいいいいいいいいいいいいいい
いいいいいいいいいいいいいいいいいいいいいいいいいいいいいいいいいいいいいいいいいいいいいいいいいい
いいいいいいいいいいいいいいいいいいいいいいいいいいいいいいいいいいいいいいいいいいいいいいいいいい
いいいいいいいいいいいいいいいいいいいいいいいいいいいいいいいいいいいいいいいいいいいいいいいいいい
いいいいいいいいいいいいいいいいいいいいいいいいいいいいいいいいいいいいいいいいいいいいいいいいいい
いいいいいいいいいいいいいいいいいいいいいいいいいいいいいいいいいいいいいいいいいいいいいいいいいい
いいいいいいいいいいいいいいいいいいいいいいいいいいいいいいいいいいいいいいいいいいいいいいいいいい
いいいいいいいいいいいいいいいいいいいいいいいいいいいいいいいいいいいいいいいいいいいいいいいいいい
いいいいいいいいいいいいいいいいいいいいいいいいいいいいいいいいいいいいいいいいいいいいいいいいいい
いいいいいいいいいいいいいいいいいいいいいいいいいいいいいいいいいいいいいいいいいいいいいいいいいい
いいいいいいいいいいいいいいいいいいいいいいいいいいいいいいいいいいいいいいいいいいいいいいいいいい
いいいいいいいいいいいいいいいいいいいいいいいいいいいいいいいいいいいいいいいいいいいいいいいいいい
いいいいいいいいいいいいいいいいいいいいいいいいいいいいいいいいいいいいいいいいいいいいいいいいいい
いいいいいいいいいいいいいいいいいいいいいいいいいいいいいいいいいいいいいいいいいいいいいいいいいい
いいいいいいいいいいいいいいいいいいいいいいいいいいいいいいいいいいいいいいいいいいいいいいいいいい
いいいいいいいいいいいいいいいいいいいいいいいいいいいいいいいいいいいいいいいいいいいいいいいいいい
いいいいいいいいいいいいいいいいいいいいいいいいいいいいいいいいいいいいいいいいいいいいいいいいいい
いいいいいいいいいいいいいいいいいいいいいいいいいいいいいいいいいいいいいいいいいいいいいいいいいい
いいいいいいいいいいいいいいいいいいいいいいいいいいいいいいいいいいいいいいいいいいいいいいいいいい
いいいいいいいいいいいいいいいいいいいいいいいいいいいいいいいいいいいいいいいいいいいいいいいいいい
いいいいいいいいいいいいいいいいいいいいいいいいいいいいいいいいいいいいいいいいいいいいいいいいいい
いいいいいいいいいいいいいいいいいいいいいいいいいいいいいいいいいいいいいいいいいいいいいいいいいい
いいいいいいいいいいいいいいいいいいいいいいいいいいいいいいいいいいいいいいいいいいいいいいいいいい
いいいいいいいいいいいいいいいいいいいいいいいいいいいいいいいいいいいいいいいいいいいいいいいいいい
いいいいいいいいいいいいいいいいいいいいいいいいいいいいいいいいいいいいいいいいいいいいいいいいいい
いいいいいいいいいいいいいいいいいいいいいいいいいいいいいいいいいいいいいいいいいいいいいいいいいい
いいいいいいいいいいいいいいいいいいいいいいいいいいいいいいいいいいいいいいいいいいいいいいいいいい
ねぇ.


%------------------------------------------------------%

\chapter{クラウド・コンピューティング}
%------------------------------------------------------%
%- クラウド・コンピューティング
%------------------------------------------------------%

\section{クラウド・コンピューティングとは}
クラウドって何なんだろうね.俺にもよくわかんないや.

\subsection*{てーぶるとか図とか}
次の\tabref{example}に色々と示す.

\begin{table}[tbh]
	\caption{可愛いのか} \label{tab:example}
	\begin{center}
		\begin{tabular}[htb]{c|c}
		\hline
		名前 & 印象 \\
		\hline
		ルイズ・フランソワーズ・ル・ブラン・ド・ラ・ヴァリエール & かわいい \\ \hline
		タバサ & かわいい \\ \hline
		キュルケ・アウグスタ・フレデリカ・フォン・アンハルツ・ツェルプストー & しらん \\ \hline
		シエスタ & 神 \\ \hline
		アンリエッタ & かわいい \\ \hline
		\end{tabular}
	\end{center}
\end{table}

次の\figref{t-lab}に何か示す.

\figPst{70}{t-lab}{田胡研のロゴ的な何か}

\dq{make}でコンパイルしてる人はルールに書いてあるからそのままでいい.
\dq{make}を使ってない人は図のあるディレクトリにいって \dq{\$ ebb *.png} や \dq{\$ xbb *.png} を実行しないと
サイズがおかしくなるよ.


\ifSRC
	\subsection*{ソースコードとか}
	ソースコードはこんな感じで出せるよ(\srcref{louise}).

	\srcPst{Java}{Louise.java}{louise}{ルイズコピペ}
\fi

おなか減ったなぁ.


%------------------------------------------------------%

% ... 章ごとに書くと楽だよ

%------------------------------------------------------%

\chapter*{謝辞}
\addcontentsline{toc}{chapter}{謝辞}
%------------------------------------------------------%
%- 謝辞
%------------------------------------------------------%

本論文を作成するにあたって,監督いただいた教授および参考文献を作成した方々に感謝いたします.


%------------------------------------------------------%
%- References
%------------------------------------------------------%

\begin{thebibliography}{99}
	\bibURL{louise}{ゼロの使い魔制作委員会}{ゼロの使い魔公式ウェブサイト}{http://www.zero-tsukaima.com/zero/index.html}{2012/12/28}
	\bib{latex}{奥村晴彦 著}{\LaTeXe 美文書作成入門 改訂第3版}{技術評論社 2004, 403pp}
\end{thebibliography}

\end{document}
