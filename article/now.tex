%------------------------------------------------------%
%- 現状
%------------------------------------------------------%
近年様々な機械学習や人工知能,人工無脳といった分野が注目を浴びている,
ここではそれらの現状について説明する.

\section{近年の機械学習}
まず初めに,機械学習とはデータから反復的に学習し,そこに潜むパターンを見つけ出すことであり,
近年その機械学習を会話に用いる動きがある.

\subsection{人工無能}
人工無脳\cite{muno}とは人工知能に対応する用語で,英語圏ではchatterbotもしくはchatbotと呼ばれ,
その訳語として会話ボットあるいはおしゃべりボットとも呼ばれることがある.
人工無脳はテキストや音声などを用いて,会話をシミュレートすることが可能なプログラムである.
一見知的に人間の様な応答をしている様に見えるが,多くの場合は会話の中の特定のキーワードを拾い
そのキーワードに対応する返答を行っている場合が多いのが現状である.

\subsection{人工知能}
人工知能\cite{tino}とは人工的にコンピュータ上などで人間と同様の知能を実現させようという試み,
或いはそのための一連の基礎技術を指すものであり,1956年にダートマス会議でジョン・マッカーシーにより命名された.
人工知能の定義は未だ不確定な部分が多く,完全に正確な定義は存在していないのが現状である.
この人工知能は人工無脳とは異なりただ単にキーワードを広くだけではなく,
取得した情報を用いて解析や情報の取得を行い,状況に応じた返答などが可能なプログラムのことを指すことが多い.

\subsection{Neural Network}
Neural Networkとは,脳細胞を構成する「Neuron(ニューロン)」の活動を単純化したモデルであり,これを利用する
ことによって人間の思考をシミュレーションすることができ,人工知能分野で活用されている.

\subsection{DeepLearning}
DeepLearning\cite{deep}とは多層構造のニューラルネットワークの機械学習の事であり,
ニューラルネットワークを多層積み重ねたモデルを機械学習させればディープラーニングとなる.
また,一般的には3層以上のニューラルネットワークがあるものとされている.

第一層から入った情報は,より深い3層へと行く過程で学習が行われ,その学習を行っていくことで
概念を認識する特徴量と呼ばれる重要な変数を自動で発見することができる.

このDeepLearningにより,東京大学の松尾豊准教授\cite{boom}を始めとする機械学習や人工知能の研究者は
「AI研究に関する大きなブレイクスルーであり、学習方法に関する技術的な革新である」と指摘している.

\section{一般的な人工知能開発フレームワーク}
現在一般的な人工知能を開発するフレームワークとしてchainerやGoogleのTensorFlowなどがある.

Chainerは,Preferred Networksが開発したニューラルネットワークを実装するためのライブラリであり,
人工知能自体の開発を行う際に高速な計算が可能なことや,様々なタイプのニューラルネットを実装可能である.
またネットワーク構造を直感的に記述できる利点がある.

GoogleのBrain Teamの研究者たちが作った機械学習ライブラリであるTensor Flowは,
Python APIとC++インターフェイス一式が用意されているため開発を行う際に非常に有効だと考えられる.

これらの人工知能開発フレームワークは実際に開発を行うときに非常に有効であり,googleの検索アルゴリズムや
データ分析などの様々な分野での応用を試みる動きがある.