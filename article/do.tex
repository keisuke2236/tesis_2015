%------------------------------------------------------%
%- 実行結果
%------------------------------------------------------%

\section{Unityの出力画面の図}
実際にUnityでのキャラクターとの対話を行う際の画面は以下のの様な画面で対話を行うことが出来る.

\figPst{100}{doUnity}{キャラクターとの対話画面}

\figref{doUnity}の画面にて実際にキャラクターと対話を行う形式で作成した人工知能のアリゴリズムが
正しく動作しているのかを確認することができる.\\

%------------------------------------------------------%

\section{実際の会話}
実際にこの人工知能利用フレームワークを用いて会話を行った例を以下の\tabref{Chat}に示す.\\

\begin{table}[tbh]
	\caption{キャラクターとの対話例} \label{tab:Chat}
	\begin{center}
		\begin{tabular}[htb]{c|c|c}
		\hline
		ユーザーの発言 & キャラクターからの返答 & 動作 \\
		\hline
		カルボナーラ作ってるんだ & 隠し味にチョコレートいれちゃう? & 悪巧みの動作 \\
		うわーマシンヘラ強すぎ,負けたよ & あーぁーゲームオーバーじゃん & がっかり \\
		\hline
		\end{tabular}
	\end{center}
\end{table}

以上の\tabref{Chat}の様にユーザーが話しかけた内容に対して返答を行う仕組みが実装されていることがわかる.\\

また,ユーザーが話しかけた内容の分野を解析し,その分野に対応している解析や出力アルゴリズムが対応しているため,
その分野のより詳しい会話を行うことができるもの\tabref{Chat}を見るとわかる.\\


\section{アルゴリズムを追加した後の会話}\label{sec:addAl}
実際にアルゴリズムを追加した後に会話を行うとどの様な変化があるかを検証してみたいと思う.\\

今回追加を行うアルゴリズムの分野は「パズドラ」
\footnote{パズル\&ドラゴンズ』は、ガンホーオンラインエンターテイメントから配信されている
iOS,Android,Kindle Fire用ゲームアプリ(パズルRPG).略称は『パズドラ』.}
という分野をもつ解析と出力を行うアルゴリズムであり,解析を行う部分の話題を解析する分野に対して,
は簡易的なパズドラというゲーム内でよく行われる会話の話題を推定するアルゴリズムを追加した.\\

出力を行う,返答アルゴリズムにも「パズドラ」の話題に対応,特化する出力アルゴリズムを追加した.\\

このパズドラのアルゴリズムを追加した後の会話を以下の\tabref{afterChat}に示す.\\

\begin{table}[tbh]
	\caption{アルゴリズム追加後の会話} \label{tab:afterChat}
	\begin{center}
		\begin{tabular}[htb]{c|c|c}
		\hline
		ユーザーの発言 & キャラクターからの返答 & 動作 \\
		\hline
		うわーマシンヘラ強すぎ,負けたよ & 魔法石課金してコンテニューしよ! & 悪巧みの動作 \\
		\hline
		\end{tabular}
	\end{center}
\end{table}

このパズドラは現在実装してある,ゲームの分野の中のゲームのタイトル名であり,
ただ単にゲームに関連のある単語であれば,ゲームの分野が解析を行いますが,
ここでパズドラというゲーム固有の単語である「マシンへラ\footnote{スマートフォンゲーム
「パズドラ」2015年12月26日現在最大のヒットポイントを所有する敵キャラクター}」
という単語を含めることで,\tabref{afterChat}の様に
パズドラの解析と出力を行うアルゴリズムが選択されます.
\\
返答返答内容も「負けたよ」という発言から,同スマホゲームの課金アイテムである「魔法石」を
用いて再度,敵に対して挑戦しようという返答を行うことができています.
\\
この様に料理やゲームといった種類を横に広げていくアルゴリズムの追加方法と,
料理やゲームのさらに細かい話題に対して対応させるアルゴリズムの追加方法などがあります.

