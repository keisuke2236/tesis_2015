%------------------------------------------------------%
%- 現状
%------------------------------------------------------%
近年人工知能などの分野が注目を浴びており,ここではそれらの現状について説明する.

\section{人工知能}
人工知能\cite{tino}とは人工的にコンピュータ上などで人間と同様の知能を実現させようという試み,
或いはそのための一連の基礎技術を指すものであり,1956年にダートマス会議でジョン・マッカーシーにより命名された.
人工知能の定義は未だ不確定な部分が多く,完全に正確な定義は存在していないのが現状である.
この人工知能は人工無脳とは異なりただ単にキーワードを広うだけではなく,
機械学習によって取得した情報を用いて解析や情報の取得を行い,状況に応じた返答などが可能なプログラムのことを指すことが多い.
機械学習とは入力されたデータから反復的に特徴を学習することでそこに潜むパターンを見つけ出す技術のことであり,
人工知能には必要不可欠な要素である.
また学習を行わず特定のキーワードを拾い返答するものを,人工無脳または対話ボットという.\cite{muno}

\section{DeepLearning}
DeepLearning\cite{deep}とは多層構造のニューラルネットワークの機械学習の事であり,
ニューラルネットワークを多層積み重ねたモデルを機械学習させればディープラーニングとなる.
またNeural Networkとは脳細胞を構成する「Neuron(ニューロン)」の活動を単純化したモデルであり,これを利用する
ことによって人間の思考をシミュレーションすることができるものである.
一般的にはこのニューラルネットワークが3層以上のあるものをディープラーニングと呼んでいる.

第1層から入った情報は,より深い3層へと学習を行うことにより,
概念を認識する特徴量と呼ばれる重要な変数を自動で発見することができる.

このDeepLearningにより,東京大学の松尾豊准教授\cite{boom}を始めとする機械学習や人工知能の研究者は
「AI研究に関する大きなブレイクスルーであり、学習方法に関する技術的な革新である」と指摘している.

\section{一般的な人工知能開発フレームワーク}\label{sec:ippan}
現在一般的な人工知能を開発するフレームワークとしてchainerやGoogleのTensorFlowなどがある.

Chainerは,Preferred Networksが開発したニューラルネットワークを実装するためのライブラリであり,
人工知能自体の開発を行う際に高速な計算が可能なことや,様々なタイプのニューラルネットを実装可能であり,
またネットワーク構造を直感的に記述できる利点がある.

GoogleのBrain Teamの研究者たちが作った機械学習ライブラリであるTensor Flowは,
Python APIとC++インターフェイス一式が用意されているため開発を行う際に非常に有効だと考えられる.

これらの人工知能開発フレームワークは実際に開発を行うときに非常に有効であり,googleの検索アルゴリズムや
データ分析などの様々な分野での応用を試みる動きがある.

\section{人工知能の様々な分野への活用}
現在の人工知能にはデータ分析からGoogleの検索アルゴリズムなどほぼ無限の用途があり,様々な分野への応用が試みられている.
現在人工知能の応用を試みている分野はあまりにも多く,GoogleのCEOは
「20年後,あなたが望もうが,望むまいが現在の仕事のほとんどが機械によって代行される.」\cite{ceo}と発言している.
ここで現在行われている活用例を,
以下の\figref{brainmap}にO2O INNOVATION LAB\cite{lab}やPepperWorld2016\cite{Pepper2}を参考に複数抜粋したものを示す.

\figPst{150}{brainmap}{人工知能の様々な活用例}

\figref{brainmap}のように人工知能はすでに様々な分野で活用されている.
今回は大きく2つに分けて人々の生活をより良いものにしていくカジュアルな分野での活用と,ビジネスに応用することで
より効果的なサービスを提供することへの活用の2種類に分けて図示した.

初めに,ビジネス分野での人工知能の活用例としてGoogleの検索アルゴリズムに活用されている例と,
ネット上の広告枠にどの広告を表示するか決めているアルゴリズム,及び自動車の自動走行に活用している例を紹介する.

Googleの検索アルゴリズムに活用している例では機械学習を用いて検索結果の最適化を行っており,
ウェブサイトの内容を学習することによってより優良なウェブサイトやユーザーの役に立つウェブサイトが検索結果の上位に表示されるように設計されている.
また質の低いコンテンツや有害コンテンツの排除も行っている.

ネット上の広告枠にどの広告を表示するか決めているアルゴリズムRTB(RealTimeBitting)について紹介する.
RTBは人工知能を用いた広告の競売システムであり,
アクセスして来たユーザーがどのような分野に興味があるかと言ったことや年齢及び性別など瞬時に算出し,
複数の広告主に1つの広告枠をいくらで入札するかを瞬時に問い合わせて1番高値で入札した広告をユーザーが閲覧するウェブサイトに表示するシステムである.

自動車の自動走行はアメリカ合衆国カリフォルニア州にある半導体メーカーのNvidia Corporationをはじめとする企業が研究を行っており,
自動運転の開発プラットフォームも発表している.
ディープラーニング技術と画像認識機能と組み合わせる事で,救急車と配送トラックといった車種の違いや,駐車中の自動車が発進しようとしているか
どうかを見分けるなど,まるで人間が目で見て判断をしているようなことが可能であり,
そのような微妙な違いに対応することで自律走行が実現できるというものである.

次に人々の生活の中で密に関わることができるカジュアルな人工知能の活用例について紹介する.
カジュアルな人工知能の活用方法の一例として\figref{brainmap}にはゲームAIへの応用と夕飯の提案及び人間との対話を挙げた.

ゲームAIへの応用では将棋を行う人工知能が名人棋士を敗る段階にまで達しており,「将棋電王戦」と称して人工知能とプロ棋士の戦いが
毎年繰り広げられているが,2014年はプロ棋士5人でわずか1勝しかできないなど将棋に関して言えば人工知能はすでに人間を超越しているといえる.

また無限の食材の組み合わせを提案するIBMが開発を行った人工知能「シェフ・ワトソン」では夕飯のメニューの提案なども行われており,
人間が現在作れている料理の種類は「シェフ・ワトソン」が提案可能なすべての料理のわずか0.0000001\%という試算もあるという.
このように人間には思いつかないような料理を提案できる人工知能も登場している.

最後に人間と対話(コミュニケーション)を行うことのできる人工知能について紹介する.
人工知能には様々な用途があることを紹介して来たが,そのうちの一つとして人工知能を用いた会話システムという活用方法もある.
コンピュータと人間とのコミュニケーションに人工知能を用いることで,より円滑なコミュニケーションを行うことができると考えられている.

以上のように様々な分野に人工知能を活用する流れがある中で,本研究ではその活用方法のうちの1つである人間とのコミュニケーションへの
人工知能の活用を行っていくこととする.

\section{知能の開発をサポートする既存フレームワークの現状}
\ref{sec:ippan}章で説明したようなフレームワークが登場する前までは,人工知能の開発は非常に手間と時間のかかるものであった.
それに加えてchainerはGoogleが開発を行った非常に高度な技術を用いるフレームワークであり,
個人でこのような機構を持った人工知能の開発を行いたいと考えたとしても事実上不可能であるのが現状である.

しかし様々なフレームワークの登場により人工知能自体を開発する事が非常に簡単になった事に加え,
より高度な技術を用いた人工知能の開発を行う事が可能になった.
このように個人でも高度な人工知能の開発が可能になった事によって現在非常に人工知能の分野が注目を浴びており,
本研究ではこの人工知能の技術を機械と人間とのコミュニケーションに応用したいと考えた.

そこで人間とのコミュニケーションを行う際に最も適したインターフェイスとはどのようなものだろうかということを考えた,
その結果インターフェイスを人型のキャラクターとの対話にするという結論となった.

しかし既存のフレームワークには,開発した人工知能を用いてキャラクターと会話を行うというところまでをサポートするフレームワークがない.
キャラクターとの対話で人工知能を試したい場合は対話を行うアルゴリズムを作成するのに加えて,
キャラクターとの対話を行うインターフェイスの準備やキャラクターを動作させるための動作ファイルなど様々な準備が必要となるのが現状である.

このように既存のフレームワークを用いて簡単に高性能なアルゴリズムを作ったとしても,それを気軽に楽しむ為の環境がないのが現状である.

\section{人工知能のアプリケーションへの応用}
現在Microsoftのりんな\cite{rinna}やソフトバンクのPepper\cite{Pepper}などの登場により,人工知能を応用した対話に注目が集まっている.

Micsoroftのりんなは,Microsoftが開発を行った人工知能であり,話しかけると女子高生のような返答を返してくれるものである.
現在LINE\footnote{LINEとは韓国最大のIT企業「NHN」の日本法人「LINE株式会社」が提供しているスマートフォン(iPhoneやAndroid)、ガラケー(フィーチャーフォン)、パソコンに対応したコミュニケーションアプリケーションです.\cite{line}}
アプリ上で公式アカウントを持っており,りんなを友達登録しているユーザー数は2016年1月13日現在では2,167,730人である.
これは日本の人口が現在1億2688万人\cite{humen}なので総人口の1.7\%にあたり,およそ100人に1人または2人がこの人工知能りんなとの対話を
行っているというのが現状である.

ソフトバンクのPepperをはじめとしたロボットに関しては,2016年現在テレビCMが行われたり
様々なPepperの用途を紹介するPepperWorld2016\cite{Pepper2}なども開催されており非常に注目を浴びている.

ロボットを人とのコミュニケーションに応用しようという動きの一例としては,
TIS株式会社と国立大学法人奈良先端科学技術大学院大学も昨年の2015年11月から共同で「マルチモーダルインタラクションを用いた
パブリックスペースにおける対話処理」について研究を開始している例があげられる.\cite{tis}
この研究は音声や画像や言語などの複数の情報源を用いて,ロボットと人とのスムーズなコミュニケーションの実現を目指す研究であり,
人間とコンピュータのコミュニケーションを実現する本格的なプロジェクトの一つであると言える.
以下の\figref{tis}にその研究のイメージとしてウェブサイト上に記載されている図を示す.

\figPst{100}{tis}{研究イメージ}

\figref{tis}のようにロボットと人とのスムーズなコミュニケーションの実現を目指していることがわかり,
すでに開発されているPepperの技術を他のロボットにも応用しようという動きがあることがわかる.

またPepperに関してはソフトバンクショップを初めとした一般家庭や喫茶店などですでに幅広く活用されており,
Pepper World 2016\cite{Pepper2}というイベントでは様々なPepperの利用用途を紹介している.
以下の\figref{usePepper}にそのウェブサイト上で紹介されている利用例の画像を挙げる.

\figPst{150}{usePepper}{PepperWorld2016で紹介されている具体例}

PepperWorld2016ではPepperだけで50種類の利用用途を示しており,その具体例を示した\figref{usePepper}
を見ると,今まで機械が行って来なかった教育や介護といった人とのコミュニケーションをとる必要とする用途が多い事がわかる.
このイベントからも人工知能を人とのコミュニケーションに活用しようという動きがある事がわかる.

イベントなどでPepperの利用用途が紹介されているだけではなく,すでにPepperは現場で利用されている.
その具体例として東京都八王子市のコメダ珈琲店\cite{coffe}にて,1スタッフとして来店された方への挨拶などを行っている姿を先日見かけた.

そこで八王子市の喫茶店コメダ珈琲店のスタッフに「Pepper君を導入してから何か変わりましたか?」とインタビューを行ったところ,
スタッフからは「Pepperを導入してからお子様が来店された時に喜び,Pepper君と話している姿をみます」といった声や
「店内が混雑している時の待ち時間にお客様がお話ししており,待ち時間の退屈さを紛らわせてくれている」といった意見を聞く事が出来た.

\figPst{100}{Pepper}{東京都八王子市のコメダ珈琲店で活躍するPepper君}

\figref{Pepper}のように珈琲店では帽子をかぶり来店されたお客様とコミュニケーションを行っている.
また私がPepper君の見える席に座っていたところ,来店される方とPepper君の目が会うたびに人々が笑顔になったり,
驚いていたりしていた.
これに加えてお年寄りの方をはじめとする様々な方が親身にPepper君に話しかけている光景を目にしており,
人工知能との対話に対する抵抗は人工知能への理解が進んだためかあまりないように見えた.
特に今後産まれてくる人々は人工知能との会話が日常のこととなっていくことが予想できる.

このようにテキストチャットの形式で対話を行う事ができるりんなの登場や人工知能を活用したロボットPepperの登場によって人工知能
に非常に注目が集まっており,このような対話システムを企業だけではなく個人でも開発したいという需要は高まっていると考えられる.

このようなロボットに対して抵抗がなくなっていることや人工知能との会話を楽しむ傾向があることなどに加えて,日本にはオタク
\footnote{おたく(オタク、ヲタク)とは、1970年代に日本で誕生した呼称であり大衆文化の愛好者を指す。元来はアニメ・SF・パソコンなどの、
なかでも嗜好性の強い趣味や玩具、の愛好者の一部に使われていた術語であったが、バブル景気期に一般的に知られはじめた。\cite{ota}}
という文化がある.
ディップ株式会社の調査\cite{dip}によると日本人口の約40%はオタクであるという調査結果となっており,非常にアニメやゲームやパソコンなどに
興味を持っている人口が多い事がわかる.

中でもアニメに関しては毎年市場が伸びており,2015年度に株式会社メディア開発綜研が調査した結果によるとその市場は2595
億円(前年比106.9\%)と過去最大の規模になったという.\cite{anime}

このように現在日本では,人工知能を搭載したロボットや女子高生人工知能との会話,アニメが注目されている事がわかった.

そしてなぜアニメがここまで注目されているのかについて考察すると,
それはアニメキャラクターの”かっこよさ”や”可愛さ”といった現実にはない二次元キャラクター特有の
”人々の理想”がそこにあるからであると私は分析している.

\figPst{100}{nowUnity}{理想の見た目のキャラクターとの理想の会話例}

そこで\figref{nowUnity}のように理想の見た目をしているキャラクターに,
理想の会話をしてくれる人工知能を加えたら間違いなくこのオタクの人々を感動させる事ができるという仮説を立てた.

それに加えて近年人工知能の開発自体は非常に楽になったが,
それを実際に試す環境までを提供するフレームワークがないという問題点がある.

今回はこの仮説と問題点をもとに二次元のアニメキャラクターの人工知能を素早く開発し,二次元のキャラクターと気軽に対話を行う環境までをサポートする
フレームワークを提案する.

