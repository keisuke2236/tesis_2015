%------------------------------------------------------%
%- 概要
%--- Abstractは文字数超過するとはみ出るので,
%--- 良い感じに調節して下さい.
%
%--- 生成された枠の一番したの行でおよそ800字ちょいになるように
%--- 文字サイズ・レイアウトを調整しています.
%------------------------------------------------------%

現在機械学習などの分野では,人工知能や人工無脳,また,Deepleaningを始めとした
革新的な技術が注目を浴びている.
\\
その人工知能を開発するだけではなく,その開発した人工知能を用いて特定の分野で
応用することにも注目が集まっており,人工知能を搭載したソフトバンクのロボットPepperなど,
各企業が人工知能やを用いた製品を開発している.
\\
そこで本研究では人工知能の活用を目的としてキャラクターと対話を行うことのできる
システムの開発を支援する人工知能利用フレームワークを開発した.
\\
既存の人工知能フレームワークでは,人工知能自体を作ることを支援するフレームワークが
多い中で,本研究では,近年注目を浴びている事項知能を有効的に利用するためのフレーワムワークを作成した.

このフレームワークを用いることで,人工知能の利用例の1つであるキャラクターとの対話システムを簡単に
作成することが可能となり,利用者が発案した対話アルゴリズムをすぐに開発し,キャラクターとの対話
という形で実現することができる.\\

また,キャラクターとの対話を行うことで,人工知能を開発していて気付きにくい点である,このキャラクター
に実際にこのセリフを言われたときに,どの様な気持ちになるだろうというキャラクターを目の前に対話をすることで初めて
わかる情報,フィードバックを開発者に伝えることができる点もこの人工知能利用フレームワークの特徴の1つである.