\documentclass[a4paper,10pt,onecolumn,oneside,openany]{jsbook}

% Import original package
\usepackage{./conf/cs}
\usepackage{./conf/style}
\input{./conf/conf}

% Define basic information
% - ここら辺を空気読んで編集して
\author{寺田 佳輔}										% 著者
\id{C0112336}											% 学籍番号
\juryoshoid C0112336									% 学籍番号 ({}を付けないで)
\title{人工知能ハブの開発と評価}				% タイトル短い人はこっち (\longTitle[AB] は空白に)
\longTitleA{}											% タイトル長い人はこっちに A,B に分けて書く
\longTitleB{}											% (\title は空白に)
\juryoshotitle{人工知能ハブの\\開発・利用および評価}		% タイトル (こっちは長い場合は \\ で改行を入れる)
\courseofcs												% ここからはみんな同じ
\clab{田胡}												% 研究室名
\teacher{田胡 和哉}										% 指導教員
\date{2016年01月18日}									% 提出日
\cnendo{2015}											% 提出年度
\nendo{2015年度}										% 提出年度


%------------------------------------------------------%
%- Document
%------------------------------------------------------%

\begin{document}
%------------------------------------------------------%
%- 目次とか色々
%------------------------------------------------------%

% 表紙など
\ifTaA
	\makejuryosho
	\makecover
	\maketitle
	\jabst{
		%------------------------------------------------------%
%- 概要
%--- Abstractは文字数超過するとはみ出るので,
%--- 良い感じに調節して下さい.
%
%--- 生成された枠の一番したの行でおよそ800字ちょいになるように
%--- 文字サイズ・レイアウトを調整しています.
%------------------------------------------------------

現在機械学習の分野では人工知能や人工無脳,Deepleaningを始めとした
革新的な技術が注目を浴びている.
加えて人工知能を開発するだけではなく,その開発した人工知能を様々な分野で応用することにも注目が集まっている.
その例としてMicrosoftが開発を行ったりんなや人工知能を搭載したソフトバンクのロボットPepperなどがあり,
各企業が人工知能などを用いた製品に力を入れて開発している.

現在人工知能自体の開発を目的としている人工知能のフレームワークでは,人工知能自体を作成する過程
をサポートしており,作成した人工知能を応用する部分を支援しているフレームワークは非常に少ない.
そこで本研究では近年注目を浴びている「人工知能を有効的に利用する」ということを支援する事を目的とした,
「キャラクターと対話を行うことのできるプログラム」の開発を支援するフレームワークを提案する.

このフレームワークを用いることで「キャラクターとの対話を行うプログラム」を簡単に作成することが可能となり,
利用者が発案した対話アルゴリズムを素早く開発し,キャラクターとの対話という形で試すことが可能となる.

フレームワークの開発は必要な機能や全体の設計については教授の指導のもとで開発を行い,
開発環境の作成や実際の実装は同じ研究室のメンバーにアドバイスを頂きながら行った.

フレームワークを開発した結果キャラクターとの対話を行う事の利点として,人工知能を開発している際に気付きにくい点である「このキャラクター
に実際にこのセリフを言われたときにどの様な気持ちになるだろう」というキャラクターと実際に対話をすることで初めて
わかる情報やフィードバックを得られる点があることがわかった.

	}
	\makejabstract
\fi

% 目次 ・ 図目次 ・ 表目次
\ifToC
	\pagenumbering{roman}
	\setcounter{tocdepth}{3}
	\tableofcontents
	\listoffigures
	\listoftables
	\clearpage
	\pagenumbering{arabic}
\fi


%------------------------------------------------------%

\chapter{現状}
%------------------------------------------------------%
%- 現状
%------------------------------------------------------%
近年人工知能などの分野が注目を浴びており,ここではそれらの現状について説明する.

\section{人工知能}
人工知能とは人工的にコンピュータ上などで人間と同様の知能を実現させようという試み,
或いはそのための一連の基礎技術を指すものであり,1956年にダートマス会議でジョン・マッカーシーにより命名された.
人工知能の定義は未だ不確定な部分が多く,完全に正確な定義は存在していないのが現状である.
この人工知能は人工無脳とは異なりただ単にキーワードを広うだけではなく,
機械学習によって取得した情報を用いて解析や情報の取得を行い,状況に応じた返答などが可能なプログラムのことを指すことが多い.
また学習を行わず特定のキーワードを拾い返答するものを,人工無脳または対話ボットという.

\section{機械学習}
機械学習とは入力されたデータから反復的に特徴を学習することでそこに潜むパターンを見つけ出す技術のことであり,
人工知能には必要不可欠な要素である.

\section{Neural Network}
Neural Networkとは,脳細胞を構成する「Neuron(ニューロン)」の活動を単純化したモデルであり,これを利用する
ことによって人間の思考をシミュレーションすることができる.
現在Neural NetWorkは主にDeepLearningで利用されている.

\section{DeepLearning}
DeepLearning\cite{deep}とは多層構造のニューラルネットワークの機械学習の事であり,
ニューラルネットワークを多層積み重ねたモデルを機械学習させればディープラーニングとなる.
また,一般的には3層以上のニューラルネットワークがあるものとされている.

第一層から入った情報は,より深い3層へと行く過程で学習が行われ,その学習を行っていくことで
概念を認識する特徴量と呼ばれる重要な変数を自動で発見することができる.

このDeepLearningにより,東京大学の松尾豊准教授\cite{boom}を始めとする機械学習や人工知能の研究者は
「AI研究に関する大きなブレイクスルーであり、学習方法に関する技術的な革新である」と指摘している.

\section{一般的な人工知能開発フレームワーク}
現在一般的な人工知能を開発するフレームワークとしてchainerやGoogleのTensorFlowなどがある.

Chainerは,Preferred Networksが開発したニューラルネットワークを実装するためのライブラリであり,
人工知能自体の開発を行う際に高速な計算が可能なことや,様々なタイプのニューラルネットを実装可能であり,
またネットワーク構造を直感的に記述できる利点がある.

GoogleのBrain Teamの研究者たちが作った機械学習ライブラリであるTensor Flowは,
Python APIとC++インターフェイス一式が用意されているため開発を行う際に非常に有効だと考えられる.

これらの人工知能開発フレームワークは実際に開発を行うときに非常に有効であり,googleの検索アルゴリズムや
データ分析などの様々な分野での応用を試みる動きがある.

\section{知能の開発をサポートする既存フレームワークの現状}
上記で説明したようなフレームワークが登場する前までは,人工知能の開発は非常に手間と時間のかかるものであった.
それに加えてchainerはGoogleが開発を行った非常に高度な技術を用いるフレームワークであり,
個人でこのような機構を持った人工知能の開発を行いたいと考えたとしても事実上不可能であるのが現状である.

しかし様々なフレームワークの登場により人工知能自体を開発する事が非常に簡単になった事に加え,
より高度な技術を用いた人工知能の開発を行う事が可能になった.
このように個人でも高度な人工知能の開発が可能になった事により,現在非常に人工知能の分野が注目を浴びている.

しかし既存のフレームワークには,開発した人工知能を用いてキャラクターと会話を行うというところまでをサポートするフレームワークがない.
キャラクターとの対話で人工知能を試したい場合は対話を行うアルゴリズムを作成するのに加えて,
キャラクターとの対話を行うインターフェイスの準備やキャラクターを動作させるための動作ファイルなど様々な準備が必要となるのが現状である.

このように既存のフレームワークを用いて簡単に高性能なアルゴリズムを作ったとしても,それを気軽に楽しむ為の環境がないのが現状である.

\newpage

\section{人工知能のアプリケーションへの応用}
現在Microsoftのりんな\cite{rinna}やソフトバンクのpepper\cite{pepper}などの登場により,人工知能を応用した対話に注目が集まっている.

Micsoroftのりんなは,Microsoftが開発を行った人工知能であり,話しかけると女子高生のような返答を返してくれるものである.
現在LINE\footnote{LINEとは韓国最大のIT企業「NHN」の日本法人「LINE株式会社」が提供しているスマートフォン(iPhoneやAndroid)、ガラケー(フィーチャーフォン)、パソコンに対応したコミュニケーションアプリケーションです.\cite{line}}
アプリ上で公式アカウントを持っており,りんなを友達登録しているユーザー数は2016年1月13日現在では2,167,730人である.
これは日本の人口が現在1億2688万人\cite{humen}なので総人口の1.7\%にあたり,およそ100人に1人または2人がこの人工知能りんなとの対話を
行っているというのが現状である.

ソフトバンクのpepperに関しては2016年現在,テレビCMなども頻繁に行われており,
ソフトバンクショップを初めとして一般家庭や喫茶店などで幅広く活用されている.

八王子市の喫茶店コメダ珈琲店\cite{coffe}のスタッフに「pepper君を導入してから何か変わりましたか?」とインタビューを行ったところ,
スタッフからは「pepperを導入してからお子様が来店された時に喜び,pepper君と話している姿をみます」といった声や
「店内が混雑している時の待ち時間にお客様がお話ししており,待ち時間の退屈さを紛らわせてくれている」といった意見を聞く事が出来た.

\figPst{90}{pepper}{東京八王子市のコメダ珈琲店で活躍するpepper君}

\figref{pepper}のように珈琲店では帽子をかぶり来店されたお客様とコミュニケーションを行っている.
また私がpepper君の見える席に座っていたところ,来店される方とpepper君の目が会うたびに人々が笑顔になったり,
驚いていたりしていたのに加えてお年寄りの方が親身にpepper君に話しかけている光景を目にしており,
人々の人工知能との対話への抵抗がなくなってきていることがわかった.
特に今後生まれてくる人々は人工知能との会話が普通のこととなっていくことが予想できる.

このようにテキストチャットの形式で対話を行う事ができるりんなの登場や人工知能を活用したロボットpepperの登場によって人工知能
に非常に注目が集まっており,このような対話システムを企業だけではなく個人でも開発したいという需要は高まっていると考えられる.

これに加えて日本にはオタク
\footnote{おたく(オタク、ヲタク)とは、1970年代に日本で誕生した呼称であり大衆文化の愛好者を指す。元来はアニメ・SF・パソコンなどの、
なかでも嗜好性の強い趣味や玩具、の愛好者の一部に使われていた術語であったが、バブル景気期に一般的に知られはじめた。\cite{ota}}
という文化がある.
ディップ株式会社の調査によると日本人口の約40%はオタクであるという調査結果となっており,非常にアニメやゲームやパソコンなどに
興味を持っている人口が多い事がわかる.

中でもアニメに関しては毎年市場が伸びており,2015年度に株式会社メディア開発綜研が調査した結果によるとその市場は2595
億円(前年比106.9\%)と過去最大の規模になったという.\cite{anime}

このように現在日本では,人工知能を搭載したロボットや女子高生人工知能との会話,アニメが注目されている事がわかった.

そしてなぜアニメがここまで注目されているのか,それはアニメキャラクターのかっこよさや可愛さといった現実にはない二次元キャラクター特有の
”人々の理想”がそこにあったからだと現状私は分析している.

\figPst{90}{hime}{理想の見た目のキャラクターとの理想の会話例\cite{hime}}

そこで\figref{hime}のように理想の見た目をしているキャラクターに,
理想の会話をしてくれる人工知能を加えたら間違いなくこのオタクの人々を感動させる事ができるという仮説を立てた.

今回はこの仮説をもとに二次元のアニメキャラクターの人工知能を気軽に開発し,二次元のキャラクターと気軽に対話を行うところまでをサポートする
フレームワークを提案する.




%------------------------------------------------------%

\chapter{提案}
%------------------------------------------------------%
%- 提案
%------------------------------------------------------%
\section{開発した人工知能の活用}
今回提案するのは先ほど説明した一般的な人工知能フレームワークを用いて開発を行った
人工知能やその返答アルゴリズムを活用するためのフレームワークです.
%------------------------------------------------------%
\subsection{知能の開発をサポートする既存フレームワーク}
既存の人工知能フレームワークは,人工知能自体を作成することをサポートしており,
その作成した人工知能を用いて会話を行います.\\
%------------------------------------------------------%
\section{開発した知能を試す環境}
今回提案するのは,フレームワークを用いて開発したアルゴリズムや,
独自のアルゴリズムを考え,作成したプログラムを実際に動かし試す環境です.
\\
既存のフレームワークは人工知能を作ることに着目して,作る工程をサポートするものが多いですが,
今回提案する人工知能利用フレームワークでは考案して,作成したアルゴリズムや人工知能を
複数登録することでUnity上で動作するキャラクターと会話を行うことができるシステムです.
\\
通常,人工知能のアルゴリズムを試したいと考えた場合,
そのプログラムに対して入力を与える入力の部分と
その処理結果を出力する出力の部分を作成する必要があり,.
作成した知能の出力結果がただ単に文字で入力して,文字で出力されれば良い場合は,
準備をするのはほぼ手間が不要であるが,キャラクターとの会話などで試したい場合,
非常に入出力の部分を作成するのに手間と時間がかかるという問題点があります.
\\
その部分をあらかじめ人工知能利用フレームワークで提供することで,
準備の手間が不要になります.
\\
また,キャラクターとの会話などで出力することで対話をする際に,
実際にそのキャラクターに言われたらどの様に感じるかをシミュレーションすることが
できるため,よりリアルなコミュニケーションを行う人工知能や人工無脳を目指して,
アルゴリズムを考え,開発することが可能になるという利点もあります.
\\
今回はその様な人工知能を利用することに着目したフレームワークを提案します.
%------------------------------------------------------%
\section{人工知能利用フレームワークの提案}
人工知能利用フレームワークは会話や動作などの返答アルゴリズムを作成した際に,
それらの作成したプログラムをフレームワーク上に適当に配置することで,
状況や話題に応じて適切な作成した返答アルゴリズムが選択され
Unity上のキャラクターと会話を楽しむことができる,
人工知能を利用することに焦点を当てたフレームワークです.
%------------------------------------------------------%
\subsection{提案する全体構成}\label{sec:allAr}
この人工知能利用フレームワークの全体の構成を次の\figref{all_kose}に示す.

\figPst{90}{all_kose}{全体の構成図}

提案する人工知能ハブは大きく分けて3つの要素で構成され,大きな流れで説明をすると
Unityでユーザーが入力した内容をもとに人工知能活用ハブがその入力内容を受け取る.
そして人工知能活用ハブの中で返答する内容が作り出され,モーションデータベースから適切な動作を選択
しUnityへ動作と返答内容を出力します.
この流れによってユーザーはキャラクターとの会話を行うことができます.
%------------------------------------------------------%
\subsection{アルゴリズムのみを簡単に追加可能な知能ハブ}
それではまずはじめに私が開発する,アルゴリズムのみを簡単に追加することができる
人工知能活用ハブを提案します.
\\
このハブでは作成した会話の返答,もしくはキャラクターの動作を選択するアルゴリズムを簡単に追加し
話題によって追加したアルゴリズムの中から適切なアルゴリズムを用いて返答を行えるようにしています.
\\
例えばゲーム関連の返答アルゴリズムを作り,試したいと考えた場合は,
そのアルゴリズムを実装したプログラムをあらかじめ準備されている抽象クラスを用いて素早く作成し,
\figref{all_kose}の返答アルゴリズム軍のプログラムに作成したプログラムを登録するだけで,
ゲームの話題が来た時にそのアルゴリズムでキャラクターが返答するシステムを作ることができる
というものです.
\\
同様にゲーム関連のキャラクターの動作を選択するアルゴリズムを作る場合は,そのアルゴリズムを
抽象クラスを用いて素早く実装し,
\figref{all_kose}の動作選択アルゴリズム軍の中に作成したプログラムを登録するだけで,ゲーム関連
の会話をしている最中は,そのアルゴリズムを用いて動作を決定する仕組みを作ることができるというものです.
\\
これらのアルゴリズムや人工知能が一つだけ実装されており,何も追加知能がない場合はその
デフォルトアルゴリズムが選択されるように設計し,複数の料理の話題に特化した話題解析アルゴリズム
やゲームの話題に特化した感情解析のアルゴリズムが実装されることでより正確な解析が可能になるだけで
はなく,返答する際もゲーム専用の返答アルゴリズムなどがあることでより円滑なコミュニケーションが
可能になるような構成を提案します.

また,様々なアルゴリズムが必要になることを考え,複数人で開発を行った際にも解析情報のデータベースによる
共有などにより,よりスムーズに連携を行うことができるほか.

人工知能ハブでは,すでに解析した感情情報などの情報は全てデータベースによって共有され,
ユーザーの入力した情報の解析を行うプログラムの開発は行わずに,すでにある感情解析プログラムの
解析結果を使ってユーザーの感情状態を考慮した「会話ボット」などの開発を行うことも可能にします.\\
%------------------------------------------------------%
\subsection{作成したアルゴリズムをUnityですぐに試せる機構}
この人工知能利用フレームワークの人工知能活用ハブに登録された知能はUnity上でのキャラクターとの
対話ですぐに試すことができます.\\

共同研究者の藤井さんによると,MMDモデルを利用しているため好きなキャラクターで動作させることが
出来,また,この人工知能利用フレームワークのために開発した,リアルタイムに動作を保管しながら
動かす技術により,よりリアルな円滑なコミュニケーションが可能になっています.\\

このように作成した人工知能をすぐにキャラクターとの対話という形で実行することができるため,
入出力をどのような設計にするかや,開発はどうするかに迷うことなく,独自の対話アルゴリズムや
人工知能の開発に専念することが可能になり,より高精度な対話を実現できると提案します.\\

%------------------------------------------------------%
\subsection{Unityが利用可能なモーションを追加する機構}
この人工知能利用フレームワークでは現在会話と動作の2つの出力を実装しています.\\

ここで返答パターンは文字列で生成され,Unityで実行されるので無限のバリエーションで返答すること
ができるが,動作(モーション)はその場で動作を生成することが難しいのが現状です.\\

そのため,あらかじめモーションデータを作る必要となるのですが,
そのモーション(動作)を定義するファイルを生成することは一般的には難しいと考えられます,
そこで共同開発の鈴木がKinectで動作を定義し,データベースに保存,人工知能活用ハブと通信可能な
プログラムを開発しました.\\

この機構があることによって,人工知能活用ハブの中で新しい動きのパターンを追加したいとなった時にも
すぐにKinectを用いて動作ファイルを生成し,データベースに登録することで使えるようなります.



%------------------------------------------------------%

\chapter{設計}
%------------------------------------------------------%
%- 構成
%------------------------------------------------------%
それでは人工知能活用ハブの構成を3段階に分けてと,Unityのキャラクターとの連携,
解析するアルゴリズムを選定する時に利用しているGoogkeAPIの3つに分けて,
人工知能利用フレームワークについて解説を行いたいと思います.
\\
\section{解析アルゴリズムの追加を可能にする機構}
まず初めにUnityのキャラクターから受け取った情報を解析する際の機構の構成について解説したいと
思います.\\
こちらの解析アルゴリズムの追加ができることで,
\subsection{解析する内容別にプログラムを保持する機能}
\subsection{会話の話題別に解析するアルゴリズムを選ぶ機能}
\subsection{解析アルゴリズムを簡単に追加する機能}

\section{解析した情報を共有する機能}
\subsection{解析情報を保存する機能}
\subsection{解析情報を取得する機能}

\section{返答アルゴリズムの追加を可能にする機構}
\subsection{返答を行うタイミング}
\subsection{返答する内容別にアルゴリズムのを保持する機能}
\subsection{会話の話題別に返答アルゴリズムを選ぶ機能}

\section{作成した知能をUnityで試す機構}
\subsection{Unityでの出力について}
\subsection{Unityとの連携に利用するWebSocket}
\subsection{Unityへの送信フォーマットと作成}
\subsection{Unityからの受信フォーマット}

\section{アルゴリズムを選定する際に用いるGoogleAPI}
\subsection{GoogleAPIについて}
\subsection{GoogleAPIの有効性}


%------------------------------------------------------%
\chapter{実装}
%------------------------------------------------------%
%- 実装
%------------------------------------------------------%

\section{開発環境}
\subsection{Javaの利用}
今回の開発では実行が高速かつ,オブジェクト指向が今回開発する人工知能フレームワークに適していると
判断したためJavaを用いて開発を行った.\\
また,当研究室に所属する学生はJavaの開発に慣れており,学習コストが低いため採用しました.\\

\subsection{Mavenフレームワーク}
Unityとの通信を行うためMavenフレームワークを用いて開発を行った.\\
なおかつ,ライブラリのバージョン管理をソースコードで行えるため,引き継ぐ際の学習コストが低いことが
利点である.\\

%------------------------------------------------------%
%インプットコントローラー
%------------------------------------------------------%
\section{解析部分の実装}
\subsection{解析コントローラー}
この解析コントローラーではすべての解析を行う分野を管理するクラスであり,
解析話題別に解析アルゴリズムを保持する解析知能ハブを作るためにInputControllerを作成しました,
それに実装した主要なメソッドなどを以下の\tabref{InputController}に示します.\\

\begin{table}[tbh]
	\caption{実装した主要メソッド} \label{tab:InputController}
	\begin{center}
		\begin{tabular}[htb]{c|c}
		\hline
		コンストラクタ & 各解析知能ハブを登録する \\
		\hline
		InputData & 入力があった時に各解析知能ハブへデータを渡す \\
		\hline
		\end{tabular}
	\end{center}
\end{table}


\tabref{InputController}のコンストラクタでは各解析分野を登録する処理を行っている.\\
その部分のソースコードを以下の\srcref{inCnt}に示す.

\srcPst{Java}{inCnt.java}{inCnt}{新しい解析分野を登録する際のソースコード}

このコンストラクタに登録された解析分野は入力があった時に呼び出され,あらかじめ抽象クラスで定義された
適切なメソッドの中のアルゴリズムを記述することで解析を行うことができる.
\\


\tabref{InputController}のInputDataメソッドではユーザーから入力があった時に入力された情報
を各解析分野の実装済みクラスに値を渡すことができます.\\

%------------------------------------------------------%
%親の抽象クラス
%------------------------------------------------------%

\subsection{解析知能ハブ}
初めに解析知能ハブの例を挙げると,感情解析知能ハブや話題解析知能ハブに当たるものであり,
各ハブはそれぞれ様々な解析アルゴリズムを所持しています.\\

解析する話題別にアルゴリズムを保持するためにAbstract Mode(以下親抽象クラス)という抽象クラスを
実装した.\\

親抽象クラスには解析する情報ごとにプログラムを保持するための機構が全て記述されており,
この親抽象クラスに実装した主要なメソッドなどを次の\tabref{Abstract Mode}に示します.

\begin{table}[tbh]
	\caption{Abstract Modeに実装した主要メソッド} \label{tab:Abstract Mode}
	\begin{center}
		\begin{tabular}[htb]{c|c}
		\hline
		init & 初期化を行うメソッド \\ \hline
		getAnalyzeParts & 解析アルゴリズムを選択するメソッド \\ \hline
		analyzeChat & 入力があった際に解析を行わせるメソッド \\ \hline
		\end{tabular}
	\end{center}
\end{table}


\tabref{Abstract Mode}のinitメソッドでは各解析アルゴリズムがもっている話題分野に登録されて
いる単語をGoogle検索にかけて,その検索結果の頻出単語を取得しています.
\\
\tabref{Abstract Mode}のgetAnalyzePartsメソッドでは実際に解析を行うアルゴリズムを生成された
頻出単語をもとに決めるメソッドです.
\\
具体的には各解析アルゴリズムごとに生成された頻出単語票のHashMapの単語表とユーザーが入力した文章を
比較して,もっとも似ている頻出単語表を持つ解析アルゴリズムが解析を行うという実装になっています.
\\



%------------------------------------------------------%
%子の抽象クラス
%------------------------------------------------------%

\subsection{解析アルゴリズムの解析知能ハブへの追加}
この人工知能利用フレームワークの解析アルゴリズムを簡単に実装する構成について説明します.\\

簡単にハブに対してアルゴリズムを追加実装するために,
実際に解析アルゴリズムを実装する際に用いる,抽象クラス「Abstract Mode Parts」
(以下,子抽象クラスと表記します)を実装しました.\\

この子抽象クラスには親抽象クラスであるAbstract Modeを実装したクラスから解析する際に呼び出されるメソッドや
データベースなどとの連携をが記述されているため,アルゴリズムを試したい場合これらの部分については追記
する必要がない点がメリットです.\\

以下の\tabref{Abstract Mode Parts}に実際に解析を行う解析知能を作るために必要な抽象クラスである
Abstract Mode Partsに実装した主要なメソッドなどを示します.
\begin{table}[tbh]
	\caption{Abstract Modeの実装} \label{tab:Abstract Mode Parts}
	\begin{center}
		\begin{tabular}[htb]{c|c}
		\hline
		クラス変数 & 保存を行うための変数が定義されている \\ \hline
		コンストラクタ & アルゴリズムの分野を記述する場所 \\ \hline
		ChatAnalyze & アルゴリズムを記述する部分 \\ \hline
		saveData & 解析結果を保存 \\ \hline
		\end{tabular}
	\end{center}
\end{table}

\tabref{Abstract Mode Parts}のクラス変数は解析した情報を保存するための変数であり,
解析結果をクラス変数に入れることで処理が終わった後に適切な形式でデータベースに保存される.
\\

\tabref{Abstract Mode Parts}のコンストラクタは,その解析アルゴリズムの分野について記述する
必要性がる,具体的に言うとaboutという変数に話題の名前を入れる必要がます,
また,そうすることで同じ話題をユーザーが話した時にそのアルゴリズムが選択されるという構造が実装されます.\\
その部分のソースコードを以下の\srcref{about}に示します.

\srcPst{Java}{about.java}{about}{コンストラクタで話題を設定するソースコード}

\srcref{about}は料理に関する話題を解析するプログラムのコンストラクタをソースコードから
抜粋したものですが,ここで1行目のプログラムを記述する.
\\

\tabref{Abstract Mode Parts}のChatAnalyzeは,実際に解析アルゴリズムを書く部分である.\\
ここで入力された内容がString型で引数として渡されてくるので,その内容を用いて解析を行う.\\
解析した情報はあらかじめ定義してあるクラス変数に保存することで,
自動でデータベースに格納されるようになっている.

\srcPst{Java}{anaAlgo.java}{anaAlgo}{解析アルゴリズムを記述するメソッドのソースコード}
\srcref{anaAlgo}は解析を行うアルゴリズムを記述するメソッドのみを抜粋したもので,
ここでデータベースに保存する情報を解析し,解析した情報を変数に格納するプログラムを記述します.
\\

\tabref{Abstract Mode Parts}のsaveDataは全ての解析が終わった後に呼び出され,変数の中に値
が入っていた場合のみその内容をデータベースにそのクラス名+データ型の名前で保存します.

最後にこの抽象クラスを拡張して作成したクラスは親クラスであるAbstract Modeを実装した,
感情解析知能ハブなどの親抽象クラスを実装したクラスのコンストラクタに登録する必要があり,
登録することでアルゴリズムの追加が完了します.\\

%------------------------------------------------------%
%現在実装しているアルゴリズム
%------------------------------------------------------%
\subsection{現在実装している解析アルゴリズム}
現在実装している解析プログラムは2種類あり,感情の解析と話題の解析である.\\

話題の解析に関しては,例えばゲームの話題解析知能を作った場合,さらに細かい何のゲームかということを
解析することを目的に作成した.\\
%------------------------------------------------------%
1つ目に感情の解析を行うアルゴリズムの実装について説明する.\\

感情の解析を行うプログラムは親抽象クラスである感情解析知能ハブに所属する解析知能の1つで,現在感情の
解析を行うプログラムはこの1つなため,必ずこのアルゴリズムがデフォルト解析アルゴリズムとして,
選ばれるようになっています.\\

具体的な解析を行うアルゴリズムに関しては「哀れ」「恥」「怒り」「嫌」「怖い」「驚き」「好き」「高ぶり」「安らか」「喜び」
の10種類の感情に分類して感情の解析をおこなっています.\\

このそれぞれの感情にはその感情に対応する単語が付いており,入力した文章の中にその単語があった時にその感情値
に1を加えて数字で表現する仕組みになっています.\\
%------------------------------------------------------%
2つ目の話題を解析する知能では料理とゲームに関する話題を解析するプログラムが実装してあり,
料理の分野では「作る」「食べる」「片付ける」の3つの話題にさらに細かく解析する仕組みがあります.
\\
また,ゲームの分野では「戦闘」「負け」「勝利」の3つの話題にさらに細かく解析する仕組みを実装しています.
\\

%------------------------------------------------------%
%データベース実装
%------------------------------------------------------%

\section{データベースの実装}
\subsection{全ての解析情報を保存する機構}
まず初めにデータベースは独自実装したデータベースクラスを用いて実現しました.\\

データベースの実装では様々な解析情報を保存する必要があるため,複数の変数型に対応するために,HashMap
のキーをStringにし,保存する値であるvalueをObject型に指定しました.\\

このデータベースクラスにはデータを保存するためのメソッドが用意されており,以下にソースコードを示します.\\

\srcPst{Java}{dbIn.java}{dbIn}{データベースの解析結果を保存するメソッド}

\srcref{dbIn}のメソッドsetDataに対して値を保存する際は解析アルゴリズムの子抽象クラスを実装する
際には記述しません.\\
子の抽象クラスの中であらかじめ定義されている変数に値を保存することで,親クラスがsaveDataメソッド
を呼び出し,saveDataメソッドにはこのデータベースクラスのsetDataに対してクラス名と値を送るように
記述してあるため,解析結果のデータを保存することがができます.\\

\subsection{解析した情報を取得する機構}
解析した情報を取得する際はこのデータベースクラスのgetDataメソッドを呼び出します.
\\
以下の\srcref{dbOut}にgetDataメソッドのソースコードを示します.
\srcPst{Java}{dbOut.java}{dbOut}{データベースの解析結果をするメソッド}

\srcref{dbOut}の2行目を見るとデータを取得する際にString型の鍵が必要になります.
\\
このString型の鍵はデータベースの中にあるデータを保存しているハッシュマップの鍵を示しており,取得したい
情報の鍵をshowDataというメソッドを用いてデータベースの中身を見ることで調べ,その値を指定して取り出します.



%------------------------------------------------------%
%OutputController
%------------------------------------------------------%

\section{出力コントローラー}
\subsection{出力情報別にアルゴリズムを保持する機構}
出力する情報をまとめ,JSON形式などを形成する出力コントローラーについて解説したいと思います.\\
まず初めに出力コントローラーに実装した主要なメソッド一覧を以下の\tabref{OutputController}に示します.\\

\begin{table}[tbh]
	\caption{実装した主要なメソッド} \label{tab:OutputController}
	\begin{center}
		\begin{tabular}[htb]{c|c}
		\hline
		コンストラクタ & 各出力知能ハブを登録する \\
		\hline
		getJson & Unityへ出力情報を送るときに呼ばれるメソッド \\
		\hline
		getTimeAction & キャラクターが自発的に発言する際に用いられるメソッド \\
		\hline
		\end{tabular}
	\end{center}
\end{table}

\tabref{OutputController}のコンストラクタの部分では,各出力知能ハブを登録します.\\
登録された出力知能ハブはgetJsonメソッドが呼ばれたときに出力内容を作成するようになっています.\\

\tabref{OutputController}のgetJsonメソッドではUnityから入力があった際に返答を行います,
そのときに呼ばれるメソッドであり,あらかじめ登録されている出力知能ハブの出力を作成するメソッドを呼び出します.\\

\tabref{OutputController}のgetTimeActionメソッドでは時間経過に応じてキャラクターが発言する
設定を有効にしているときに呼び出されるメソッドであり,このメソッドが呼ばれると各出力知能ハブの
自発的に発言する際に用いるメソッドを呼び出します.\\

%------------------------------------------------------%
%出力知能はぶ Abstract_Mode
%------------------------------------------------------%


\subsection{出力知能ハブ}
出力する動作や返答内容といった出力情報別にアルゴリズムを保持する機構について解説したいと思います.\\

この機構も情報解析の時と同じ仕組みで構成されており,この出力の情報別に保持する機構についても
Abstact Modeという親抽象クラスが作成されているので,その抽象クラスを実装することで出力知能ハブである,
返答知能ハブや動作選択知能ハブを作成することができます.
\\
以下の\tabref{abstractmode}に出力知能ハブを作成するために必要な抽象クラスである
Abstract Modeの主要メソッドを示します.\\

\begin{table}[tbh]
	\caption{実装した主要なメソッド} \label{tab:abstractmode}
	\begin{center}
		\begin{tabular}[htb]{c|c}
		\hline
		init & 初期化 \\
		\hline
		getOutput & 返答内容の作成 \\
		\hline
		getTimerAction & キャラクターが自発的に発言する際に用いられるメソッド \\
		\hline
		\end{tabular}
	\end{center}
\end{table}

\tabref{abstractmode}のinitメソッドでは初期設定を行っており,Googleの検索結果データベースを
最新の情報に更新をするなどの処理を行っています.\\

\tabref{abstractmode}のgetOutputメソッドではUnityにユーザーが話しかけてきたときに返答内容を
作成するアルゴリズムから出力内容を取得して,その内容を返すメソッドになっています.\\
また,返答するアルゴリズムを選択する機構もこの部分にあります.
\\
その選択は以下のようにHashMapを用いてユーザーが発言した内容から作成した頻出単語票であるHashMapと各解析を実際に行う
アルゴリズムが持っている話題をもとに作成した頻出単語票であるHashMapを比較して,もっとも頻出単語票が似ている
ものを選ぶという仕組みになっています.\\

\srcPst{Java}{getOutput.java}{getOutput}{getOutput.javaのソースコード}

まず初めに,\srcref{getOutput}の8,9行目で最新の発言情報を取得し,10行めで発言内容をGoogleAPIに
渡すことで頻出単語票を作成します.
\\
次にあらかじめ作成してある各解析アルゴリズムごとの話題単語の頻出単語票とGoogleAPIを用いて取得した頻出単語票を
11行めから25行めにかけて比較して,一番最適な解析アルゴリズムを選択しています.
\\
最後に45行目にて選択された解析アルゴリズムに解析を行わせ,解析結果をそのまま返しています.\\


\tabref{abstractmode}のgetTimerActionメソッドは時間経過に応じて反応するときに呼び出され,
実際に出力内容を作成するアルゴリズムに対して,自発的に発言する際の出力内容を作成させ,取得します.\\



\subsection{出力アルゴリズムの出力知能ハブへの追加}
実際に出力情報を作成するためのアルゴリズムを記述するプログラムを簡単に出力知能ハブへ追加するために
出力専用のAbstract Mode Partsという抽象クラスを作成しました.\\

その抽象クラスを用いることで3行プログラムを書くだけで新しいアルゴリズムを追加できるようになっています.\\

それではまず初めに,その抽象クラスに実装した以下の\tabref{parts}に示した主要なメソッドについて解説したいと思います.\\

\begin{table}[tbh]
	\caption{実装した主要なメソッド} \label{tab:parts}
	\begin{center}
		\begin{tabular}[htb]{c|c}
		\hline
		コンストラクタ & 担当分野の設定 \\
		\hline
		Action & 返答アルゴリズムの \\
		\hline
		TimeAction & キャラクターが自発的に発言する際に用いられるメソッド \\
		\hline
		dataRefresh & 常にデータベースを最新に保つためのメソッド \\
		\hline
		\end{tabular}
	\end{center}
\end{table}

\tabref{parts}のコンストラクタでは出力を行う際に担当する分野や話題について記述する部分です,解析の
時と同じように変数aboutに対して適切な担当する話題名を入れることで,GoogleAPIを用いて
その話題名に関する頻出単語票が自動で生成されます,その生成された頻出単語表は先ほど説明したアルゴリズムの
選定に利用されます.\\

以下の\srcref{aisatu}に挨拶の分野を指定する場合のプログラムの例を示します. \\

\srcPst{Java}{aisatu.java}{aisatu}{話題を指定する際のサンプルソースコード}

\srcref{aisatu}では話題を挨拶に指定しており,ユーザーが挨拶と関係のある単語を発話した時にこのアルゴリズムが
選択され,実際に返答内容を作成します.\\

\subsection{現在実装している出力アルゴリズム}


\section{Unityとの通信の実装}
\subsection{Unityからの入力情報の受信}
\subsection{Unityへの命令の送信}

\section{追加したモーションの利用}
\subsection{動作選択アルゴリズムの実装}

\section{GoogleAPIと形態素解析を用いた頻出単語表の作成する機構}
\subsection{形態素解析による検索ワードの作成}
\subsection{GoogleAPIを利用して検索結果を取得}
\subsection{検索結果のフィルタリング}
\subsection{頻出単語表の作成}




%------------------------------------------------------%
\chapter{実行結果}
%------------------------------------------------------%
%- 実行結果
%------------------------------------------------------%

\section{Unityの出力画面の図}
実際にキャラクターとの対話を行う際の画面は以下の様な画面で対話を行うことが可能である.

\figPst{100}{doUnity}{キャラクターとの対話画面}

\figref{doUnity}の画面にて実際にキャラクターと対話を行う形式で作成した人工知能のアリゴリズムが
正しく動作しているのを確認することが可能である.

%------------------------------------------------------%

\section{実際の会話}
実際に人工知能利用フレームワークを用いて会話を行う例を以下の\tabref{Chat}に示す.

\begin{table}[tbh]
	\caption{キャラクターとの対話例} \label{tab:Chat}
	\begin{center}
		\begin{tabular}[htb]{c|c|c}
		\hline
		ユーザーの発言 & キャラクターからの返答 & 動作 \\
		\hline
		カルボナーラ作ってるんだ & 隠し味にチョコレートいれちゃう? & 悪巧みの動作 \\
		うわお!クッパ強い,負けたよ & ゲームオーバーだねー & がっかり \\
		\hline
		\end{tabular}
	\end{center}
\end{table}

以上の\tabref{Chat}の様にユーザーが話しかけた内容に対して,返答を行う仕組みが実装されていることがわかる.

またユーザーが話しかけた内容から,現在の会話の話題を解析することでその話題に対応している解析や出力を行うアルゴリズムが処理を行う仕組みにより,
詳しい解析や出力を行うことが可能となっていることが\tabref{Chat}を見ることでわかる.


\section{アルゴリズムを追加した後の会話}\label{sec:addAl}
実際にアルゴリズムを追加した後,どの様な変化があるかを検証する.

今回追加を行うアルゴリズムの話題はゲームの「スーパーマリオブラザーズ\footnote{『スーパーマリオブラザーズ』(Super Mario Bros.)は,任天堂が発売したファミリーコンピュータ用ゲームソフト.}」(以下,マリオ)
という話題をもつ解析と出力を行うアルゴリズムである.
話題を解析するアルゴリズムは,
マリオに特化したさらに細かい話題の解析を行うアルゴリズムを実装した.

ユーザーと会話を行う際に用いる,返答アルゴリズムに関しても「マリオ」の話題に対応を行い,
マリオの話題に対して特化しているアルゴリズムを追加した.
このマリオのアルゴリズムを追加した後の会話を以下の\tabref{afterChat}に示す.

\begin{table}[tbh]
	\caption{アルゴリズム追加後の会話} \label{tab:afterChat}
	\begin{center}
		\begin{tabular}[htb]{c|c|c}
		\hline
		ユーザーの発言 & キャラクターからの返答 & 動作 \\
		\hline
		うわお!クッパ強い,負けたよ & きのこを取っておこう! & がっかり \\
		\hline
		\end{tabular}
	\end{center}
\end{table}

この「マリオ」という話題は,ゲームという大枠の中の単一ゲームタイトル名である.
そのためユーザーが話しかけた内容が,単に幅広くゲームに関連のある単語であれば,ゲームの話題が解析を行う.
しかしここで「マリオ」というゲームに登場する敵キャラクター「クッパ\footnote{クッパとは、
マリオシリーズに登場するキャラクターであり,敵キャラクター正式名称は「大魔王クッパ」}」
という単語を含めることにより,\tabref{afterChat}の様に
マリオの解析と出力に特化したアルゴリズムが選択される.

返答内容に関しても「負けたよ」という発言から,このゲームのプレイヤーを強化することができるアイテム
である「きのこ」を取得して再度,敵に対して挑戦しようという返答を行うことができていることがわかる.

この様に料理やゲームなどの会話を行う際に対応する種類を横に広げていくアルゴリズムの追加方法と,
料理やゲームのさらに細かい話題に対して解析を行うことができるようにするアルゴリズムの追加方法の両方に対応している.


%------------------------------------------------------%
\chapter{結論}
%------------------------------------------------------%
%- 結論
%------------------------------------------------------%

\section{結論}
今回作成した人工知能利用フレームワークを使い,考案したアルゴリズムを作成し
キャラクターと会話することで試すことはできたと考えている.
しかし,話題が固定された状況で返答アルゴリズムを書く必要があり,
現在どの話題で,どのアルゴリズムが回答するかを選択する部分のプログラムを書きたい場合は
その部分のプログラムを直接書き換えなくてはならない欠点がある.

\subsection{アルゴリズムの追加による出力の変化}
\ref{sec:addAl}の実行結果を見て分かる通り,アルゴリズムを追加することでその分野の話題に
なった時に追加したアルゴリズムが応答していることが分かる.
また,解析を行うアルゴリズムを追加したことによって,その分野のさらに詳しい話題の分析が\ref{sec:addAl}
の返答を見て分かる通り,可能となった.

\subsection{簡単にアルゴリズムを追加できたか}
考案したアルゴリズムを素早く試すために,アルゴリズムの追加する際の構造を簡単化したため,
特定の話題の時に解析や出力情報のアルゴリズムの作成と追加を非常に簡単に実現できる機構は
達成できた.

会話を行うことができるアルゴリズムを,この人工知能利用フレームワークでは3行のプログラムソースコード
の記述で実現できるため,非常に開発を開始するまでの時間を短くすることが出来,
また,Unityによる出力先のサポートにより,この会話内容を本物のキャラクターに言われたらどの様に
感じるかもシミュレーションを行うことができるため,文字だけでの出力しか行えない場合よりも
よりリアルなコミュニケーションを行う人工知能や人工無脳の作成を目指すことができることがわかった.



%------------------------------------------------------%

\chapter*{謝辞}
\addcontentsline{toc}{chapter}{謝辞}
%------------------------------------------------------%
%- 謝辞
%------------------------------------------------------%

本論文を作成するにあたって,監督いただいた教授および参考文献を作成した方々に感謝いたします.


%------------------------------------------------------%
%- References
%------------------------------------------------------%

\begin{thebibliography}{99}
	\bibURL{muno}{wikipedia}{人工無脳 Wikipedia}
	{https://ja.wikipedia.org/wiki/人工無脳}{2015年12月27日}

	\bibURL{tino}{wikipedia}{人工知能 Wikipedia}
	{https://ja.wikipedia.org/wiki/人工知能}{2015年12月27日}

	\bibURL{deep}{wikipedia}{DeepLearning}
	{https://ja.wikipedia.org/wiki/ディープラーニング}{2015年12月27日}

	\bibURL{boom}{東京工科大学}{東京工科大学公式ホームページ教員紹介}
	{http://www.teu.ac.jp/info/lab/teacher/cs/index.html?id=1635}{2015年12月27日}


	\bib{latex}{奥村晴彦 著}{\LaTeXe 美文書作成入門 改訂第3版}{技術評論社 2004, 403pp}

	\bib{latex}{マルチェッロ・マッスィミーニジュリオ・トノーニ 著}{\LaTeXe 意識はいつ生まれるのか}{亜紀書房}

	\bibURL{google}{mwSoft blog}{HttpClientとHttpCleanerでGoogle検索結果を解析する例}{http://blog.mwsoft.jp/article/34841195.html}{2015年8月12日}

	\bibURL{kuromoji}{mwSoft blog}{Java製形態素解析器「Kuromoji」を試してみる}{http://www.mwsoft.jp/programming/lucene/kuromoji.html}{2015年8月12日}

	\bibURL{gitkuromoji}{github}{形態素解析器kuromoji}
	{https://github.com/atilika/kuromoji}{2015年8月12日}

\end{thebibliography}

\end{document}
