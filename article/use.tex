%------------------------------------------------------%
%- 提案
%------------------------------------------------------%
\section{開発した人工知能の活用}
今回提案するのは先ほど説明した一般的な人工知能フレームワークを用いて開発を行った
人工知能を活用するためのフレームワークである.\\
%------------------------------------------------------%
\subsection{知能の開発をサポートする既存フレームワーク}
既存の人工知能フレームワークは,人工知能自体を作成することをサポートしている.
具体例で言うとChainerはPreferred Networksが開発したニューラルネットワーク
を実装するためのライブラリであり,実際の開発をすることの手助けをしている.
%------------------------------------------------------%
\section{開発した知能を試す環境}
今回提案するのは,フレームワークを用いて開発したアルゴリズムや,
独自のアルゴリズムを考え,作成したプログラムを実際に動かし試す環境である.\\

通常,人工知能のアルゴリズムを試したいと考えた場合,そのプログラムに対して入力を与える入力の部分と
その処理結果を出力する出力の部分を作成する必要がある.\\

作成した知能の出力結果がただ単に文字で入力して,文字で出力されれば良い場合は,
準備をするのはほぼ手間が不要であるが,キャラクターとの会話などで試したい場合,
非常に入出力の部分を作成するのに時間がかかってしまう.\\

そこで,今回は作成したアルゴリズムをすぐに試す環境を提供するフレームワークを提案する.
%------------------------------------------------------%
\section{人工知能利用フレームワークの提案}
人工知能利用フレームワークは会話や動作などの返答アルゴリズムを作成した際に,
それらの作成したプログラムをフレームワーク上に適当に配置することで,
状況や話題に応じて適切な作成した返答アルゴリズムが選択され
Unity上のキャラクターと会話を楽しむことができる,
人工知能を利用することに焦点を当てたフレームワークである.
%------------------------------------------------------%
\subsection{全体構成}
この人工知能利用フレームワークの全体構成を


\subsection{アルゴリズムのみを簡単に追加可能な知能ハブ}
\subsection{作成したアルゴリズムをUnityですぐに試せる機構}
\subsection{Unityが利用可能なモーションを追加する機構}