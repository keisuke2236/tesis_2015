%------------------------------------------------------%
%- 実行結果
%------------------------------------------------------%

\section{Unityの出力画面の図}
実際にキャラクターとの対話を行う際の画面は以下の様な画面で対話を行うことが可能である.

\figPst{100}{doUnity}{キャラクターとの対話画面}

\figref{doUnity}の画面にて実際にキャラクターと対話を行う形式で作成した人工知能のアリゴリズムが
正しく動作しているのを確認することが可能である.

%------------------------------------------------------%

\section{実際の会話}
実際に人工知能利用フレームワークを用いて会話を行う例を以下の\tabref{Chat}に示す.

\begin{table}[tbh]
	\caption{キャラクターとの対話例} \label{tab:Chat}
	\begin{center}
		\begin{tabular}[htb]{c|c|c}
		\hline
		ユーザーの発言 & キャラクターからの返答 & 動作 \\
		\hline
		カルボナーラ作ってるんだ & 隠し味にチョコレートいれちゃう? & 悪巧みの動作 \\
		うわお!クッパ強い,負けたよ & ゲームオーバーだねー & がっかり \\
		\hline
		\end{tabular}
	\end{center}
\end{table}

以上の\tabref{Chat}の様にユーザーが話しかけた内容に対して,返答を行う仕組みが実装されていることがわかる.

またユーザーが話しかけた内容から,現在の会話の話題を解析することでその話題に対応している解析や出力を行うアルゴリズムが処理を行う仕組みにより,
詳しい解析や出力を行うことが可能となっていることが\tabref{Chat}を見ることでわかる.


\section{アルゴリズムを追加した後の会話}\label{sec:addAl}
実際にアルゴリズムを追加した後,どの様な変化があるかを検証する.

今回追加を行うアルゴリズムの話題はゲームの「スーパーマリオブラザーズ\footnote{『スーパーマリオブラザーズ』(Super Mario Bros.)は,任天堂が発売したファミリーコンピュータ用ゲームソフト.}」(以下,マリオ)
という話題をもつ解析と出力を行うアルゴリズムである.
話題を解析するアルゴリズムは,
マリオに特化したさらに細かい話題の解析を行うアルゴリズムを実装した.

ユーザーと会話を行う際に用いる,返答アルゴリズムに関しても「マリオ」の話題に対応を行い,
マリオの話題に対して特化しているアルゴリズムを追加した.
このマリオのアルゴリズムを追加した後の会話を以下の\tabref{afterChat}に示す.

\begin{table}[tbh]
	\caption{アルゴリズム追加後の会話} \label{tab:afterChat}
	\begin{center}
		\begin{tabular}[htb]{c|c|c}
		\hline
		ユーザーの発言 & キャラクターからの返答 & 動作 \\
		\hline
		うわお!クッパ強い,負けたよ & きのこを取っておこう! & がっかり \\
		\hline
		\end{tabular}
	\end{center}
\end{table}

この「マリオ」という話題は,ゲームという大枠の中の単一ゲームタイトル名である.
そのためユーザーが話しかけた内容が,単に幅広くゲームに関連のある単語であれば,ゲームの話題が解析を行う.
しかしここで「マリオ」というゲームに登場する敵キャラクター「クッパ\footnote{クッパとは、
マリオシリーズに登場するキャラクターであり,敵キャラクター正式名称は「大魔王クッパ」}」
という単語を含めることにより,\tabref{afterChat}の様に
マリオの解析と出力に特化したアルゴリズムが選択される.

返答内容に関しても「負けたよ」という発言から,このゲームのプレイヤーを強化することができるアイテム
である「きのこ」を取得して再度,敵に対して挑戦しようという返答を行うことができていることがわかる.

この様に料理やゲームなどの会話を行う際に対応する種類を横に広げていくアルゴリズムの追加方法と,
料理やゲームのさらに細かい話題に対して解析を行うことができるようにするアルゴリズムの追加方法の両方に対応している.

