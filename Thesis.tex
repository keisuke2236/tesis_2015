\documentclass[a4paper,10pt,onecolumn,oneside,openany]{jsbook}

% Import original package
\usepackage{./conf/cs}
\usepackage{./conf/style}
%======================================================%
%----- T-lab LaTeX Configuration template
%======================================================%

%------------------------------------------------------%
%- Import package
%------------------------------------------------------%

\usepackage{url}						% URL
\usepackage{ifthen}						% 条件分岐


%------------------------------------------------------%
%- if
%------------------------------------------------------%

\newif\ifTaA					% 表紙と概要
\newif\ifToC					% 目次
\newif\ifPDF 					% 画像出力
\newif\ifSRC					% ソースコード


%------------------------------------------------------%
%- Conditional branch
%--- コンパイルしたい条件によって変えて
%------------------------------------------------------%

\TaAtrue					% 表紙と概要を出力するか
%\TaAfalse					% ┗ しない場合
\ToCtrue					% 目次を出力するか
%\ToCfalse					% ┗ しない場合
\PDFtrue					% 画像出力をPDFにするか
%\PDFfalse					% ┗ しない場合

%- ソースコードを使うか (※ README を読む事)
%\SRCtrue					% 使う場合は true に
\SRCfalse					% ┗ 使わない場合


%------------------------------------------------------%
%- Image output setting
%------------------------------------------------------%

\ifPDF
	% PDF image output
	\usepackage[dvipdfmx]{graphicx}
	\usepackage[dvipdfmx]{color}
	\usepackage[dvipdfmx]{colortbl}
\else
	% dviout image output
	\usepackage[dviout]{graphicx}
	\usepackage[dviout]{color}
	\usepackage[dviout]{colortbl}
\fi


%------------------------------------------------------%
%- listings
%------------------------------------------------------%

\ifSRC

	\usepackage{listings}
	\usepackage{jlisting}

	\lstset{
		classoffset=0,
		numbers={left},
		stepnumber={1},
		sensitive={true},
		frame={tRBl},
		framesep={5pt},
		frameround={fttt},
		rulesep = 2pt,
		showstringspaces={false},
		tabsize={2},
		breaklines=true,
		xleftmargin=5mm,
		xrightmargin=3mm,
		%framexleftmargin=6mm,							% 行番号をフレームに入れる
		basicstyle={\ttfamily \footnotesize},
		numberstyle={\scriptsize},
		stringstyle={\ttfamily \color[cmyk]{0,0.8,0,0}},
		commentstyle={\color[rgb]{0,0.5,0}},
		keywordstyle={\ttfamily \color[rgb]{0,0,1}}
	}

	%- Java
	\newenvironment{Java}[0]{
	\lstset{
		language={Java},
		classoffset=1,
		keywordstyle={\ttfamily \color[rgb]{1, 0, 0}},
		morekeywords={
			Louise,
		}
	}}{}

	%- Code paste
	\newcommand{\srcPst}[4]{
		\vspace{3mm}
			\begin{#1} \lstinputlisting[caption=#4 ( #2 ), label=src:#3]{./src/#2} \end{#1}
		\vspace{5mm}
	}

	\newcommand{\srcref}[1]{{\bf \lstlistingname~\ref{src:#1}}}		% 参照
	\renewcommand{\lstlistingname}{{\bf リスト}}					% キャプション
	\renewcommand{\thelstnumber}{\arabic{lstnumber}:}				% 行番号の表示

\fi


%------------------------------------------------------%
%- Renewcommand
%------------------------------------------------------%


%------------------------------------------------------%
%- Define original command
%------------------------------------------------------%

%- Reference
\newcommand{\figref}[1]{{\bf \figurename~\ref{fig:#1}}}
\newcommand{\tabref}[1]{{\bf \tablename~\ref{tab:#1}}}
\newcommand{\equref}[1]{{\bf 式~(\ref{equ:#1})}}

%- Quotation
\newcommand{\dq}[1]{`` #1 ''}
\newcommand{\bdq}[1]{{\bf \dq{#1}}}

%- Bibliography
\newcommand{\bib}[4]{\bibitem{#1} #2 : ``#3''{ }(#4).}
\newcommand{\bibURL}[5]{\bibitem{#1} #2 : ``#3''{ }\url{#4}{ }(#5).}

%- Image paste
\newcommand{\figPst}[3]{
	\vspace{3mm} \begin{figure}[tbh]
		\begin{center}
			\includegraphics[width=#1mm]{./figure/#2.png}
			\caption{#3}\label{fig:#2}
		\end{center}
	\end{figure} \vspace{-2mm}
}


% Define basic information
% - ここら辺を空気読んで編集して
\author{寺田 佳輔}										% 著者
\id{C0112336}											% 学籍番号
\juryoshoid C0112336									% 学籍番号 ({}を付けないで)
\title{人工知能利用フレームワークの開発}				% タイトル短い人はこっち (\longTitle[AB] は空白に)
\longTitleA{}											% タイトル長い人はこっちに A,B に分けて書く
\longTitleB{}											% (\title は空白に)
\juryoshotitle{人工知能利用フレームワークの開発}		% タイトル (こっちは長い場合は \\ で改行を入れる)
\courseofcs												% ここからはみんな同じ
\clab{田胡}												% 研究室名
\teacher{田胡 和哉}										% 指導教員
\date{2016年01月18日}									% 提出日
\cnendo{2015}											% 提出年度
\nendo{2015年度}										% 提出年度


%------------------------------------------------------%
%- Document
%------------------------------------------------------%

\begin{document}
%------------------------------------------------------%
%- 目次とか色々
%------------------------------------------------------%

% 表紙など
\ifTaA
	\makejuryosho
	\makecover
	\maketitle
	\jabst{
		%------------------------------------------------------%
%- 概要
%--- Abstractは文字数超過するとはみ出るので,
%--- 良い感じに調節して下さい.
%
%--- 生成された枠の一番したの行でおよそ800字ちょいになるように
%--- 文字サイズ・レイアウトを調整しています.
%------------------------------------------------------%

ルいず

	}
	\makejabstract
\fi

% 目次 ・ 図目次 ・ 表目次
\ifToC
	\pagenumbering{roman}
	\setcounter{tocdepth}{3}
	\tableofcontents
	\listoffigures
	\listoftables
	\clearpage
	\pagenumbering{arabic}
\fi


%------------------------------------------------------%

\chapter{現状}
%------------------------------------------------------%
%- 現状
%------------------------------------------------------%
近年人工知能などの分野が注目を浴びており,ここではそれらの現状について説明する.

\section{人工知能}
人工知能とは人工的にコンピュータ上などで人間と同様の知能を実現させようという試み,
或いはそのための一連の基礎技術を指すものであり,1956年にダートマス会議でジョン・マッカーシーにより命名された.
人工知能の定義は未だ不確定な部分が多く,完全に正確な定義は存在していないのが現状である.
この人工知能は人工無脳とは異なりただ単にキーワードを広うだけではなく,
機械学習によって取得した情報を用いて解析や情報の取得を行い,状況に応じた返答などが可能なプログラムのことを指すことが多い.
機械学習とは入力されたデータから反復的に特徴を学習することでそこに潜むパターンを見つけ出す技術のことであり,
人工知能には必要不可欠な要素である.
また学習を行わず特定のキーワードを拾い返答するものを,人工無脳または対話ボットという.

\section{DeepLearning}
DeepLearning\cite{deep}とは多層構造のニューラルネットワークの機械学習の事であり,
ニューラルネットワークを多層積み重ねたモデルを機械学習させればディープラーニングとなる.
またNeural Networkとは脳細胞を構成する「Neuron(ニューロン)」の活動を単純化したモデルであり,これを利用する
ことによって人間の思考をシミュレーションすることができるものである.
一般的にはこのニューラルネットワークが3層以上のあるものをディープラーニングと呼んでいる.

第1層から入った情報は,より深い3層へと学習を行うことにより,
概念を認識する特徴量と呼ばれる重要な変数を自動で発見することができる.

このDeepLearningにより,東京大学の松尾豊准教授\cite{boom}を始めとする機械学習や人工知能の研究者は
「AI研究に関する大きなブレイクスルーであり、学習方法に関する技術的な革新である」と指摘している.

\section{一般的な人工知能開発フレームワーク}\label{sec:ippan}
現在一般的な人工知能を開発するフレームワークとしてchainerやGoogleのTensorFlowなどがある.

Chainerは,Preferred Networksが開発したニューラルネットワークを実装するためのライブラリであり,
人工知能自体の開発を行う際に高速な計算が可能なことや,様々なタイプのニューラルネットを実装可能であり,
またネットワーク構造を直感的に記述できる利点がある.

GoogleのBrain Teamの研究者たちが作った機械学習ライブラリであるTensor Flowは,
Python APIとC++インターフェイス一式が用意されているため開発を行う際に非常に有効だと考えられる.

これらの人工知能開発フレームワークは実際に開発を行うときに非常に有効であり,googleの検索アルゴリズムや
データ分析などの様々な分野での応用を試みる動きがある.

\section{人工知能の様々な分野への活用}
現在人工知能はデータ分析からGoogleの検索アルゴリズムなど様々な分野への応用が試みられている.
以下の\figref{brainmap}にO2O INNOVATION LAB\cite{lab}を参考に人工知能の活用例を複数挙げたものを示す.

\figPst{150}{brainmap}{人工知能の様々な活用例}

\figref{brainmap}のように人工知能はすでに様々な分野で活用されている.
今回は大きく2つに分けて人々の生活をより良いものにしていくカジュアルな分野での活用と,ビジネスに応用することで
より効果的なサービスを提供することへの活用の2種類に分けて図示した.

初めに,ビジネス分野での人工知能の活用例としてGoogleの検索アルゴリズムに活用されている例と,
ネット上の広告枠にどの広告を表示するか決めているアルゴリズム,及び自動車の自動走行に活用している例を紹介する.

Googleの検索アルゴリズムに活用している例では機械学習を用いて検索結果の最適化を行っており,
ウェブサイトの内容を学習することによってより優良なウェブサイトやユーザーの役に立つウェブサイトが検索結果の上位に表示されるように設計されている.
また質の低いコンテンツや有害コンテンツの排除も行っている.

ネット上の広告枠にどの広告を表示するか決めているアルゴリズムRTB(RealTimeBitting)について紹介する.
RTBは人工知能を用いた広告の競売システムであり,
アクセスして来たユーザーがどのような分野に興味があるかと言ったことや年齢及び性別など瞬時に算出し,
複数の広告主に1つの広告枠をいくらで入札するかを瞬時に問い合わせて1番高値で入札した広告をユーザーが閲覧するウェブサイトに表示するシステムである.

自動車の自動走行はアメリカ合衆国カリフォルニア州にある半導体メーカーのNvidia Corporationをはじめとする企業が研究を行っており,
自動運転の開発プラットフォームも発表している.
ディープラーニング技術と画像認識機能と組み合わせる事で,救急車と配送トラックといった車種の違いや,駐車中の自動車が発進しようとしているか
どうかを見分けるなど,まるで人間が目で見て判断をしているようなことが可能であり,
そのような微妙な違いに対応することで自律走行が実現できるというものである.

次に人々の生活の中で密に関わることができるカジュアルな人工知能の活用例について紹介する.
カジュアルな人工知能の活用方法の一例として\figref{brainmap}にはゲームAIへの応用と夕飯の提案及び人間との対話を挙げた.

ゲームAIへの応用では将棋を行う人工知能が名人棋士を敗る段階にまで達しており,「将棋電王戦」と称して人工知能とプロ棋士の戦いが
毎年繰り広げられているが,2014年はプロ棋士5人でわずか1勝しかできないなど将棋に関して言えば人工知能はすでに人間を超越しているといえる.

また無限の食材の組み合わせを提案するIBMが開発を行った人工知能「シェフ・ワトソン」では夕飯のメニューの提案なども行われており,
人間が現在作れている料理の種類は「シェフ・ワトソン」が提案可能なすべての料理のわずか0.0000001\%という試算もあるという.
このように人間には思いつかないような料理を提案できる人工知能も登場している.

最後に人間と対話(コミュニケーション)を行うことのできる人工知能について紹介する.
人工知能には様々な用途があることを紹介して来たが,そのうちの一つとして人工知能を用いた会話システムという活用方法もある.
コンピュータと人間とのコミュニケーションに人工知能を用いることで,より円滑なコミュニケーションを行うことができると考えられている.

以上のように様々な分野に人工知能を活用する流れがある中で,本研究ではその活用方法のうちの1つである人間とのコミュニケーションへの
人工知能の活用を行っていくこととする.

\section{知能の開発をサポートする既存フレームワークの現状}
\ref{sec:ippan}章で説明したようなフレームワークが登場する前までは,人工知能の開発は非常に手間と時間のかかるものであった.
それに加えてchainerはGoogleが開発を行った非常に高度な技術を用いるフレームワークであり,
個人でこのような機構を持った人工知能の開発を行いたいと考えたとしても事実上不可能であるのが現状である.

しかし様々なフレームワークの登場により人工知能自体を開発する事が非常に簡単になった事に加え,
より高度な技術を用いた人工知能の開発を行う事が可能になった.
このように個人でも高度な人工知能の開発が可能になった事により,現在非常に人工知能の分野が注目を浴びている.

このように高度な人工知能の開発が個人でも行えるようになったため,
本研究ではこの人工知能の技術を機械と人間とのコミュニケーションに応用したいと考えた.

人間とのコミュニケーションを行う際に最も適したインターフェイスとはどのようなものだろうかということを考えたときに,
まず最初に思いつくインターフェイスは人型のキャラクターとの対話を行えるインターフェイスであると考えた.

しかし既存のフレームワークには,開発した人工知能を用いてキャラクターと会話を行うというところまでをサポートするフレームワークがない.
キャラクターとの対話で人工知能を試したい場合は対話を行うアルゴリズムを作成するのに加えて,
キャラクターとの対話を行うインターフェイスの準備やキャラクターを動作させるための動作ファイルなど様々な準備が必要となるのが現状である.

このように既存のフレームワークを用いて簡単に高性能なアルゴリズムを作ったとしても,それを気軽に楽しむ為の環境がないのが現状である.

\newpage

\section{人工知能のアプリケーションへの応用}
現在Microsoftのりんな\cite{rinna}やソフトバンクのpepper\cite{pepper}などの登場により,人工知能を応用した対話に注目が集まっている.

Micsoroftのりんなは,Microsoftが開発を行った人工知能であり,話しかけると女子高生のような返答を返してくれるものである.
現在LINE\footnote{LINEとは韓国最大のIT企業「NHN」の日本法人「LINE株式会社」が提供しているスマートフォン(iPhoneやAndroid)、ガラケー(フィーチャーフォン)、パソコンに対応したコミュニケーションアプリケーションです.\cite{line}}
アプリ上で公式アカウントを持っており,りんなを友達登録しているユーザー数は2016年1月13日現在では2,167,730人である.
これは日本の人口が現在1億2688万人\cite{humen}なので総人口の1.7\%にあたり,およそ100人に1人または2人がこの人工知能りんなとの対話を
行っているというのが現状である.

ソフトバンクのpepperに関しては2016年現在,テレビCMなども頻繁に行われており,
ソフトバンクショップを初めとして一般家庭や喫茶店などで幅広く活用されている.

またTIS株式会社と国立大学法人奈良先端科学技術大学院大学も昨年の2015年11月から共同で「マルチモーダルインタラクションを用いた
パブリックスペースにおける対話処理」について研究を開始している.\cite{tis}
この研究は音声や画像や言語などの複数の情報源を用いて,ロボットと人とのスムーズなコミュニケーションの実現を目指す研究であり,
人間とコンピュータのコミュニケーションを実現する本格的なプロジェクトの一つであると言える.
以下の\figref{tis}にその研究のイメージとしてウェブサイト上に記載されている図を示す.

\figPst{120}{tis}{研究イメージ}

\figref{tis}を見るとロボットと人とのスムーズなコミュニケーションの実現を目指していることがわかり,
すでに開発されているpepperの他にも人工知能をパブリックスペースで活用しようという動きがあることがわかる.


八王子市の喫茶店コメダ珈琲店\cite{coffe}のスタッフに「pepper君を導入してから何か変わりましたか?」とインタビューを行ったところ,
スタッフからは「pepperを導入してからお子様が来店された時に喜び,pepper君と話している姿をみます」といった声や
「店内が混雑している時の待ち時間にお客様がお話ししており,待ち時間の退屈さを紛らわせてくれている」といった意見を聞く事が出来た.

\figPst{100}{pepper}{東京都八王子市のコメダ珈琲店で活躍するpepper君}

\figref{pepper}のように珈琲店では帽子をかぶり来店されたお客様とコミュニケーションを行っている.
また私がpepper君の見える席に座っていたところ,来店される方とpepper君の目が会うたびに人々が笑顔になったり,
驚いていたりしていた.
これに加えてお年寄りの方をはじめとする様々な方が親身にpepper君に話しかけている光景を目にしており,
人工知能との対話に対する抵抗は人工知能への理解が進んだためかあまりないように見えた.
特に今後産まれてくる人々は人工知能との会話が日常のこととなっていくことが予想できる.

このようにテキストチャットの形式で対話を行う事ができるりんなの登場や人工知能を活用したロボットpepperの登場によって人工知能
に非常に注目が集まっており,このような対話システムを企業だけではなく個人でも開発したいという需要は高まっていると考えられる.

これに加えて日本にはオタク
\footnote{おたく(オタク、ヲタク)とは、1970年代に日本で誕生した呼称であり大衆文化の愛好者を指す。元来はアニメ・SF・パソコンなどの、
なかでも嗜好性の強い趣味や玩具、の愛好者の一部に使われていた術語であったが、バブル景気期に一般的に知られはじめた。\cite{ota}}
という文化がある.
ディップ株式会社の調査によると日本人口の約40%はオタクであるという調査結果となっており,非常にアニメやゲームやパソコンなどに
興味を持っている人口が多い事がわかる.

中でもアニメに関しては毎年市場が伸びており,2015年度に株式会社メディア開発綜研が調査した結果によるとその市場は2595
億円(前年比106.9\%)と過去最大の規模になったという.\cite{anime}

このように現在日本では,人工知能を搭載したロボットや女子高生人工知能との会話,アニメが注目されている事がわかった.

そしてなぜアニメがここまで注目されているのかについて考察すると,
それはアニメキャラクターの”かっこよさ”や”可愛さ”といった現実にはない二次元キャラクター特有の
”人々の理想”がそこにあるからであると私は分析している.

\figPst{100}{nowUnity}{理想の見た目のキャラクターとの理想の会話例}

そこで\figref{nowUnity}のように理想の見た目をしているキャラクターに,
理想の会話をしてくれる人工知能を加えたら間違いなくこのオタクの人々を感動させる事ができるという仮説を立てた.

今回はこの仮説をもとに二次元のアニメキャラクターの人工知能を素早く開発し,二次元のキャラクターと気軽に対話を行う環境までをサポートする
フレームワークを提案する.




%------------------------------------------------------%

\chapter{提案}
%------------------------------------------------------%
%- 提案
%------------------------------------------------------%
\section{開発した人工知能の活用}
今回提案するのは先ほど説明した一般的な人工知能フレームワークを用いて開発を行った
人工知能やその返答アルゴリズムを活用するためのフレームワークである.
%------------------------------------------------------%
\subsection{知能の開発をサポートする既存フレームワーク}
既存の人工知能フレームワークは,人工知能自体を作成することをサポートしており,
その作成した人工知能を用いて会話を行う.\\
%------------------------------------------------------%
\section{開発した知能を試す環境}
今回提案するのは,フレームワークを用いて開発したアルゴリズムや,
独自のアルゴリズムを考え,作成したプログラムを実際に動かし試す環境である.
\\
既存のフレームワークは人工知能を作ることに着目して,作る工程をサポートするものが多い,
今回提案する人工知能利用フレームワークでは考案し,作成したアルゴリズムや人工知能を
複数登録することでUnity上で動作するキャラクターと会話を行うことができるシステムである.
\\
通常,人工知能のアルゴリズムを試したいと考えた場合,
そのプログラムに対して入力を与える入力の部分と
その処理結果を出力する出力の部分を作成する必要がある.
\\
作成した知能の出力結果がただ単に文字で入力して,文字で出力されれば良い場合は,
出力画面を準備をするのは手間が不要であるが,キャラクターとの会話などで試したい場合,
非常に入出力の部分を作成するのに手間と時間がかかるという問題点がある.
\\
その部分をあらかじめ人工知能利用フレームワークで提供することで,
準備の手間が不要になるという利点がある.
\\
また,キャラクターとの会話などで出力することで対話をすることにより,
実際にそのキャラクターに言われたらどの様に感じるかをシミュレーションすることが
できるため,よりリアルなコミュニケーションを行う人工知能や人工無脳を目指して,
アルゴリズムを考え,開発することが可能になるという利点がある.
\\
今回はその様な人工知能を利用することに着目したフレームワークを提案する.
%------------------------------------------------------%
\section{人工知能利用フレームワークの提案}
人工知能利用フレームワークは会話や動作などの返答アルゴリズムを作成した際に,
それらの作成したアルゴリズムをフレームワーク上に適当に記述することで,
状況や話題に応じて適切な作成した返答アルゴリズムが選択され
Unity上のキャラクターと会話を楽しむことができる,
人工知能を利用することに焦点を当てたフレームワークである.
%------------------------------------------------------%
\subsection{提案する全体構成}\label{sec:allAr}
この人工知能利用フレームワークの全体の構成を次の\figref{all_kose}に示す.

\figPst{90}{all_kose}{全体の構成図}

提案する人工知能利用フレームワークにはUnityで作成したキャラクターを出力する部分と
キャラクターに行わせる動作を考える人工知能ハブ,また,Unity上で利用するためのモーションを保存している
モーションデータベースの3つから構成されます.\\
今回私が担当し,作成したのは人工知能ハブであり,人工知能ハブについて解説を行いたいと思います.\\

私が担当した,人工知能ハブは大きく分けて3つの要素で構成され,大きな流れで説明をすると
Unityでユーザーが入力した内容をもとに人工知能活用ハブがその入力内容を受け取る.
そして人工知能活用ハブの中で返答する内容が作り出され,モーションデータベースから適切な動作を選択
しUnityへ動作と返答内容を出力する.
この流れによってユーザーはキャラクターとの会話を行うことが可能となっている.
%------------------------------------------------------%
\subsection{アルゴリズムのみを簡単に追加可能な知能ハブ}
それではまずはじめに私が開発する,アルゴリズムのみを簡単に追加することができる
人工知能活用ハブを提案する.
\\
このハブでは作成した会話の返答,もしくはキャラクターの動作を選択するアルゴリズムを簡単に追加し
話題によって追加したアルゴリズムの中から適切なアルゴリズムを用いて返答を行えるようにしている.
\\
例えばゲーム関連の返答アルゴリズムを作り,試したいと考えた場合は,
そのアルゴリズムを実装したプログラムをあらかじめ準備されている抽象クラスを用いて作成し,
\figref{all_kose}の返答アルゴリズム軍のプログラムに作成したプログラムを登録するだけで,
ゲームの話題が来た時にそのアルゴリズムでキャラクターが返答するシステムを作ることができる
というものである.
\\
同様にゲーム関連のキャラクターの動作を選択するアルゴリズムを作る場合は,そのアルゴリズムを
抽象クラスを用いて実装することが可能であり,
\figref{all_kose}の動作選択アルゴリズム軍の中に作成したプログラムを登録するだけで,ゲーム関連
の会話をしている最中は,そのアルゴリズムを用いて動作を決定する仕組みを作ることができるというものである.
\\
これらのアルゴリズムや人工知能が一つだけ実装されており,何も追加知能がない場合はその
デフォルトアルゴリズムが選択されるように設計し,複数の料理の話題に特化した話題解析アルゴリズム
やゲームの話題に特化した感情解析のアルゴリズムが実装されることでより正確な解析が可能になるだけで
はなく,返答する際もゲーム専用の返答アルゴリズムなどがあることでより円滑なコミュニケーションが
可能になるような構成を提案する.

また,様々なアルゴリズムが必要になることを考え,複数人で開発を行った際にも解析情報のデータベースによる
共有などにより,よりスムーズに連携を行うことができるほか,
人工知能ハブでは,すでに解析した感情情報などの情報は全てデータベースによって共有され,
ユーザーの入力した情報の解析を行うプログラムの開発は行わずに,すでにある感情解析プログラムの
解析結果を使ってユーザーの感情状態を考慮した「会話ボット」などの開発を行うことも可能にする.\\
%------------------------------------------------------%
\subsection{作成したアルゴリズムをUnityですぐに試せる機構}
この人工知能利用フレームワークの人工知能活用ハブに登録された知能はUnity上でのキャラクターとの
対話ですぐに試すことができる.\\

共同研究者の藤井克成の論文\cite{fuji}によると,MMDモデルを利用しているため好きなキャラクターで動作させることが
出来,また,この人工知能利用フレームワークのために開発したリアルタイムに動作を保管しながら
動かす技術により,よりリアルな円滑なコミュニケーションが可能になっている.\\

このように作成した人工知能をすぐにキャラクターとの対話という形で実行することができるため,
入出力をどのような設計にするかや,開発はどうするかに迷うことなく,独自の対話アルゴリズムや
人工知能の開発に専念することが可能になり,より高精度な対話を実現できると考え,
この人工知能利用フレームワークを提案する.\\

%------------------------------------------------------%
\subsection{Unityが利用可能なモーションを追加する機構}
この人工知能利用フレームワークでは現在会話と動作の2つの出力を実装している.\\

ここで返答パターンは文字列で生成され,Unityで実行されるので様々なバリエーションで返答すること
ができるが,動作(モーション)はその場で動的にプログラムを用いて動作を定義し,
モーションファイルを生成することが難しいのが現状である.\\

そのため,あらかじめモーションデータを作成しておく必要性がありますが,
そのモーション(動作)を定義するファイルを生成することは一般的には難しいと考えられ,また手間もかかる,
そこで共同開発の鈴木智博がKinectで動作を定義し,データベースに保存,人工知能活用ハブと通信可能な
プログラムを開発した\cite{suzuki}.\\

この機構があることによって,人工知能活用ハブの中で新しい動きのパターンを追加したいとなった時にも
すぐにKinectを用いて自ら追加したい動作をキャプチャすることで動作ファイルが生成され,
データベースに登録することで使えるようなる.



%------------------------------------------------------%

\chapter{設計}
%------------------------------------------------------%
%- 構成
%- プログラムが実際どういう関連性があるのかについて少し細かく説明する
%------------------------------------------------------%
知能ハブの構成を3段階に分けてと,Unityのキャラクターとの連携,
解析や出力をするアルゴリズムを選定する時に利用しているGoogkeAPIの流れで,
人工知能利用フレームワークの構成を解説します.
\\
最初に人工知能ハブの全体的な解析から出力の流れを示す図を以下の  に提示します
%------------------------------------------------------%
\section{入力された情報を解析する機構}
まず初めにUnity上のキャラクターへユーザが発言し,知能ハブが受け取った情報を解析する際の機構に
ついて解説したいと思います.\\

以下\figref{analyze_chart}に入力された情報が解析され,解析結果がデータベースに格納されるまでの
構成を示す.

\figPst{100}{analyze_chart}{解析が行われるまでの図}
\figref{analyze_chart}の大まかな流れを説明すると,初めにユーザーがUnity上で動作している
キャラクターに話しかけると,その内容をUnityが聞き取り,知能ハブへと送信します.
送信した情報は知能ハブで受け取られ,その入力された情報は,
それぞれ解析したい情報ごとに作られているハブへと渡されます.\\
\figref{analyze_chart}の場合,感情を解析する感情解析知能ハブと話題を解析する話題解析知能
ハブがあるため,入力された情報はこの2つの解析ハブへと送信されます.\\

情報が送信されると各解析ハブは入力された情報をもとに,登録されている各解析アルゴリズムの中から
もっとも適切な解析アルゴリズムを選択し,実際の解析を行わせます.\\

実際に解析された情報は知能ハブ全体で共有されているデータベースへ保存することで,様々な場所から
利用することができるようになります.\\

それでは以下の章で解析知能ハブの中のそれぞれの機能について説明したいと思います.


\subsection{解析する情報別にアルゴリズムを保持する機能}
\figref{analyze_chart}を見て分かる通り,入力された情報は各解析する情報ごとに入力データを
渡していきます.\\
そして,各解析知能は入力された情報からその物事を解析するためのアルゴリズムを複数持っており,
それを具体例を用いて表した図を以下の\figref{feel_analyze}に示します.

\figPst{100}{feel_analyze}{感情解析知能ハブとそれに付随する解析アルゴリズム}

具体的に説明をすると,\figref{feel_analyze}の感情解析知能ハブは,感情を解析するための
複数のアルゴリズムを持っていることになります.\\

各,感情解析知能ハブや話題解析知能ハブなどはアルゴリズム型の配列を持っており,
その配列内に解析アルゴリズムの抽象クラスを実装したものを格納するだけで複数のアルゴリズムを保持し,
適切なアルゴリズムが選択されるように設計されています.\\

また,解析する情報である感情や,話題といった種類はその他にも抽象クラスを実装し,
解析知能ハブに登録することで追加することができます.\\


\subsection{会話の話題別に解析するアルゴリズムを選ぶ機能}
この人工知能ハブでは現在話している話題をもとに,どの解析アルゴリズムを選択するかを判定しています.\\
なので料理に関する話題をしているときは,料理関連の単語や会話に対応した解析アルゴリズムがあれば
それが解析を行い,ない場合はその他の解析アルゴリズムの中でもっとも適した解析アルゴリズムが
解析を行う設計になっています.\\
この話題を推定する際にはgoogle検索のAPIを用いており,入力された内容を検索にかけてその結果から
話題を推定しています.\\
こうすることで例えば,以下の\figref{kuppa_down}「クッパ
	\footnote{クッパ:ゲーム「スーパーマリオブラザーズ」に登場する敵キャラクター}
が落ちた」という入力があったときに「クッパ 落ちた」で検索をした結果を取得します.

\figPst{100}{kuppa_down}{2015年12月20日現在の「クッパ 落ちた」のGoogle検索結果}

\figref{kuppa_down}の上位5件の検索結果を見るとゲームの話題であると判定され,ゲームに特化した
感情解析を行うアルゴリズムが選択されます.\\

また,この時に「クッパ
	\footnote{クッパ:クッパは韓国料理の一種。 スープとご飯を組み合わせた雑炊のような料理}
って美味しいよね」という入力があった場合,以下の\figref{kuppa_umai}の「クッパ 美味しい」
の検索上位5件を見て分かる通り,韓国料理のクッパの話題となるため料理に特化した感情解析を行う
アルゴリズムが選択されます.
\figPst{100}{kuppa_umai}{2015年12月20日現在の「クッパ 美味しい」のGoogle検索結果}

このようにその時々に合わせて,適切な解析を行うアルゴリズムが選択されるような構造があり,これによ
って,より高精度な解析を行うことができます.\\

もし,このような機能がない場合,「クッパが落ちた」という文章は「落ちた」というキーワードから
,たとえクッパという単語が料理名だと判明しても,ゲームの敵キャラクターとわからない限りは
マイナスイメージな文と解析されると推測できます.\\

また,この両方を適切に解析できる,つまり現在の話題に限らず,感情を解析できるアルゴリズムを作成
した場合は,そのアルゴリズムのみを感情解析知能ハブに登録することで確実にそのアルゴリズムが解析
を行うように設定することが可能です.

\subsection{解析アルゴリズムを簡単に追加する機能}
実際に解析を行うアルゴリズム自体を簡単に追加する機能について解説します.\\
この,実際に解析を行うアルゴリズムの実装はあらかじめ定義されている抽象クラスを実装することで
完了し,その実装の手順もソースコードの行数に換算すると最短3行でアルゴリズムを追加することが可能
です.\\

\figref{feel_analyze}のゲームプレイ時における感情解析知能を追加したい場合は,抽象クラスを
実装後,感情解析知能ハブにある抽象クラス型を保持する配列に対して,
作成した抽象クラスを拡張したプログラムを入れることで実装を行うことができます.

話題を解析する知能ハブが親の抽象クラスを実装したもので,それに付随する解析アルゴリズムは子の抽象
クラスを実装したものという構図になり,その関係性を以下の\figref{analyze_abs}に示します.

\figPst{100}{analyze_abs}{抽象クラスの関係と抽象クラスの実装例}

\figref{analyze_abs}の通り,親の抽象クラスだけでなく,子の抽象クラスに関しても簡単に実装を
行い,追加する機構がある.

%------------------------------------------------------%
\newpage
%------------------------------------------------------%

\section{解析結果を保存する機構}
この人工知能ハブには解析を行った情報やその他の様々な情報を保存するためのデータベースクラスが実装
されている.\\

そして,このデータベースクラスはすべてのクラスで共有で利用できるように,すべての解析知能や出力
を作成する知能の抽象クラスに含まれている.\\

\subsection{解析情報を保存する機能}
このデータベースは,解析した情報を保存する機能がある.\\
しかし,情報を保存するにあたり,解析を行うアルゴリズムを作る人によって解析結果の形式が異なる
ことが予想できるため,どのようなオブジェクトでも保存が可能なようにobject型を利用している.\\

実際に保存を行う場合は各解析アルゴリズム内でデータベースオブジェクトのメソッドに対して保存したい
内容を引数で渡すだけで保存を行うことができる.\\

保存する際に付けられる名前は明確性と同一名のデータが存在しないように,その解析アルゴリズムの
プログラム名+データの形式という形で保存する.\\

例えばMode-Topic-Gameというゲーム話題解析知能が文字列で話題を保存したい場合はそのアルゴリズム
の中で,解析が終わった時にデータベースオブジェクトの保存を行うメソッドに対して値を渡す.\\

そして,その保存した情報に対してMode-Topic-Game-Stringという名前をつけることで明確性と
同一名のデータが存在しないようにしている.\\

\subsection{解析情報を取得する機能}
その次に解析した情報を出力内容を作成するアルゴリズムの中から呼び出す方法について解説します.\\
実際に解析を行う際にはデータベースオブジェクトの情報取得メソッドに対して先ほどの解析情報の保存
で説明した,欲しい情報の名前を指定することでその情報を取得することができる.\\

%------------------------------------------------------%
\newpage
%------------------------------------------------------%


\section{解析情報を元に出力内容を作成する機構}
解析された情報を元にUnityのヤラクターに送信する出力内容を作成する工程についてユーザーがUnity上
のキャラクターとの会話をする例を表した\figref{output}を用いて説明します.

\figPst{100}{output}{出力情報を作成するまでの流れ}

\figref{output}を見て分かる通り出力作成知能ハブにも出力したい情報ごとにアルゴリズムを保持する
機構があり,今回の\figref{output}の場合は会話を行うための会話知能ハブとキャラクターの動作を
選択するための動作選択知能ハブの2つがある.\\

それぞのハブでは入力された情報を元に,解析知能ハブの時と同じように最適なアルゴリズムを選択します,
選択されたアルゴリズムはそれぞれデータベースにある,利用したい情報を取得し,返答内容や動作を決定
します.\\

それぞれのアルゴリズム処理結果はUnityへの命令形式であるJSON形式にまとめられ,
websocket通信でUnityへと送信され,Unityが命令を解釈,キャラクターが動作するという流れになります.


\subsection{返答を行うタイミング}
人工知能ハブが返答を行うことができるタイミングは2種類あります.\\
1つめは相手から入力があった場合の返答,2つ目が自ら発言する場合に返答する場合です.\\
\\
相手から入力があった場合の返答はUnityから情報が送信されてきた時に出力を作成する知能ハブを呼び出
すことで出力内容を作成し,返答を行っています.\\

自ら発言を行う場合は実装を行ってあるタイマーを利用して発言を行います.
特定の時間や変数の値になった時に発言を行うように設定することが可能であり,出力内容作成知能ハブの
自発的に発言する出力内容を作成するメソッドをそのタイミングで呼び出すことが可能になっています.\\
\\
感情値や時間経過,状況の変化のあった時にwebsocketを用いて通信をUnityへ送信できるため,
自発的に発言しているように見せることが可能です.\\

また,自ら発言する場合と,返答を行う場合でアルゴリズムが異なることが多いため,\figref{analyze_abs}
のアルゴリズムを実装するための子の抽象クラスには返答する際のアルゴリズムと自ら発言する際の
アルゴリズムを書くメソッドが用意されています.\\

\subsection{会話の話題別に返答アルゴリズムを保持する機能}
返答を行う際も,解析を行うときと同様に話題別に返答アルゴリズムを保持しています.\\

また,返答アルゴリズムを保持する仕組みに関しても同じく,返答アルゴリズム型の配列を持っており,
その配列内に返答アルゴリズムの抽象クラスを実装したものを格納するだけで複数のアルゴリズムを保持し,
適切なアルゴリズムが選択される部分についても同じ仕組みで動いています.\\

\subsection{会話の話題別に返答アルゴリズムを選ぶ機能}
出力情報ごとにアルゴリズムを保持しているため,解析知能ハブの時と同じように,その時々
に合わせて最適なアルゴリズムが解析を行うようになっています.


%------------------------------------------------------%
\newpage
%------------------------------------------------------%


\section{作成した知能をUnityで試す機構}
作成したアルゴリズムをすぐに実行し,試す環境として今回は統合開発環境を内蔵し、
複数のプラットホームに対応するゲームエンジンであるUnityを採用しました.\\

このゲームエンジンを用いることでウェブブラウザ上で動作するキャラクターを簡単に作成することが
でき,ブラウザ上で動作するため,様々なプラットフォームで試すことができます.\\

また,ブラウザを搭載していないデバイスの場合でもUnity自体が複数のプラットフォームに
対応しているため,様々な人が開発したアルゴリズムをすぐに試すことができます.

\subsection{UnityWebPlayerでの出力について}
今回作成した人工知能利用フレームワークではUnityWebPlayerを用いてブラウザ上でキャラクターとの
コミュニケーションを取れるように設計しました.\\

\figPst{100}{unity}{Unity Web Playerによるキャラクターの表示画面}

\figref{unity}のようにブラウザを搭載しているPCやmacなどのデバイスならばキャラクターを表示する
ことが出来,作成したアルゴリズムをすぐに試すことが可能です.\\

\subsection{Unityとの連携に利用するWebSocket}
Unityとの通信にはWeb Socketを用いています.\\

web socketとはウェブサーバーとウェブブラウザとの間の通信のために規定を予定している双方向通信用
の技術規格であり,それを採用した理由としてあげられるのが任意のタイミングでのpush通知が可能な点です.\\

push通知が可能になることによって,人工知能利用フレームワークから好きなタイミングで命令を送信し,
Unity上のキャラクターを動作させることができるようになるだけではなく,Unity側のプログラムとしても
命令がきた時にだけキャラクターを動作させ,ユーザーから入力があった時だけサーバへ入力情報を送信
すればよいので処理が軽減されるという利点もあります.\\

\subsection{Unityへの送信フォーマットと作成}
\figref{output}の「返答内容をJSON形式にまとめてpush送信する」という部分の解説を行います.\\
このUnityへの送信は汎用性の高いJSON形式\footnote{JSON形式:軽量なデータ記述言語の1つ}を用い
て送信を行っています.\\

このJSON形式のデータを実際に作成しているのは\figref{output}の出力知能作成ハブであり,
各それぞれの返答内容作成知能ハブや動作選択知能ハブから受け取った情報をまとめてJSON形式
にしています.\\

そして,以下に実際にやり取りを行っているJSON形式のフォーマットを示します.\\

(((((((((なんか票を入れる))))))))))))))

\subsection{Unityからの受信フォーマット}
Unityから情報を受け取る際にもJSON形式を用いており,


(((((((((なんか票を入れる))))))))))))))

%------------------------------------------------------%
\newpage
%------------------------------------------------------%

\section{アルゴリズムを選定する際に用いるGoogleAPI}
\subsection{GoogleAPIについて}
\subsection{GoogleAPIの有効性}


%------------------------------------------------------%
\chapter{実装}
%------------------------------------------------------%
%- 実装
%------------------------------------------------------%
\section{開発環境}
\subsection{Javaの利用}
今回の開発では,実行が高速かつオブジェクト指向が今回開発する人工知能フレームワークに適していると
判断したためJavaを用いて開発を行った.
また,当研究室に所属する学生はJavaの開発に慣れており,学習コストが低いため採用した.

\subsection{Mavenフレームワーク}
Unityとの通信を行うためMavenフレームワークを用いて開発を行った.
ライブラリのバージョン管理をソースコードで行える利点もある.


%------------------------------------------------------%
%インプットコントローラー
%------------------------------------------------------%
\section{解析部分の実装}
\subsection{解析コントローラー}
解析コントローラーでは,「話題解析」や「感情解析」と言った解析を行う分野自体を管理するクラスである.
今回はその解析分野\footnote{解析分野:話題を解析する場合の解析分野は話題であり,感情を解析する場合の解析分野は話題となる.}
別に,解析アルゴリズムを保持する各解析知能ハブ
\footnote{話題解析知能ハブや感情解析知能ハブなどの解析分野別に作成されるアルゴリズムを複数所持するクラス}
を作るためにInputControllerを作成した,
実装した主要なメソッドなどを以下の\tabref{InputController}に示す.

\begin{table}[tbh]
	\caption{実装した主要構成要素} \label{tab:InputController}
	\begin{center}
		\begin{tabular}[htb]{c|c}
		\hline
		コンストラクタ & 各解析知能ハブを登録する \\
		\hline
		InputData & 入力があった時に各解析知能ハブへデータを渡す \\
		\hline
		\end{tabular}
	\end{center}
\end{table}


\tabref{InputController}のコンストラクタでは「感情」や「話題」などの各解析分野を登録する処理を行う.
その部分のソースコードを一部抜粋したものを以下の\srcref{inCnt}に示す.

\srcPst{Java}{inCnt.java}{inCnt}{新しい解析分野を登録する際のソースコードの一部}

\srcref{inCnt}の6行目8行目10行目で解析分野が,この解析知能ハブに登録されていることがわかる.

\tabref{InputController}のInputDataメソッドではユーザーから入力があった時に入力された情報
を各解析分野の実装済みクラスに値を渡す.
その情報を用いて各解析知能ハブはそれぞれの解析分野にあった内容を解析する.
今回の場合は\tabref{InputController}の20行目で実際に各解析知能ハブの解析をおこなうメソッドを呼び出している.

%------------------------------------------------------%
%親の抽象クラス
%------------------------------------------------------%
\subsection{解析知能ハブ}
初めに解析知能ハブとは,感情解析知能ハブや話題解析知能ハブに当たるものであり,
各ハブはそれぞれ様々な解析アルゴリズムを所持しているものである.

解析する話題別にアルゴリズムを保持するために,Abstract Mode(以下親抽象クラス)という抽象クラスを
実装したこれが解析知能ハブの抽象クラスである.
親抽象クラスには解析する情報ごとに,プログラムを保持するための機構が記述されており
この親抽象クラスに実装した主要なメソッドなどを次の\tabref{Abstract Mode}に示す.

\begin{table}[tbh]
	\caption{Abstract Modeに実装した主要構成要素} \label{tab:Abstract Mode}
	\begin{center}
		\begin{tabular}[htb]{c|c}
		\hline
		init & 初期化を行うメソッド \\ \hline
		getAnalyzeParts & 解析アルゴリズムを選択するメソッド \\ \hline
		analyzeChat & 入力があった際に解析を行わせるメソッド \\ \hline
		\end{tabular}
	\end{center}
\end{table}


\tabref{Abstract Mode}のinitメソッドでは各解析アルゴリズムが保持している話題
を,独自実装を行ったGoogleAPIに渡すことで頻出単語表を作成し,その検索結果の頻出単語表を取得している.

\tabref{Abstract Mode}のgetAnalyzePartsメソッドでは,実際に解析を行うアルゴリズムを生成された
頻出単語をもとに決めるメソッドである.
具体的には各解析アルゴリズムごとに生成された頻出単語表
\footnote{頻出単語表:検索結果の文字列から固有名詞だけをのこし,出現した単語とその数をカウントしたもの}
とユーザーが入力した文章をGoogleAPIを用いて作成した頻出単語表を比較して,
もっとも似ている頻出単語表を持つ解析アルゴリズムが,解析を行うという実装になっている.
%------------------------------------------------------%
%子の抽象クラス
%------------------------------------------------------%
\subsection{解析アルゴリズムの各解析知能ハブへの追加}\label{sec:anaAdd}
人工知能利用フレームワークの,解析アルゴリズムを簡単に実装する構成について説明する.
各解析知能ハブに対して,アルゴリズムを追加実装するために用いる,抽象クラス「Abstract Mode Parts」
(以下,子抽象クラス)を実装した.

この子抽象クラスには,親抽象クラスであるAbstract Modeを実装したクラスから解析する際に呼び出されるメソッドや
データベースなどとの連携が記述されている.
アルゴリズムを新規に追加し試したい場合,これらの部分については追記する必要がない点がメリットである.

以下の\tabref{Abstract Mode Parts}に実際に解析知能を作るために,必要な抽象クラスである
Abstract Mode Partsに実装した主要な要素を示す.

\begin{table}[tbh]
	\caption{Abstract Modeの主要構成要素} \label{tab:Abstract Mode Parts}
	\begin{center}
		\begin{tabular}[htb]{c|c}
		\hline
		クラス変数 & 保存を行うための変数が定義されている \\ \hline
		コンストラクタ & アルゴリズムの分野を記述する場所 \\ \hline
		ChatAnalyze & アルゴリズムを記述する部分 \\ \hline
		saveData & 解析結果を保存 \\ \hline
		\end{tabular}
	\end{center}
\end{table}

\tabref{Abstract Mode Parts}のクラス変数は,解析した情報を保存するための変数である.
解析結果をクラス変数に入れることで,処理が終わった後に適切な形式でデータベースに自動で保存される.

\tabref{Abstract Mode Parts}のコンストラクタでは,その解析アルゴリズムの分野を記述する
必要性がある.
具体的にはaboutというString型の変数に話題を入れることで,
その話題をユーザーが話した時にそのアルゴリズムが選択されて返答されるという構造が実装される.
その部分のソースコードを一部抜粋したものを以下の\srcref{about}に示す.

\srcPst{Java}{about.java}{about}{コンストラクタで話題を設定するソースコードの一部}

\srcref{about}は料理に関する話題を解析するプログラムの,コンストラクタを抜粋したものだが,
ここで1行目のプログラムを記述する.

\tabref{Abstract Mode Parts}のChatAnalyzeは,実際に解析アルゴリズムを書く部分である.
ここで入力された内容が,String型で引数として渡されてくるのでその内容を用いて解析を行う.
解析した情報はあらかじめ定義してあるクラス変数に保存することで,自動でデータベースに格納されるようになっている.

\srcPst{Java}{anaAlgo.java}{anaAlgo}{解析アルゴリズムを記述するメソッドのソースコードの一部}

\srcref{anaAlgo}は解析を行うアルゴリズムを記述するメソッドのみを抜粋したもので,
ここでデータベースに保存する情報を解析し,解析した情報を変数に格納するプログラムを記述する.

次にデータを保存する際に用いるメソッドについて解説を行う為,継承元クラスのソースコードを一部抜粋したものを
以下の\srcref{anaParts}に示す.
\srcPst{Java}{anaParts.java}{anaParts}{継承元クラスから一部抜粋した定義済みのメソッド}

\srcref{anaParts}のsaveDataは,継承元のクラスに定義されているメソッドである.
このメソッドは全ての解析が終わった後に親抽象クラスから呼び出され,
変数の中に値が入っていた場合のみその内容をデータベースにそのクラス名+データ型の名前をつけて保存する.
最後にこの抽象クラスを拡張して作成したクラスは,親クラスであるAbstract Modeを実装した
感情解析知能ハブや話題解析知能ハブなどの親抽象クラスに登録することでアルゴリズムの追加が完了する.

%------------------------------------------------------%
%現在実装しているアルゴリズム
%------------------------------------------------------%
\subsection{現在実装している解析アルゴリズム}
現在実装している解析プログラムは2種類あり,感情の解析と話題の解析である.
話題の解析に関しては例えばゲームの話題解析知能を作った場合,さらに細かい「どのゲームか」ということや
「どんなシーンなのか」などをさらに詳しく解析することを目的に作成した.

%------------------------------------------------------%
1つ目に感情の解析を行うアルゴリズムの実装について説明する.
感情の解析を行うプログラムは,親抽象クラスである感情解析知能ハブに所属する解析知能の1つである.
現在感情の解析を行うプログラムは1つなため,必ずこのアルゴリズムがデフォルト解析アルゴリズムとして,
選ばれるようになっている.

具体的な解析を行うアルゴリズムに関しては「哀れ」「恥」「怒り」「嫌」「怖い」「驚き」「好き」「高ぶり」「安らか」「喜び」
の10種類の感情に分類して感情の解析を行なっている.
このそれぞれの感情にはその感情に対応する単語が付いており,入力した文章の中にその単語があった時にその感情値
に1を加えて数字でその感情に関する文字列が出てきた回数を表現する仕組みになっている.

%------------------------------------------------------%
2つ目の話題を解析する知能では料理とゲームに関する話題を解析するプログラムが実装してあり,
料理の分野では「作る」「食べる」「片付ける」の3つの話題にさらに細かく解析する仕組みがある.
また,ゲームの分野では「戦闘」「負け」「勝利」の3つの話題にさらに細かく解析する仕組みを実装している.

%------------------------------------------------------%
%データベース実装
%------------------------------------------------------%

\section{データベースの実装}
\subsection{全ての解析情報を保存する機構}
まず初めにデータベースは独自実装したデータベースクラスを用いて実現した.
データベースの実装では様々な解析情報を保存する必要があるため,複数の変数型に対応するために,HashMap
のキーをStringにし,保存する値であるvalueをObject型に指定した.

このデータベースクラスにはデータを保存するためのメソッドが用意されており,以下にソースコードから
一部抜粋したものを示す.

\srcPst{Java}{dbIn.java}{dbIn}{データベースの解析結果を保存するメソッドの一部}

\srcref{dbIn}のsetDataメソッドは先ほどの\ref{sec:anaAdd}で説明したsaveDataメソッドから呼び出されて利用される.
保存を行う際にはデータの重複が起きないようにクラス名+データの形式をデータに対応するキー
\footnote{データを取得する際に指定する文字列}としている.

\subsection{解析した情報を取得する機構}
解析した情報を取得する際はこのデータベースクラスのgetDataメソッドを呼び出す.
以下の\srcref{dbOut}にgetDataメソッドのソースコードを示す.

\srcPst{Java}{dbOut.java}{dbOut}{データベースの解析結果をするメソッド}

\srcref{dbOut}の2行目を見ると,データを取得する際にString型の鍵が必要であることがわかる.
このString型の鍵はデータベースの中にあるデータを保存しているHashMapの鍵を示しており,取得したい
情報の鍵をshowDataというメソッドを用いて情報を取得するための鍵を調べ,その鍵を用いて情報を取得する.

%------------------------------------------------------%
%OutputController
%------------------------------------------------------%

\section{出力を行う知能ハブの実装}
Unityへ命令の送信を行う部分の実装について説明する.

\subsection{出力コントローラー}
出力する情報をまとめ,JSON形式などを成形する出力コントローラーについて解説する.
出力コントローラーに,実装した主要な構成要素の一覧を以下の\tabref{OutputController}に示す.

\begin{table}[tbh]
	\caption{実装した主要な構成要素} \label{tab:OutputController}
	\begin{center}
		\begin{tabular}[htb]{c|c}
		\hline
		コンストラクタ & 各出力知能ハブを登録する \\
		\hline
		getJson & Unityへ出力情報を送るときに呼ばれるメソッド \\
		\hline
		getTimeAction & キャラクターが自発的に発言する際に用いられるメソッド \\
		\hline
		\end{tabular}
	\end{center}
\end{table}

\tabref{OutputController}のコンストラクタの部分では,各出力知能ハブをこのクラスに登録している.
登録された出力知能ハブはgetJsonメソッドが呼ばれたときに出力内容を作成するようになっている.

\tabref{OutputController}のgetJsonメソッドでは,Unityから入力があった際に呼ばれるメソッドであり,
あらかじめ登録されている出力知能ハブの,出力を作成するメソッドを呼び出す実装になっている.

\tabref{OutputController}のgetTimeActionメソッドでは,時間経過に応じてキャラクターが発言する
設定を有効にしているときに呼び出されるメソッドであり,このメソッドが呼ばれると各出力知能ハブの
自発的に発言する際に用いるメソッドを呼び出すことで応答を行う.

%------------------------------------------------------%
%出力知能はぶ Abstract_Mode
%------------------------------------------------------%
\subsection{出力知能ハブ}
出力する動作や返答内容といった出力情報別に,アルゴリズムを保持する機構について解説する.
この機構も情報解析の時と同じ仕組みで構成されており,
この出力の情報別に保持する機構についてもAbstact Modeという親抽象クラスが作成されている.
その抽象クラスを実装することで返答知能ハブや動作選択知能ハブといったアルゴリズムを複数所持するクラスを新規に作成することが可能である.

以下の\tabref{abstractmode}に出力知能ハブを作成するために必要な抽象クラスである
Abstract Modeの主要構成要素を示す.\\

\begin{table}[tbh]
	\caption{実装した主要な構成要素} \label{tab:abstractmode}
	\begin{center}
		\begin{tabular}[htb]{c|c}
		\hline
		init & 初期化 \\
		\hline
		getOutput & 返答内容の作成 \\
		\hline
		getTimerAction & キャラクターが自発的に発言する際に用いられるメソッド \\
		\hline
		\end{tabular}
	\end{center}
\end{table}

\tabref{abstractmode}のinitメソッドでは初期設定を行っており,独自実装をおこなったGoogleの検索結果データベースを
最新の情報に更新をする処理を行っている.

\tabref{abstractmode}のgetOutputメソッドでは,キャラクターにユーザーが話しかけてきたときに,返答内容を
作成するアルゴリズムから出力内容を取得し返答を行う.
返答するアルゴリズムを選択する機構もこの部分にあり,
アルゴリズムの選択はユーザーが発言した内容から作成した頻出単語表と各解析を実際に行う
アルゴリズムが持っている話題をもとに作成したHashMapを比較して,もっとも頻出単語表が似ているものを選ぶという仕組みになっている.
そのソースコードから一部抜粋したものを\srcref{getOutput}に示す.

\srcPst{Java}{getOutput.java}{getOutput}{getOutput.javaのソースコードの一部}

まず初めに,\srcref{getOutput}の8行,9行目で最新の発言情報を取得し,10行目で発言内容をGoogleAPIに
渡すことで頻出単語表を作成する.
次にあらかじめ作成してある各解析アルゴリズムごとの,話題単語の頻出単語表とGoogleAPIを用いて取得した頻出単語表を
11行目から25行目にかけて比較し,一番最適な解析アルゴリズムを選択している.
最後に45行目にて選択された解析アルゴリズムに解析を行わせ,解析結果をそのまま返している.

\tabref{abstractmode}のgetTimerActionメソッドは時間経過に応じて反応するときに呼び出され,
実際に出力内容を作成するアルゴリズムに対して,自発的に発言する際の出力内容を作成させて取得する.

\subsection{出力アルゴリズムの出力知能ハブへの追加}
実際に出力情報を作成するためのアルゴリズムを記述するプログラムを,簡単に出力知能ハブへ追加するために
出力専用のAbstract Mode Partsという子抽象クラスを作成した.
その抽象クラスを用いることで3行プログラムを書くだけで新しいアルゴリズムを追加できるようになっている.

それではまず初めに,その抽象クラスに実装した以下の\tabref{parts}に示した主要なメソッドについて解説する.

\begin{table}[tbh]
	\caption{実装した主要なメソッド} \label{tab:parts}
	\begin{center}
		\begin{tabular}[htb]{c|c}
		\hline
		コンストラクタ & 担当分野の設定 \\
		\hline
		Action & 返答アルゴリズムの \\
		\hline
		TimeAction & キャラクターが自発的に発言する際に用いられるメソッド \\
		\hline
		dataRefresh & 常にデータベースを最新に保つためのメソッド \\
		\hline
		\end{tabular}
	\end{center}
\end{table}

\tabref{parts}のコンストラクタでは,出力を行う際に担当する分野や話題について記述する部分である.
解析を行う際と同じように変数aboutに対して適切な担当する話題名を入れることで,GoogleAPIを用いて
その話題名に関する頻出単語表が自動で生成される.
その生成された頻出単語表も解析の時と同じく先ほど説明したアルゴリズムの選定に利用される.
以下の\srcref{aisatu}に挨拶の分野を指定する場合のプログラムを一部抜粋したものを示す.

\srcPst{Java}{aisatu.java}{aisatu}{話題を指定する際のサンプルソースコードの一部}

\srcref{aisatu}の3行目では話題を挨拶に指定しており,ユーザーが挨拶と関係のある単語を発話した時にこのアルゴリズムが
選択され,実際に返答内容を作成するようになるように実装されている.

\subsection{現在実装している出力アルゴリズム}\label{sec:back}
現在実装している出力知能ハブは2つあり,会話を行う知能ハブと動作選択を行う知能ハブの2種類である.
会話を行う知能ハブには2つのアルゴリズムが搭載されており,料理とゲームの話題に関するアルゴリズムを実装した.

料理のアルゴリズムでは,解析した時に取得した作る,食べる,片付けるの状態を用いて,返答を行う.

ゲームのアルゴリズムでは戦闘,負け,勝利の状態を解析しているのでそれを用いて返答を行っている.

次に動作を選択するアルゴリズムでは,共同開発の鈴木さんのデータベースから動作一覧とその1つ1つの動作に関係する単語,
及び関係する単語をGoogleAPIをもちいて頻出単語表を作ったものを取得する.

動作を選択するアルゴリズムはその動作に関連付けられている頻出単語表と,ユーザーの発言内容から作成した
頻出単語表を比較して,最も関連性のあるモーションを選択するようになっており,
その具体的なアルゴリズムや通信に関しては\ref{sec:motion}に記述する.
%------------------------------------------------------%
%Unityとの通信に関して
%------------------------------------------------------%

\section{Unityとの通信の実装}
\subsection{通信方式}
Unityとの通信にはWebSocketを用いており,双方向任意のタイミングでの情報の送受信が可能となっている.

Unityとの情報の送受信を行うために人工知能利用フレームワークの中に送受信を行うためのクラスである,
NewWSEndpointを実装した.
以下の\srcref{endpoint}にソースコードを一部抜粋したものを示す.

\srcPst{Java}{endpoint.java}{endpoint}{WebSocketのJavaサーバー側実装の一部}

\subsection{Unityからの入力情報の受信}
\srcref{endpoint}のソースコードではウェブブラウザ(Unityクライアント)が,
接続を行った時にそのセッションを保存する為に23行目でリストにセッションを格納している.
この実装により,セッションが確立され,受信を行う準備が完了する.

Unityからメッセージが来た場合は\srcref{endpoint}の44行目のprocessUploadが呼ばれる.
33行目には同じく受信するメソッドであるOnMessageというメソッドがあるが,
これはUnity上のキャラクター以外の端末からメッセージを受け取った際に呼ばれるものである.

processUploadメソッドではバイト形式で入力情報を受け取るため,47行目にてバイトをString型に変換している.
変換した後は48行目でその値を入力情報の解析を行う解析知能ハブへ渡している.
その処理が終わった後に49行目で返答するjson形式の出力情報を作成し,51行目で全ての接続クライアントに対して命令を送信している.

最後にこれ以上通信を行わない場合は\srcref{endpoint}の27行目で,セッションが切断された時にリストから削除する処理を行っている.

\subsection{Unityへの命令の送信}
\srcref{endpoint}の51行目で全ての接続クライアントに対して出力知能ハブから得たjson形式の値を
送信している.

人工知能利用フレームワークは家庭で利用されることを目標にしており,
例えばテレビのブラウザでキャラクターと対話,PCの画面でキャラクターと対話,スマホの画面でキャラクターと対話
を行った際に全てのデバイスから同じキャラクターと対話することを実現するために51行目では
全てのクライアントへ対して情報を送信するように実装をおこなった.

%------------------------------------------------------%
%データベースとの通信とモーションデータベースについて
%------------------------------------------------------%

\section{人工知能利用フレームワークに追加したモーションの利用}\label{sec:motion}
\subsection{動作選択アルゴリズムの実装}
先ほど\ref{sec:back}で説明した,動作選択を行う際に用いているアルゴリズムについて解説する.
以下に動作を選択する際に用いている動作選択アルゴリズムを一部抜粋したものを示す.

\srcPst{Java}{motion.java}{motion}{動作選択アルゴリズムの一部抜粋}

まず初めに,\srcref{motion}の3行目に動作選択が行われる際に呼び出される,Actionメソッドが記述されている,
このメソッドでは8行目に記述されたメソッドmatchを呼び出し,その中でキャラクターが行うモーションを選択している.
\srcref{motion}のmatchメソッドでは15行目から17行目にかけてデータベースを管理するコントローラーから
データベースの情報を取得している.
このコントローラーはサーバー上のMongoDBからモーションデータなどの情報を取得している.

また,このコントローラの実装に関しては共同開発の鈴木さんの論文\cite{suzuki}を参照すると,
このコントローラーからデータベースへの接続を行なっていることがわかる.

実際にどの動作を実行するかを判定しているアルゴリズムを解説する.
\srcref{motion}の22行目から39行目を見ると,
そこではユーザーが発言した内容から作成した頻出単語表と,データベースの中にあるモーションごとに関連付けられている頻出単語表を比較している.
その2つの頻出単語表を比較し,一番似ている頻出単語表を持っている動作が選択され,40行目でその動作名が返される仕組みとなっている.
%------------------------------------------------------%
%GoogleAPI
%------------------------------------------------------%
\section{GoogleAPIによる頻出単語表の作成}
今回実装したGoogleAPIでは,Google検索を用いてウェブ上から情報を取得する機能から
kuromojiを用いて形態素解析を行わせる機能,及び頻出単語表を作成する機能と3つの機能から成り立っている.

\subsection{形態素解析による検索ワードの作成}
ユーザーが入力した情報からどのような分野の単語なのかということを調べるにあたり,
その検索する際のキーワードというものは検索結果やアルゴリズムを選択する際の精度に関わるため,非常に重要である.

そこでJavaの形態素解析器であるkuromoji\cite{gitkuromoji}を用いて形態素解析を行い,
適切な検索ワードで検索を行えるようにmwSoft blog\cite{kuromoji}を参照,参考にしつつ実装を行った.

具体的にはkuromojiにはSearchモードというモードがあり,それを利用することで「今日の夕飯何にしよう」を
「今日 夕飯 何 しよ」のように検索で利用しやすい形に分解することが可能である.
このように入力を行った情報をそのまま検索するのではなく,形態素解析を行ってから検索を行うことで
より良い検索結果を取得できると考えられる.

\subsection{GoogleAPIを利用して検索結果を取得}
検索をかける際にはHttpClientを用いて検索を行っている,HttpClientはPOST通信を用いてサーバーへ接続を
行い,XML形式で結果を受け取るもので,これを用いることでgoogleの検索結果をそのまま文字として取得することが可能である.
この機構を実現するためにmwSoft blog\cite{google}を参考にし,これに加えて並列初期化機能などの複数の機能を独自実装した.

以下の\srcref{google}に実装を行ったGoogleAPIのソースを一部抜粋したものを記載する.
\srcPst{Java}{google.java}{google}{GoogleAPIの一部抜粋ソースコード}

\srcref{google}の20行目にてgoogleの検索結果ページのURLを作成し,通信することでGoogle検索の結果の内容を取得している.

\srcref{google}の23行目から37行目にかけて検索キーワードを作成しており,先ほど説明したkuromojiの
検索しやすい単語ごとに区切るSearchモードの機能だけではなく,今回は助詞と助動詞を検索ワードから排除して検索を行っている.

\srcref{google}の47行目では実際にウェブサイト上から情報を取得しており,取得した結果を
GoogleSerchResultAnalyzerクラスを用いて解析を行い,リストに格納している.

\subsection{検索結果のフィルタリング}
検索結果にはどの単語を検索しても必ず含まれる単語が複数ある.
例としては「キャッシュ」といった検索を行った際に表示される文字列や「類似」という単語である.
これらの単語は頻出単語表を作成する上で不要なため,フィルタリングを行っており実際にフィルタリングを行っているのは
\srcref{google}の84行目から108行目である.
84行目から行われているのは記号を取り除く作業であり,頻出単語とはなりえない記号を取り除いている.

\subsection{頻出単語表の作成}
頻出単語表を作成するにあたって独自にWordCounterというクラスを実装した.
以下にWordCounterから一部抜粋したソースコードを記載する.

\srcPst{Java}{word.java}{word}{WordCounterの一部抜粋ソースコード}

\srcref{word}の7行目に定義されているwordcountというメソッドに対して,単語ごとにカウントを行って欲しい
文字列を渡すことで解析を行うことが可能である.
引数としてはborderを設定することができ,1回しか出てこない単語に関してはカウントを行わないという設定を行うことが可能である.

初めにカウントを行うにあたって,入力された文章を形態素解析にかける必要があるため,実装では
\srcref{word}の11行目にてkuromojiを用いた形態素解析を行なっている.
次に形態素解析を行った文章に出てくる単語をカウントする部分については,\srcref{word}の14行目から24行目にて
行っている.
最後にカウントした結果をHashMap形式にし,41行目でそのHashMapを返していることがわかる.
以下の\srcref{words}に例として「今日の夕飯何にしよう」で検索を行った結果をもとに作成した頻出単語表を記述する.

\srcPst{Java}{words.json}{words}{頻出単語表}

\srcref{words}の頻出単語表を見ると国名が多く,その国々の料理の検索結果を取得していることがわかる.
例えば日という単語であれば日=5の様に表記されており,日という単語が5回出てきたことがわかる.
また,そのほかにも料理のレシピを検索することが可能なウェブサービスであるクックパッドのクックの文字を多く取得しているほか,
調味料の名前であるキッコーマンなどの単語も取得することが可能である.

以上の様に検索結果からその検索した単語と同じ様な意味を持つ単語を取得することが可能である.

%------------------------------------------------------%
\chapter{実行結果}
%------------------------------------------------------%
%- 実行結果
%------------------------------------------------------%

\section{Unityの出力画面の図}
実際にUnityでのキャラクターとの対話を行う際の画面は以下のの様な画面で対話を行うことが出来る.

\figPst{100}{doUnity}{キャラクターとの対話画面}

\figref{doUnity}の画面にて実際にキャラクターと対話を行う形式で作成した人工知能のアリゴリズムが
正しく動作しているのかを確認することができる.

%------------------------------------------------------%

\section{実際の会話}
実際にこの人工知能利用フレームワークを用いて会話を行った例を以下の\tabref{Chat}に示す.

\begin{table}[tbh]
	\caption{キャラクターとの対話例} \label{tab:Chat}
	\begin{center}
		\begin{tabular}[htb]{c|c|c}
		\hline
		ユーザーの発言 & キャラクターからの返答 & 動作 \\
		\hline
		カルボナーラ作ってるんだ & 隠し味にチョコレートいれちゃう? & 悪巧みの動作 \\
		うわお!クッパ強い,負けたよ & ゲームオーバーだねー & がっかり \\
		\hline
		\end{tabular}
	\end{center}
\end{table}

以上の\tabref{Chat}の様にユーザーが話しかけた内容に対して返答を行う仕組みが実装されていることがわかる.

また,ユーザーが話しかけた内容の分野を解析し,その分野に対応している解析や出力アルゴリズムが対応しているため,
その分野のより詳しい会話を行うことができるもの\tabref{Chat}を見るとわかる.


\section{アルゴリズムを追加した後の会話}\label{sec:addAl}
実際にアルゴリズムを追加した後に会話を行うとどの様な変化があるかを検証してみたいと思う.

今回追加を行うアルゴリズムの分野はゲームの「スーパーマリオブラザーズ\footnote{『スーパーマリオブラザーズ』(Super Mario Bros.)は,任天堂が発売したファミリーコンピュータ用ゲームソフト.}」(以下,マリオ)

という分野をもつ解析と出力を行うアルゴリズムであり,解析を行う部分の話題を解析する分野に対して,
は簡易的な「マリオ」というゲーム内でよく行われる会話の話題を推定するアルゴリズムを追加した.

出力を行う,返答アルゴリズムにも「マリオ」の話題に対応,特化する出力アルゴリズムを追加した.

このマリオのアルゴリズムを追加した後の会話を以下の\tabref{afterChat}に示す.

\begin{table}[tbh]
	\caption{アルゴリズム追加後の会話} \label{tab:afterChat}
	\begin{center}
		\begin{tabular}[htb]{c|c|c}
		\hline
		ユーザーの発言 & キャラクターからの返答 & 動作 \\
		\hline
		うわお!クッパ強い,負けたよ & きのこを取っておこう! & がっかり \\
		\hline
		\end{tabular}
	\end{center}
\end{table}

この「マリオ」はゲームの分野の中のゲームのタイトル名であり,
ただ単にゲームに関連のある単語であれば,ゲームの分野が解析を行う,
ここで「マリオ」というゲーム固有の単語である「クッパ\footnote{クッパとは、
マリオシリーズに登場するキャラクターであり,敵キャラクター正式名称は「大魔王クッパ」}」
という単語を含めることで,\tabref{afterChat}の様に
マリオの解析と出力を行うアルゴリズムが選択される.

返答内容に関しても「負けたよ」という発言から,同ゲームのプレイヤー強化アイテムである「きのこ」を
取って再度,敵に対して挑戦しようという返答を行うことができていることがわかる.
この様に料理やゲームといった種類を横に広げていくアルゴリズムの追加方法と,
料理やゲームのさらに細かい話題に対して対応させるアルゴリズムの追加方法などがある.


%------------------------------------------------------%
\chapter{結論}
%------------------------------------------------------%
%- 結論
%------------------------------------------------------%

\section{結論}
\subsection{アルゴリズムの追加による出力の変化}
\subsection{Googleを用いた会話の話題推定の精度}
\subsection{簡単にアルゴリズムを追加できたか}


%------------------------------------------------------%

\chapter*{謝辞}
\addcontentsline{toc}{chapter}{謝辞}
%------------------------------------------------------%
%- 謝辞
%------------------------------------------------------%

本研究を進めるにあたり,数々のご意見とご指導をくださった田胡和哉教授と柴田千尋助教に心より感謝いたします.
また,本研究への助言をくださった田胡研究室の皆様と開発をする際に参考にさせていただいた文献を書いた方々に感謝いたします.
加えてインタビューに答えていただいたコメダ珈琲八王子店のスタッフ方々に感謝いたします.


%------------------------------------------------------%
%- References
%------------------------------------------------------%

\begin{thebibliography}{99}
	\bibURL{muno}{wikipedia}{人工無脳 Wikipedia}
	{https://ja.wikipedia.org/wiki/人工無脳}{2015年12月27日}

	\bibURL{tino}{wikipedia}{人工知能 Wikipedia}
	{https://ja.wikipedia.org/wiki/人工知能}{2015年12月27日}

	\bibURL{deep}{wikipedia}{DeepLearning}
	{https://ja.wikipedia.org/wiki/ディープラーニング}{2015年12月27日}

	\bibURL{rinna}{Microsoft}{りんな}
	{http://rinna.jp/rinna/}{2015年1月12日}

	\bibURL{pepper}{Softbank}{pepper}
	{http://www.softbank.jp/robot/consumer/products/}{2015年1月12日}

	\bibURL{line}{line}{line}
	{http://xn--o9jo1npkwb.com/}{2015年1月12日}

	\bibURL{humen}{総務省統計局}{2015年確定日本人人口}
	{http://www.stat.go.jp/data/jinsui/new.htm}{2015年1月12日}

	\bibURL{coffe}{コメダ珈琲}{コメダ珈琲八王子北口店}
	{http://www.komeda.co.jp/search/shopdetail.php?id=716}{2015年1月12日}

	\bibURL{ota}{Wikipedia}{オタクとは}
	{https://ja.wikipedia.org/wiki/おたく}{2015年1月12日}

	\bibURL{dip}{ディップ株式会社}{日本のオタク人口}
	{http://www.dip-net.co.jp/news/press-release/2014/04/4075.html}{2015年1月12日}

	\bibURL{anime}{株式会社メディア開発綜研}{アニメ市場調査}
	{http://www.mdri.co.jp/review/}{2015年1月12日}

	\bibURL{hime}{Twitter}{hiyokunohaneさんの投稿}
	{https://twitter.com/hiyokunohane/status/579306999357157376}{2015年1月12日}

	\bibURL{boom}{東京工科大学}{東京大学松尾 豊}
	{http://ymatsuo.com/japanese/}{2015年12月27日}

	\bib{fuji}{藤井克成}{Unityによるキャラクターの制御}{2015年度卒業論文}

	\bib{suzuki}{鈴木智博}{モーションデータベースの開発}{2015年度卒業論文}

	\bib{latex}{奥村晴彦 著}{\LaTeXe 美文書作成入門 改訂第3版}{技術評論社 2004, 403pp}

	\bib{latex}{マルチェッロ・マッスィミーニジュリオ・トノーニ 著}{\LaTeXe 意識はいつ生まれるのか}{亜紀書房}

	\bibURL{google}{mwSoft blog}{HttpClientとHttpCleanerでGoogle検索結果を解析する例}{http://blog.mwsoft.jp/article/34841195.html}{2015年8月12日}

	\bibURL{kuromoji}{mwSoft blog}{Java製形態素解析器「Kuromoji」を試してみる}{http://www.mwsoft.jp/programming/lucene/kuromoji.html}{2015年8月12日}

	\bibURL{gitkuromoji}{github}{形態素解析器kuromoji}
	{https://github.com/atilika/kuromoji}{2015年8月12日}

\end{thebibliography}

\end{document}
